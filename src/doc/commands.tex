\documentclass[11pt]{book}


\usepackage{multind}
\usepackage[pdftitle={POY 5.0 Documentation},pdfauthor={Andres Varon et. al.},
pdfkeywords={phylogenetic analysis, direct optimization, POY}, color links, linkcolor=blue, urlcolor=blue]{hyperref}
\usepackage{color}
%\usepackage{ctable}
\usepackage{rotating}
\usepackage{color, soul}
\usepackage{xspace}
\usepackage{framed}
\usepackage{lipsum}
\usepackage{graphicx}
\usepackage{microtype}
\usepackage[htt]{hyphenat}
%\usepackage[T1]{fontenc}
\usepackage {marvosym}
\usepackage{tabularx}
\usepackage{verbatim}
\usepackage{amssymb,amsmath}
\usepackage{rotating} % to create landscape pages
%\usepackage{makeidx}

%Indexes
\makeindex{general}
\makeindex{poy3}

\newlength\sidebar
\newlength\envrule
\newlength\envborder
\newlength\boxwidth

\setlength\sidebar{1.5mm}
\setlength\envrule{0.4pt}
\setlength\envborder{2.5mm}
\setlength\itemindent{1cm}
\sethlcolor{yellow}
\setcounter{secnumdepth}{2}
\setcounter{tocdepth}{2}

\definecolor{exampleborder}{rgb}{0,0,.7}
\definecolor{examplebg}{rgb}{.9,.9,1}
\definecolor{shadecolor}{rgb}{.9,.9,1}
\definecolor{statementborder}{rgb}{.9,0,0}
\definecolor{statementbg}{rgb}{1,.9,.9}

\newsavebox\envbox
\newlength\notelength

\newenvironment{statement}[1][NOTE]{
% Default statement has no title %ilya: Statement is used for NOTES
\SpecialEnv{#1}{statementborder}{statementbg}{statementborder}{}%
}{%
\endSpecialEnv}

\def\Empty{}

% #1 title (if any)
% #2 sidebar (and title bg) color
% #3 background color
% #4 border color (or null for no border)
% #5 \enspace

	\newenvironment{SpecialEnv}[5]{%
	\par
	\def\EnvSideC{#2}% To use later (in end)
	\def\EnvBackgroundC{#3}%
	\def\EnvFrameC{#4}% 
	\flushleft

%\setlength\leftskip{-\sidebar}%
%\addtolength\leftskip{-\envborder}%
\noindent \nobreak
% Check if title is null:
\ifx\delimiter#1\delimiter\else
% If a title is specified, then typeset it in reverse color
\colorbox{\EnvSideC}{%
%\hspace{-\leftskip}% usually positive
%\hspace{-\fboxsep}%
\footnotesize\sffamily\bfseries\textcolor{white}{#1}%
%\hspace{\envborder}}%
}
\par\nobreak
\setlength\parskip{-0.2pt}% Tiny overlap to counter pixel round-off errors
\nointerlineskip 
\fi

% Make side-bar
\textcolor{\EnvSideC}{\vrule width\sidebar}%
% collect body in \envbox:
\begin{lrbox}\envbox 
\setlength{\boxwidth}{\linewidth}
\settowidth{\notelength}{NOT}
\addtolength{\boxwidth}{-1\notelength}
\addtolength{\boxwidth}{-1\sidebar}
\addtolength{\boxwidth}{-1\envborder}
\begin{minipage}[l]{\boxwidth}%

% insert counter, if any:
\ifx\delimiter#5\delimiter\else
%\theexample.\enspace
\fi
\ignorespaces
}{\par
\end{minipage}
\end{lrbox}%
% body is collected. Add background color
\setlength\fboxsep\envborder
\ifx\EnvFrameC\Empty % no frame
\colorbox{\EnvBackgroundC}{\usebox\envbox}%
\else % frame
\setlength\fboxrule\envrule
\addtolength\fboxsep{-\envrule}%
\fcolorbox{\EnvFrameC}{\EnvBackgroundC}{\usebox\envbox}%
\fi
\nobreak \hspace{-2\envborder}\null
\endflushleft
}


\newenvironment{poyexamples}{ \subsubsection{Examples} \begin{itemize}}{\end{itemize}}
% We define a command environment for new command definitions, and store
% whatever command we are dealing with in the @commandname macro. Inside a
% command we can specify the defaults, the syntax, and the arguments to be used.
\newenvironment{command}[2]{
    \def\tmpa{}
    \def\tmpb{#2}
    \def\@commandname{#1}
    \subsection{#1}\index{general}{#1}
    \ifx\tmpa\tmpb 
        \label{comm:#1} 
    \else 
        \label{comm:#2} 
    \fi}
    {}
% Syntax definition. We use the name of the command as stored in @commandname
\newcommand{\syntax}{\subsubsection{Syntax} \@commandname} 
\newcommand{\atsymbol}{@}

% We need to choose properly if we need to open a description or not inside
% an argument or argument group environment. We will use these two variables
% to keep the proper value to be used on each point.
\def\opendescription{\begin{description}}
\def\closedescription{}

% We define a pair of commands to initialize and finalize a description list 
% and set the necessary values of opendescription and closedescription.
\newcommand{\initdescription}{
    \def\closedescription{\end{description}}
    \opendescription
    \def\opendescription{}
}
\newcommand{\finishdescription}{ 
    \closedescription
    \def\closedescription{}
    \def\opendescription{\begin{description}}
}

% A variable to store a prefix for argument definitions, usually a colon.
\def\argprefix{} 

\newenvironment{script}{\begin{verbatim}}{\end{verbatim}}
\newcommand{\poyexample}[2]{\item \commandstyle{#1} \\#2}
\newcommand{\commandstyle}[1]{\texttt{#1}}
\newcommand{\poycommand}[1]{\commandstyle{#1}}

% Argument specifications
\newcommand{\poyargument}[1]{\commandstyle{#1}}
\newcommand{\obligatory}[1]{\commandstyle{\argprefix#1}}
\newcommand{\optional}[1]{\commandstyle{[\argprefix#1]}}
\newcommand{\optionall}[1]{\commandstyle{[#1]}}
\newenvironment{arguments} {\subsubsection{Arguments}}{ \finishdescription }
\newenvironment{argumentgroup}[2]{\paragraph{#1} #2}{ \finishdescription }
\renewcommand{\labelitemi}{$\bullet$}
\renewcommand{\labelitemii}{$\cdot$}
\renewcommand{\labelitemiii}{$\diamond$}
\renewcommand{\labelitemiv}{$\ast$}
\newcommand{\argumentdefinition}[4]{
    % Check if we are inside an itemize environment or not, if not, start
    % it.
    \initdescription
    % We will check if the second argument is empty; if so, we don't need
    % the add the : prefix for the argument's value.
    \def\tmpa{}
    \def\tmpb{#2}
    \def\tmpc{#4}
    \index{general}{#1}
    \index{general}{\@commandname!#1}
    \ifx\tmpa\tmpb
        \item[\poyargument{#1}]
            \ifx\tmpa\tmpc
                \label{comm:#1}
            \else
                \label{comm:#4}
            \fi
                #3
    \else
        \def\argprefix{:}
        \item[\poyargument{#1#2}]
            \ifx\tmpa\tmpc
                \label{comm:#1}
            \else
                \label{comm:#4}
            \fi
            #3
    \fi
    \def\argprefix{}
    }


\newenvironment{poydescription}{\subsubsection{Description}}{}

% The primitive types of a POY script.
\newcommand{\poystring}{\commandstyle{STRING}\xspace}
\newcommand{\poyfloat}{\commandstyle{FLOAT}\xspace}
\newcommand{\poyint}{\commandstyle{INTEGER}\xspace}
\newcommand{\poybool}{\commandstyle{BOOL}}
\newcommand{\poylident}{\commandstyle{LIDENT}\xspace}

\newcommand{\poydefaults}[2]{\subsubsection{Defaults} \commandstyle{\@commandname(#1)} #2}

\newenvironment{poyalso}{\subsubsection{See also}
\begin{itemize}}{\end{itemize}}

% Cross References
\newcommand{\cross}[1]{\item \commandstyle{#1} (Section~\ref{comm:#1})}
\newcommand{\ncross}[2]{\item \commandstyle{#1} (Section~\ref{comm:#2})}

\newcommand{\ccross}[1]{\commandstyle{#1} (Section~\ref{comm:#1})}
\newcommand{\nccross}[2]{\commandstyle{#1} (Section~\ref{comm:#2})}

% The typesetting of POY
\newcommand{\poy}{\commandstyle{POY5}\xspace}
\newcommand{\poyv}{\commandstyle{POY5}\xspace} % poyv stands for "poy version 4"

% Using same footnotes multiple times (used for authorship)
\newcommand{\footnoteremember}[2]{
  \footnote{#2}
  \newcounter{#1}
  \setcounter{#1}{\value{footnote}}
}
\newcommand{\footnoterecall}[1]{
  \footnotemark[\value{#1}]
}

% Hyphenations
\hyphenation{mo-le-cu-lar an-aly-ses an-aly-sis au-to-ma-ti-cally ho-mo-lo-gy chro-mo-so-me chro-mo-so-me-le-vel op-ti-mi-za-tion}

\makeatletter
\def\thickhrulefill{\leavevmode \leaders \hrule height 1pt\hfill \kern \z@}
\renewcommand{\maketitle}{\begin{titlepage}%
    \let\footnotesize\small
    \let\footnoterule\relax
    \parindent \z@
    \reset@font
    \null
    \vskip 50\p@
    \begin{center}
      {{\huge \texttt{POY} 5.0 R.C.}
      \par 
      \vskip \baselineskip
      \hrule height 1pt 
      \par 
      \vskip \baselineskip
      Program Documentation 
      \par
      \small Version 5.0.\buildnumber }\par
    \end{center}
    \vskip 65\p@
    \begin{flushright}
      \@author \par
    \end{flushright}
    \vskip 65\p@
    \begin{center}
    \includegraphics[width=\textwidth]{doc/figures/amnhlogoblue2.pdf}
    \end{center}
    \vfil
    \null
  \end{titlepage}%
  \setcounter{footnote}{0}%
}
\makeatother

\newcommand{\smallbuildnumber}{5.0}\newcommand{\buildnumber}{Black Sabbath Development build 2a12624f96c4}
\author{\textbf {Program and Documentation} \\ Andr\'es Var\'on \\ Nicholas Lucaroni \\Lin Hong \\Ward C. Wheeler \\ \bigskip 
\textbf{Documentation} \\ Louise M. Crowley \\ Megan Cevasco \\ John S. S. Denton}

%%%%%%%%%%%%%%%%%%%%%%%%%
%	BODY TEXT OF THE DOCUMENT     
%%%%%%%%%%%%%%%%%%%%%%%%%

\begin{document}

\maketitle

\thispagestyle{empty}

\vspace*{5.00cm}

\begin{flushleft}
\textbf{Previous Version POY 4}\\
\vspace*{1.00cm}
\textbf {Program and Documentation} \\ Andr\'es Var\'on \\Le Sy Vinh \\ Illya Bomash \\ Ward C. Wheeler \\
\vspace*{0.50cm}
\textbf{Documentation} \\ Ilya T\"emkin \\ Megan Cevasco \\ Kurt M. Pickett \\ Juli\'an Faivovich \\ Taran Grant \\ William Leo Smith
\end{flushleft}

\vspace*{5.00cm}

\begin{flushleft}
    \small
{\it
Andr\'es Var\'on}\\
Jane Street Capitol, 1 New York Plaza, New York, NY, U.S.A. \\
\smallskip 
{\it
Louise M. Crowley, Lin Hong, Nicholas Lucaroni, Ward C. Wheeler \\
}
Division of Invertebrate Zoology, American Museum of Natural History, New York, NY, U.S.A.\\
\smallskip
{\it
John S. S. Denton\\
}
Richard Gilder Graduate School and Department of Ichthyology, American Museum of Natural History, New York, NY, U.S.A.\\
\smallskip
{\it
Megan Cevasco} \\
Coastal Carolina University, Department of Biology, Conway, SC, U.S.A. \\
\vspace*{0.75cm}

{\it
Illya Bomash}\\
Department of Physiology and Biophysics, Weill Medical College of Cornell University, New York, NY, U.S.A.\\
\smallskip
{\it
Juli\'an Faivovich}\\
Divisi\'on Herpetolog\'ia, Museo Argentino de Ciencias Naturales - CONICET, Buenos Aires, Argentina.\\
\smallskip
{\it
Taran Grant}\\
Universidade de S\~{a}o Paulo, Instituto de Bioci\^{e}ncias, Departamento de Zoologia, Cidade Universit\'aria, 
S\~{a}o Paulo, Brasil.\\
\smallskip
{\it
Kurt M. Pickett}\\
Department of Biology, University of Vermont, Burlington, VT, U.S.A. \\
\smallskip
{\it
William Leo Smith}\\
Department of Zoology, The Field Museum of Natural History, Chicago, IL, U.S.A.\\
\smallskip
{\it
Ilya T\"emkin} \\
Northern Virginia Community College, Annandale Campus, VA, U.S.A. \\
\smallskip
{\it
Le Sy Vinh}\\
College of Technology and Information Technology Institute, Vietnam National University, Hanoi, Vietnam.  \\

\vspace*{0.25cm}
The American Museum of Natural History\\
\copyright  2013 by The American Museum of Natural History, \\
All rights reserved. Published 2013.

\vspace*{0.25cm}
\emph{Var\'on, A., N. Lucaroni, L. Hong, W. C. Wheeler.} 2013. \texttt{POY} 5.0. \buildnumber\ R.C. New York, 
American Museum of Natural History. Documentation by L. M. Crowley, M. Cevasco, J. S. S. Denton. 
\url{http://research.amnh.org/scicomp/projects/poy.php}

\vspace*{0.25cm}

Available online at
\url{http://research.amnh.org/scicomp/projects/poy.php}
and
\url{http://code.google.com/p/poy/} 

Comments or queries relating to the documentation should be sent to \href{mailto:crowley@amnh.org}{crowley@amnh.org}
\end{flushleft}

\tableofcontents

%%%%%%%%%%%%%%%%%%%%%%%%%%%%%%%%%%%%%%%%
%WHAT IS POY?
%%%%%%%%%%%%%%%%%%%%%%%%%%%%%%%%%%%%%%%%

\chapter{What is \poy}

\poy is a flexible, multi-platform program for the phylogenetic analysis of a diversity of data types under different optimality criteria ---
parsimony and likelihood.
An essential feature of \poy is that it implements the concept of dynamic homology \cite{wheeler2001a, wheeler2001} allowing 
optimization of   {\bf \emph{unaligned}} sequences. \poy offers flexibility for designing heuristic search strategies and implements an array of 
algorithms including multiple random addition sequence, swapping, tree fusing, tree drifting, and ratcheting. As output, \poy 
generates a comprehensive character diagnosis, graphical representations of cladograms and their user-specified consensus, 
support values and implied alignments.  In addition, \poy can also output synteny block maps from the analysis of both 
chromosomal and genomic data. \poy provides a unified approach to co-optimizing different types of data, such as morphological 
and molecular sequence data. In addition, \poy can analyze entire chromosomes and genomes, taking into account large-scale 
genomic events (translocations, inversions, and duplications).

\section{The structure of \poy documentation}
Chapter 2, \emph{\poy Quick Start}, will get you started using \poy. The first few sections are intended to provide detailed 
instructions on how to obtain and install \poy, introduce the user to two of the program's working environments, the 
\emph{Graphical User Interface} and the \emph{Interactive Console}. These sections also show how to initiate a \poy 
session and point to the various resources to obtain further assistance. Subsequent sections build on that knowledge and 
give step-by-step examples on how to conduct a basic analysis and visualize the results. The following chapter, 
\emph{\poy Commands}, describes \poy commands and their valid syntax. It also includes examples of simple operations 
for every command. Chapter 4 discusses the heuristic procedures used in \poy. Their understanding helps creating building 
efficient search strategies. More advanced operations are described in the fifth chapter, \emph{\poy Tutorials}. 
% In addition to the general index, this document contains a \emph{\texttt{POY3.0} Command Line Index}, intended to provide a link 
%between the commands used in \texttt{POY3} and the commands used in \poy. 

\section{What's new in \poy}
There are myriad new features and options in \poy.  These are described and documented in full in the pages that follow.  
\begin{itemize}
\item{New optimality criterion--likelihood:\\
	-- Maximum Average Likelihood (MAL) analysis can now be performed on qualitative data of any alphabet size 
	and aligned sequence data (including gaps as missing, independent, or coupled in 5-state models).\\  
	-- Most Parsimonious Likelihood (MPL) can also be employed on these data types as well 
	as unaligned sequences under an MPL-DO heuristic.\\
	-- Multiple models are available and different models can be assigned to partitions within a combined analysis.\\
	-- Model selection (AIC, AICc and BIC) improved.}
\item{The MAUVE genome aligner algorithm has been implemented as an annotation option
for unannotated chromosomal and genomic (multi-chromosomal) data.}
\item{The transform option \texttt{level} has been added to increase control and heuristic effectiveness 
for amino acid and custom alphabet sequence character types.}
\item{Search-Based sequence optimization has been added through the \texttt{transform} command.}
\item{Additional median solvers implemented for rearrangement analysis in \texttt{break\_inv}, \texttt{chromosome}, 
and \texttt{genome} sequence characters.}
\item{XML-based output for easy parsing of diagnostic information.}
\item{A change in the default indel cost from 2 to 1.  After over 20 years (MALIGN to POY), time for a change.}
\item{New required packages for compilation to support likelihood and median solvers.}
\item{A diversity of bug fixes and smaller enhancements.}
\end{itemize}

%%%%%%%%%%%%%%%%%%%%%%%%%%%%%%%%%%%%%%%%
%QUICKSTART
%%%%%%%%%%%%%%%%%%%%%%%%%%%%%%%%%%%%%%%%

\chapter{\poy Quick Start}

\section{What is \poy}

\poy is a flexible, multi-platform program for phylogenetic analysis of molecular and other data under various optimality criteria. An essential feature of \poy is that it implements the concept of dynamic homology allowing optimization of unaligned sequences. \poy offers great flexibility for designing heuristic search strategies and implements an array of algorithms including swapping, tree fusing, tree drifting, and ratcheting. As output, \poy generates a comprehensive character diagnosis, graphical representations of cladograms and their user-specified consensus, as well as support values, and implied alignments. \poy provides a unified approach to co-optimizing different types of data, such as morphological and molecular sequence data. In addition, \poy can analyze entire chromosomes and genomes and take into account large-scale genomic events (translocations, inversions, and duplications).

Currently \poy is beta software, and therefore it has some known glitches. Most
of them will be worked out in the following months, and updated versions will be
available on the program's webpage as we produce them. Our current schedule of
work expect to have a final official version of the parsimony components of the
program and a release of the beta components of the maximum
likelihood components in mid may of 2007. For the list of known issues see the
Section~\ref{sec:known_issues}.

\section{The structure of \poy documentation}
The first chapter, \emph{\poy Quick Start}, will get you started using \poy. The first few sections are intended to provide detailed instructions on how to obtain and install \poy, introduce the user to the program's two working environments, the \emph{Graphical User Interface} and \emph{POY Interactive Console}. These sections also show how to initiate a \poy session and point to the various resources to obtain further assistance. Subsequent sections (starting with \emph{1.13 Using \poy}) build on that knowledge and give step-by-step examples on how to conduct a basic analysis and visualize the results. The \emph{\poy Quick Start} is not a tutorial on \poy; using \poy assumes a knowledge of \poy commands and their valid syntax that are detailed in the second chapter, \emph{\poy Commands}. More advanced operations are described in the third chapter, \emph{\poy Tutorials}. In addition to the general index, this document contains a \emph{\texttt{POY3.0} Command Line Index}, intended to provide a link between the commands used in \texttt{POY3} and the commands used in \poy. 

The Quick Start is created primarily for a typical user with limited experience using command-line applications and assumes little or no knowledge of Unix. Consequently, certain operations suggested here could be performed more efficiently by an experienced user, but in attempt to make the software as accessible as possible, we provide simple and intuitive step-by-step (platform-specific where necessary) instructions. 

\section{Requirements: software and hardware}

\subsection{Software}
\poy is a platform-independent, open-source program that is compiled for Mac OSX, Microsoft Windows, and Linux systems. \poy \emph{binaries}\index{general}{binaries} (compiled application file) is the only piece of software necessary to run \poy. Other utility programs (that are typically installed with major operation systems), can facilitate preparation of \poy scripts (\poy command batch files) and formatting datafiles.

\begin{flushleft}
	\begin{minipage}[c]{0.074\textwidth}
	   	\includegraphics[width=\textwidth]{figures/figLogoWindows.jpg}
	\end{minipage}%
	\quad
	\begin{minipage}[t]{0.88\textwidth}
		   	\subsubsection{Windows}
	\end{minipage}
		\begin{description}
			\item[Notepad] is a basic text editor that can be used to create \poy
			scripts and format datafiles. By default, it is located in the \emph{Accessories}
			folder under \emph{All Programs} of the \emph{Start} menu.
			\item[Command Prompt] provides a working environment for
			\poy and is used to initiate a \poy session.
		It can be accessed from the same \emph{Accessories} folder as Notepad.
		\end{description}

	\begin{minipage}[c]{0.074\textwidth}
   		\includegraphics[width=\textwidth]{figures/figLogoMac.jpg}
	\end{minipage}%
	\quad
	\begin{minipage}[t]{0.88\textwidth}
	   	\subsubsection{Mac OSX}
	\end{minipage}
			\begin{description}
				\item[Terminal]  is an interface for UNIX operating systems and it
				provides a working environment for \poy; it is used to initiate a
				\poy session. Terminal is located in the \emph{Utilities} folder within
				the \emph{Applications} folder. (The program X11, that is
				also provided with OS X, can be used as an alternative to Terminal.)
				\item[TextEdit] is a basic text editor that can be used to create \poy
				scripts and format datafiles. By default, it is located in the
				\emph{Applications} folder. More flexible text editors, such as
				shareware applications like BBEdit or TextWrangler, are good alternatives.
			\end{description}		

	\begin{minipage}[c]{0.074\textwidth}
   		\includegraphics[width=\textwidth]{figures/figLogoLinux.jpg}
	\end{minipage}
	\quad
	\begin{minipage}[t]{0.88\textwidth}
	   	\subsubsection{Linux} 
	\end{minipage}

    A simple text editor, such as \href{http://www.nano-editor.org/}{\emph{nano}}, is sufficient, though more powerful
    editors (such as \href{http://www.vim.org}{\emph{vim}} or
    \href{http://www.gnu.org/software/emacs/}{\emph{emacs}}) can make your life much
    easier writting scripts for \poy.

\end{flushleft}

\subsection{Hardware}
\poy runs on a variety of computers computers from laptops and desktops to Beowulf clusters 
of various sizes to symmetric multiprocessing hardware. There are no
particular requirements for disk space. Processor speed and memory (and
communications bandwidth and latency in parallel environments) are important.
Depending on the size and complexity of the data, and the computational
complexity of requested operations, a \poy session can consume large amounts of memory.
However, the flexible structure of \poy allows for partitioning of individual
tasks to prevent overwhelming the hardware. Strict guidelines cannot be
provided because the performance depends on the specifics of a given
dataset; however, one can estimate the memory requirements by running a test command
or script under less demanding settings.

\section{Obtaining and installing \poy}

Compressed files of \poy binaries, source code, and documentation in PDF format are available 
for various Linux distributions, Microsoft Windows XP, and Mac OSX Tiger at the American Museum 
of Natural History Computational Sciences \poy website:
\begin{center}
\url{http://research.amnh.org/scicomp/projects/poy.php}
\end{center}
The following detailed step-by-step instruction will guide you through downloading,  decompressing, and installing \poy binaries for various platforms.

\begin{flushleft}
	\begin{minipage}[c]{0.074\textwidth}
	   	\includegraphics[width=\textwidth]{figures/figLogoWindows.jpg}
	\end{minipage}
	\quad
	\begin{minipage}[t]{0.88\textwidth}
		   	\subsubsection{Windows}
	\end{minipage}
		\begin{itemize}
			\item
                Download the
                \href{http://research.amnh.org/scicomp/projects/poy.php}{Windows compressed binary} file to the desktop.

			\item 
                Open the zipped file and run the windows installer
                POY\_installer.msi. The installer will create shortcuts in the
                \emph{Programs} menu.
		\end{itemize}

	\begin{minipage}[c]{0.074\textwidth}
   		\includegraphics[width=\textwidth]{figures/figLogoMac.jpg}
	\end{minipage}
	\quad
	\begin{minipage}[t]{0.88\textwidth}
	   	\subsubsection{Mac OSX}
	\end{minipage}
	            \begin{itemize}
			\item Download the
            \href{http://research.amnh.org/scicomp/projects/poy.php}{disk
            image} file to the desktop and open it. A disk named \emph{poy4} will be
            mounted.
            		\item Drag the \poy application from the
            disk \emph{poy4} and drop it into the \emph{Applications}
            folder on the hard drive.
		\end{itemize}

	\begin{minipage}[c]{0.074\textwidth}
   		\includegraphics[width=\textwidth]{figures/figLogoLinux.jpg}
	\end{minipage}
	\quad
	\begin{minipage}[t]{0.88\textwidth}
	   	\subsubsection{Linux}
	\end{minipage}
		\begin{itemize}
			\item  Download the 
    \href{http://research.amnh.org/scicomp/projects/poy.php}{gzipped} file.
    			\item Untar and ungzip the \emph{poy4.tar.gz} file.
			\item Run the command \texttt{tar -Pxvzf poy4.tar.gz} as a
    super user in the newly created \emph{poy4} directory.
    The GUI will be installed in \texttt{/opt/poy4/Contents/POY} directory
    and terminal binaries in \texttt{/opt/poy4/Resources/ncurses\_poy} directory.
		\end{itemize} 

\end{flushleft}

\subsubsection{Compiling from the Source}

In order to compile \poy the following tools are required:

\begin{enumerate}
    \item \href{http://www.gnu.org/software/make/}{The GNU Make tool.}
    \item \href{http://www.ocaml.org}{OCaml version 3.10.0. or later.}
    \item \href{http://gcc.gnu.org/}{A C compiler, for example The GNU Compiler Collection}.
    \item \href{http://www.gnu.org/software/ncurses/}{The ncurses library} if
        you want the nice interactive console, thought this one is
        \emph{not obligatory}.
    \item \href{http://tiswww.case.edu/php/chet/readline/rltop.html}{The
        readline library} if you want
        to compile POY without interactive console. This is the typical
        selection when compiling POY for parallel environments under MPI.
\end{enumerate}

Download, ungzip, and untar the
\href{http://research.amnh.org/scicomp/projects/poy.php}{\poy source code};
In order to compile under default setting just do:
\begin{verbatim}
./configure
make
make install
\end{verbatim}
All the configuration options can be found in {\tt ./configure --help}.

\poy can also be run in parallel environments using the
\href{http://www-unix.mcs.anl.gov/mpi/}{Message Passing Interface}. There are
multiple implementations, and if you have a parallel environment, most likely
your system administrator has already installed one. Ask him for the proper
paths to set in your config file.

\section{The Graphical User Interface}

\poy provides two working environments: the \emph{Graphical User Interface} and the \emph{Interactive Console}.  The \emph{Graphical User Interface} has a user-friendly appearance as any other native stand-alone application where different functions are accessible through menus and windows. Thus, the entire analysis can be carried out clicking on appropriate selections and, where necessary, typing specifications in designated fields. Although intuitive and accessible, the the \emph{Graphical User Interface} does not provide all the available options and is not as flexible for designing a search strategy as \emph{Interactive Console}. The interactive console, however, requires a detailed knowledge of \poy commands, their arguments, and the conventions of \poy scripting. All these features are described in detail in the \emph{POY Commands} chapter.

Even though the Mac OSX version of the \emph{Graphical User Interface} is used for screen shots throughout this chapter, its Linux and Windows versions contain the same items and functionality, differing only by the generic window format specific to each platform.

When \poy is first open, two items appear on the screen: the menu bar across the top and the \emph{POY Launcher} window (Figure~\ref{fig:menu_launcher_window}). Note that in Linux the menu bar is within the launcher window.
\begin{figure}[htpb]
    \begin{center}
        \includegraphics[width=0.5\textwidth]{figures/menu_launcher_window.jpg}
    \end{center}
    \caption{The \poy menu bar and the \emph{POY Launcher} window. These items appear when \poy is opened.}
    \label{fig:menu_launcher_window}
\end{figure}

\subsection{POY menu bar}
The menu bar contains the following drop-down menus:
\begin{description}
\item[POY] This menu is present only in Mac OSX and contains generic items as other Mac OSX applications. It includes \emph{Quit POY} tap that closes the program. This menu also allows to display the \emph{About POY} window (Figure~\ref{fig:about_window}) which lists the current version of \poy, a copywright statement, and the address of \poy website.
\item[Analysis] This menu contains options for different types of tree searches, calculation of support values, tree diagnosis, and their respective outputs. Other items in this menu open the \emph{POY launcher} and the \emph{Interactive Console}.
\item[Edit] This menu contains standard tools for deleting, copying, cutting, pasting, undoing, and selecting.
\item[View] The only item in this menu is the \emph{Output} window that contains two fields: the \emph{Results and Errors} and\emph{Status}, which display, respectively, the results (including warning and error messages) and the current state of the analysis.
\item[Help] This menu provides a link to the \poy \emph{QuickStart and Program Documentation} in PDF format.
\end{description}

\subsubsection{About POY}

The about window contains the program version number and contact information for
the program. It can be found in the \poy menu of Mac OS X, or the \emph{View} menu of
Windows and Linux.
\begin{figure}[htpb]
    \begin{center}
        \includegraphics[width=0.35\textwidth]{figures/About_Window.jpg}
    \end{center}
    \caption{The \emph{About POY} window.}
    \label{fig:about_window}
\end{figure}

\subsection{POY Launcher} 
\emph{POY Launcher} is the only window that automatically opens upon starting
\poy. It allows the user to import a previously created script,
designate a working directory, specify the number of processors,
and start the analysis.

\begin{description}
	\item[Select the script to run...]
     Allows the user to specify the location of a \poy script.
	\item[Select the working directory...]
    By default the working directory is set to be the same as the
    directory containing the selected \poy script but the default
    can be modified by the user. The working directory is the
    directory that contains the data files and where the results
    are reported.
	\item[Select the number of processors]
    The selection of the number of processors is disabled for LINUX
    and WINDOWS platforms. Once specified, the selection is applied
    to all subsequent analyses in the current \poy session.
	\item[Run the analysis]
    Clicking the \emph{Run} button starts the execution of the selected
    script. Once the script is executed, the \emph{Run} button
    becomes the \emph{Cancel} button that can be used to interrupt
    a \poy session.
\end{description}

If the script is executed without the properly selected script and
working directory (of their names contain errors), \poy issues an
error message in the upper part of the \emph{POY Launcher} window,
such as \texttt{POY finished with an error}.

\subsection{The \emph{Analysis} menu}
The \emph{Analysis} menu is the main toolbox of \poy. Its selections are subdivided into four functional categories that deal with tree searching, support calculation, tree diagnosis, and data import (including the initialization the \emph{Interactive Console}). Each of the menu items is described below in order as it
appears on the menu. Most options are the same for different kinds of analysis. Therefore, all the options are described in detail only for the \emph{Simple Search} analysis. The descriptions of other analyses is made with reference to the the \emph{Simple Search} and focus on options unique to each kind of analysis.

\subsubsection{\emph{Tree searching options}}

\subsubsection{Simple Search}
The \emph{Simple Search} window (Figure~\ref{fig:simple_search_window})
provides the most common and basic options for a standard tree search
in \poy that must either (in some cases or) selected by clicking appropriate buttons or typed in. Note that \emph{all} the empty fields must be filled out. The window is subdivided in four sections: 

\begin{figure}
\centering
\begin{minipage}[c]{0.48\textwidth}
   		\includegraphics[width=\textwidth]{figures/SimpleSearch_Menu.jpg}
\end{minipage}
\quad
\begin{minipage}[c]{0.48\textwidth}
	   	\includegraphics[width=\textwidth]{figures/SimpleSearch_Window.jpg}
   	\end{minipage}
	
\caption{The \emph{Simple Search} window. Selecting \emph{Simple Search} from the \emph{Analysis} menu (left) and viewing the \emph{Simple Search} window options (right).}
\label{fig:simple_search_window}
\end{figure}

\begin{description}
    \item[Input Files]
        Contains the list of files that are to be read by \poy. These include
        character files, and tree files. 
    \item[Search Parameters]
        Sets the number of randomized trees (independent random addition replicates)
        to be generated.
    \item[Sequence Alignment Parameters]
        Specifies the substitution, indel, and gap opening costs. Enter \texttt{0} if no
        gap opening cost is desired.
    \item[Output Files]
        Designates the names and locations of files containing different kinds of results
        (implied by their respective titles) of the analysis.
\end{description}

Once all the parameters are selected, click the \emph{Make Script} button and another
window--the \emph{Script Editor} (Figure~\ref{fig:ScriptEditor_Window})--containing the generated script appears on screen. The
script can be edited by typing in the commands directly in the \emph{Script Editor} window,
 saved (by clicking the \emph{Save As} button), or replaced with another script (using 
 the \emph{Open} button. To start the analysis click the \emph{Run} button in the 
 \emph{Script Editor} window. When the \emph{Run} button is pressed, \poy will issue a
 request to  to save the script to be executed. Thus, not only does \poy execute the script but
 it also created the record of what kind of analysis (including all user-defined specifications) was performed.
 
 \begin{figure}[htpb]
    \begin{center}
        \includegraphics[width=0.5\textwidth]{figures/ScriptEditor_Window.jpg}
    \end{center}
    \caption{Viewing the \emph{Script Editor} window. The script was generated from the \emph{Simple Search} window by clicking the \emph{Make Script} button. Because no fields were filled, the script does not contain the input files and shows the commands under default settings that would be executed under a simple search strategy.}
    \label{fig:ScriptEditor_Window}
\end{figure}

\subsubsection{Search with Ratchet}

\emph{Search with Ratchet} (Figure~\ref{fig:search_with_ratchet_window}) provides means to escape the local optimum using the parsimony ratchet. In addition to the same four parameter groups described for the \emph{Simple Search} window, the \emph{Search Parameters} section provides the following ratchet parameters fields:

\begin{figure}
\centering
\begin{minipage}[c]{0.48\textwidth}
   		\includegraphics[width=\textwidth]{figures/SearchWithRatchet_Menu.jpg}
\end{minipage}
\quad
\begin{minipage}[c]{0.48\textwidth}
	   	\includegraphics[width=\textwidth]{figures/SearchWithRatchet_Window.jpg}
   	\end{minipage}
	
\caption{The \emph{Search with Ratchet} window. Selecting \emph{Search with Ratchet} from the \emph{Analysis} menu (left) and viewing the \emph{Search with Ratchet} window options (right).}
\label{fig:search_with_ratchet_window}
\end{figure}

\begin{description}
    \item[Ratchet iterations] The number of iterations for the parsimony
        ratchet.
    \item[Severity] The severity parameter of the parsimony ratchet (the weight
        change factor for the selected characters).
    \item[Percentage] The percentage of characters to be affected by the
        parsimony ratchet.
\end{description}

\subsubsection{Search with Perturb}

\emph{Search with Perturb} provides a different way to escape the local optima by changing
the transformation cost matrix of the molecular characters
(Figure~\ref{fig:search_with_perturb_window}). In addition to the
same four parameter groups described for the Simple Search window, the Search
with Perturb window provides three extra fields with the parameters for the
transformation cost matrix perturbation as follows:

\begin{figure}
\centering
\begin{minipage}[c]{0.48\textwidth}
   		\includegraphics[width=\textwidth]{figures/SearchWithPerturb_Menu.jpg}
\end{minipage}
\quad
\begin{minipage}[c]{0.48\textwidth}
	   	\includegraphics[width=\textwidth]{figures/SearchWithPerturb_Window.jpg}
   	\end{minipage} 
\caption{The \emph{Search with Perturb} window. Selecting \emph{Search with Perturb} from the \emph{Analysis} menu (left) and viewing the \emph{Search with Perturb} window options (right).}
\label{fig:search_with_perturb_window}
\end{figure}

\begin{description}
    \item[Perturb iterations] Sets the number of perturb iterations to be performed.
    \item[Substitutions] Specifies the cost of the perturbed substitutions.
    \item[Indels] Specifies the cost of the perturbed indels.
\end{description}

\subsubsection{\emph{Support calculation options}}

None of the support calculation windows include functions for tree building and searching. Therefore, one of the input files must contain trees for which support values are going to be calculated.

\subsubsection{Bootstrap}

The \emph{Bootstrap} window (Figure~\ref{fig:bootstrap}) specifies parameters
for estimating the Bootstrap support values. The window consists
of essentially same four sections as the \emph{Simple Search} window
except for \emph{Pseudoreplicates}, a single field in the \emph{Bootstrap Parameters} section that specifies the number of Bootstrap pseudoreplicates.

\begin{figure}
\centering
\begin{minipage}[c]{0.48\textwidth}
   		\includegraphics[width=\textwidth]{figures/Bootstrap_Menu.jpg}
\end{minipage}
\quad
\begin{minipage}[c]{0.48\textwidth}
	   	\includegraphics[width=\textwidth]{figures/Bootstrap_Window.jpg}
   	\end{minipage}
\caption{The \emph{Bootstrap} window. Selecting \emph{Bootstrap} from the \emph{Analysis} menu (left) and viewing the \emph{Bootstrap} window options (right).}
\label{fig:bootstrap}
\end{figure}

\subsubsection{Bremer}

The \emph{Bremer} option (Figure~\ref{fig:search_for_bermer_menu}) is divided in two windows: the \emph{Search for Bremer} window, that specifies the Bremer support calculation parameters, and the \emph{Report Bremer} window to format the output of the results (Figure~\ref{fig:search_report_bremer}). 

\paragraph{Search for Bremer}

\begin{figure}[htpb]
    \begin{center}
        \includegraphics[width=0.65\textwidth]{figures/SearchForBremer_Menu.jpg}
    \end{center}
    \caption{ Selecting the \emph{Bremer} windows from the \emph{Analysis} menu.}
    \label{fig:search_for_bermer_menu}
\end{figure}

\begin{figure}
\centering
\begin{minipage}[c]{0.48\textwidth}
   		\includegraphics[width=\textwidth]{figures/SearchForBremer_Window.jpg}
\end{minipage}
\quad
\begin{minipage}[c]{0.48\textwidth}
	   	\includegraphics[width=\textwidth]{figures/ReportBremer_Window.jpg}
   	\end{minipage}
\caption{Viewing the options of the \emph{Search for Bremer} (left) and the \emph{Report Bremer}(right) windows.}
\label{fig:search_report_bremer}
\end{figure}

The script produced in this window collects trees visited during a search. This
search can take a long time, as all the tree search heuristics are turned off,
with the goal of sampling a wide variation of trees, and guarantee that all
clades have Bremer support values. 

In addition to the standard four sections defined for the \emph{Simple Search} window,
not that one of the output files is the \emph{Temporary Trees} file, which will
contain \emph{all the information required to produce the bremer support tree
results in the Report Bremer Window}. Make sure that you pick a file that you
will not destroy for this output.

If it takes \poy too long to finish searching for Bremer, the search can be interrupted and the intermediate results stored in the \emph{Temporary Trees} file (however at the risk that that bremer support values can be inflated). The trees from the \emph{Temporary Trees} file can then be reported using the \emph{Report Bremer} window.

\paragraph{Report Bremer}
The script produced in this window takes the ``Temporary Trees'' file generated in the Search for Bremer window in the ``File with trees for
bremer calculation'' field. 

\subsubsection{Jackknife}

The \emph{Jackknife} window (Figure~\ref{fig:jackknife}) specifies parameters for estimating the
Jackknife support values. The window consists of the same
four sections as the \emph{Simple Search} window except for
\emph{Jackknife Parameters} that has two fields, \emph{Pseudoreplicates}
and \emph{Remove}.

\begin{figure}
\centering
\begin{minipage}[c]{0.48\textwidth}
   		\includegraphics[width=\textwidth]{figures/Jackknife_Menu.jpg}
\end{minipage}
\quad
\begin{minipage}[c]{0.48\textwidth}
	   	\includegraphics[width=\textwidth]{figures/Jackknife_Window.jpg}
   	\end{minipage}
\caption{The \emph{Jackknife} window. Selecting \emph{Jackknife} from the \emph{Analysis} menu (left) and viewing the \emph{Jackknife} window options (right).}
\label{fig:jackknife}
\end{figure}

\begin{description}
    \item[Pseudoreplicates] Specifies the number of resampling iterations.
    \item[Remove] Specifies the percentage of characters being deleted during a pseudoreplicate.
\end{description}

\subsubsection{\emph{Diagnosis}}

\subsubsection{Diagnose Tree}

The \emph{Diagnose Tree} window (Figure~\ref{fig:diagnosetree}) specifies parameters for reporting a tree diagnosis. This window lacks the \emph{Search Parameters} section because the diagnosis is performed on the trees resulted from prior searches and no new trees are generated during the diagnosis procedure.

\begin{figure}
\centering
\begin{minipage}[c]{0.48\textwidth}
   		\includegraphics[width=\textwidth]{figures/diagnose_menu.jpg}
\end{minipage}
\quad
\begin{minipage}[c]{0.48\textwidth}
	   	\includegraphics[width=\textwidth]{figures/diagnose_window.jpg}
   	\end{minipage}
\caption{The \emph{Diagnose} window. Selecting \emph{Diagnose Tree} from the \emph{Analysis} menu (left) and viewing the \emph{Diagnose} window options (right).}
\label{fig:diagnosetree}
\end{figure}

\subsubsection{\emph{Script editing and the Interactive Console}}

\subsubsection{Open POY Script}

Selecting \emph{Open POY Script} (Figure~\ref{fig:open_poy_script}) displays the \emph{POY Launcher} 
window (Figure~\ref{fig:menu_launcher_window}), the function of which is described above.

\begin{figure}[htpb]
    \begin{center}
        \includegraphics[width=0.5\textwidth]{figures/OpenPoyScript_Menu.jpg}
    \end{center}
    \caption{The \emph{Open POY script} selection opens the \emph{POY Launcher} window.}
    \label{fig:open_poy_script}
\end{figure}

\subsubsection{Run Interactive Console}

Selecting \emph{Run Interactive Console} (Figure~\ref{fig:runinteractive}) opens the ncurses interface
that enables the user to run the analysis interactively by entering
\poy commands directly via the command-line \emph{Interactive
Console}. Note that the interactive console \emph{does not run in parallel.}

\begin{figure}
\centering
\begin{minipage}[c]{0.48\textwidth}
   		\includegraphics[width=\textwidth]{figures/RunInteractive_Menu.jpg}
\end{minipage}
\quad
\begin{minipage}[c]{0.48\textwidth}
	   	\includegraphics[width=\textwidth]{figures/create_script_window.jpg}
   	\end{minipage}
\caption{The \emph{Run interactive console} selection (left) opens \poy interactive console in a new window. The \emph{Create Script} selection opens the \emph{Script Editor} window (Figure~\ref{fig:ScriptEditor_Window}).}
\label{fig:runinteractive}
\end{figure}

Selecting \emph{Run Interactive Console} (Figure~\ref{fig:runinteractive}) opens the ncurses interface (Figure \ref{fig:figinterface}) that enables the user to run the analysis interactively by entering
\poy commands directly via the command-line \emph{Interactive
Console}. Note that the interactive console \emph{does not run in parallel.}

\subsubsection{Create Script}
The \emph{Create Script} selection opens a blank \emph{Script Editor} window that allows to create, save, and execute  a customized script.

\subsection{The \emph{View} menu}

The \emph{View} menu contains the \emph{Output} window which is subdivided into two fields: the upper \emph{Results and Errors} and lower \emph{Status} (Figure~\ref{fig:results_and_status_windows}). These fields display, respectively, the results (including warning and error messages) and the current state of the analysis. These fields are not updated automatically and in order to display the current state of the analysis the user must click
the \emph{Update} button.

\begin{figure}
\centering
\begin{minipage}[c]{0.48\textwidth}
   		\includegraphics[width=\textwidth]{figures/View_Menu.jpg}
\end{minipage}
\quad
\begin{minipage}[c]{0.48\textwidth}
	   	\includegraphics[width=\textwidth]{figures/output_window.jpg}
   	\end{minipage}
\caption{Selecting the \emph{Output} window (left) and viewing the \emph{Results and Errors} and  {Status of Search} fields.}
\label{fig:results_and_status_windows}
\end{figure}

\section{\poy Interactive Console}

\emph{POY Interactive Console} provides a command line-based environment with enhanced ability to display the results and the state of the analysis. Using the console requires familiarity with \poy commands, their arguments, and the conventions of \poy scripting (which are discussed the \emph{POY Commands} chapter). The console has the same appearance regardless of the operation system under which \poy is run (except under parallel environment settings; see below). It has four windows: \emph{POY Output}, \emph{Interactive Console}, \emph{State of Stored Search}, and \emph{Current Job} (Figure \ref{fig:figinterface}).

\begin{figure}[htbp]
   \centering
   \includegraphics[width=0.7\textwidth]{figures/figinterface.jpg}
   \caption{\poy interface displayed in the Terminal window prior to analysis. Note the cursor at the \poy prompt in the \emph{Interactive Console} and that the \emph{State of Stored Search} and \emph{Current Job} windows are empty.}
   \label{fig:figinterface}
\end{figure}

\begin{description}
\item[POY Output window] displays the status of the imported data, outputs the results of the phylogenetic analyses (such as trees, character diagnoses, and implied alignments), reports errors, and displays descriptions of \poy commands. By default, \poy reports the list of imported data and generates error messages. Other outputs, however, must be requested using the \commandstyle{report()} command.
\item[Interactive Console]  is used to instruct \poy to import data, specify the kinds of analysis to be performed, and to request the desired output interactively by typing \poy commands at the \poy prompt (\texttt{poy$>$}). The commands are executed by hitting the Return key. The commands can be executed one at a time or entered sequentially until the Return key is pressed. (See Section~\ref{commands} on the structure and syntax of \poy commands.) Separating commands by spaces is optional but increases legibility. Alternatively, a file containing the list of commands (\poy script, see below) can also be imported and executed at the prompt in the Interactive Console.
\item[State of Stored Search]  displays the time (in seconds) elapsed since the initiation of the current operation. This window also reports the number of trees currently in memory and provides the range of their costs.
\item[Current Job] describes the currently running operation. When the operation is completed, the window is blank.
\end{description} 

\begin{figure}[htbp]
   \centering
   \includegraphics[width=0.7\textwidth]{figures/figprocess.jpg}
   \caption{\poy interface during a process. The \emph{POY Output} window displays (by default) the information on the input datafiles. The \emph{Interactive Console} lists the commands that have been consecutively executed. The \emph{Current Job} window shows the state of the current operation and the current tree score. The \emph{State of Stored Search} shows the time elapsed  since the last command, \commandstyle{swap}, was initiated.}
   \label{fig:figprocess}
\end{figure}

This \poy interface is not available for parallel environments. Once the program is called, \poy commands can be executed interactively or scripts can be submitted as when using the \emph{Interactive Console}. By default, the \poy will print the output on screen (the same output that is reported in \emph{POY Output} under non-parallelized setting).

\subsection{Starting a \poy session using the \emph{Interactive Console}}

\begin{flushleft}
	\begin{minipage}[c]{0.075\textwidth}
	   	\includegraphics[width=\textwidth]{figures/figLogoWindows.jpg}
	\end{minipage}
	\quad
	\begin{minipage}[t]{0.89\textwidth}
		\subsubsection{Windows}
	\end{minipage}
			\begin{itemize}
                \item{Start$>$All Programs$>$POY$>$POY Interactive Console}
			\end{itemize}

	\begin{minipage}[c]{0.075\textwidth}
   		\includegraphics[width=\textwidth]{figures/figLogoMac.jpg}
	\end{minipage}%
	\quad
	\begin{minipage}[t]{0.89\textwidth}
	   	\subsubsection{Mac OSX}
	\end{minipage}
			\begin{itemize}
				\item {Double-click \poy application icon to start the program.}
				\item {Select \emph{Run Interactive Console} from the
				\emph{Analyses} menu.}
			\end{itemize}		

	\begin{minipage}[c]{0.075\textwidth}
   		\includegraphics[width=\textwidth]{figures/figLogoLinux.jpg}
	\end{minipage}
	\quad
	\begin{minipage}[t]{0.89\textwidth}
	   	\subsubsection{Linux}
	\end{minipage}
    Add \texttt{/opt/poy4/Resources/} to your \texttt{PATH} and then simply run
    \texttt{ncurses\_poy} from a terminal.
\end{flushleft}

\begin{figure}[htbp]
   \centering
   \includegraphics[width=0.7\textwidth]{figures/figprelim1.jpg}
   \caption{Specifying the location of datafiles. The folder \texttt{POY-Data} is dragged from the \texttt{POY v3-4} folder directly in the Terminal window.}
   \label{fig:figprelim1}
\end{figure}

\begin{figure}[htbp]
   \centering
   \includegraphics[width=0.7\textwidth]{figures/figprelim2.jpg}
   \caption{Starting \poy. At the folder containing datafiles, entering \texttt{poy} starts a \poy session.}
   \label{fig:figprelim2}
\end{figure}

\subsection{Entering commands}
The \emph{Interactive Console} is the only part of the interface that allows communication with \poy; that is where commands and scripts are executed. Once the \poy interface is called, the cursor appears in the \emph{Interactive Console} and \poy is ready to accept commands. \poy interface does not support using the mouse and, as true for most command-line applications, the cursor can be moved using the left and right arrow keys, and the Backspace (in Windows) or Delete (in Mac) keys are used to erase individual characters to the left of the current cursor position. To eliminate the need of retyping commands anew during a \poy session, keyboard shortcuts can be used: Control-P (``previous'') and Control-N (``next'') will scroll through the commands entered during the session.

\subsection{Browsing the output}
As more output is reported in the \emph{POY Output} window, only the most recent reports will be seen in the window. Using the Up and Down keys allows to scroll up and down the \emph{POY Output} window to see the welcome line, and previously printed reports and help descriptions. Pressing Up and Down keys automatically places the cursor in the lower left corner of the \emph{POY Output} window indicating that you are interacting with that window. It is important to know that only 1000 lines are stored in the memory and the output that was reported before that will not be accessible by scrolling. If it is desired to keep the entire output or specific items in the output, the log can be created (using the command \poycommand{set()}, see~\ccross{log}) or specific outputs can be redirected to files (see~\ccross{report}).

\subsection{Switching between the windows}
To return to the \emph{Interactive Console} start typing and the cursor will automatically be placed back at the \poy prompt. When an operation is in progress (which is shown in the \emph{Current Job} window), the cursor stays in the upper left corner of the \emph{State of Current Search} window, and switching between the \emph{Interactive Console} and the \emph{POY Output} window is disabled. There are no user interactions with the \emph{Current Job} and \emph{State of the State of Current Search}.

\subsection{Interrupting a process}
To interrupt a process, press Control-C. By default, an error, \texttt{Error:}\\ \texttt{Interrupted}, is reported in the \emph{POY Output} window. The program does not close, however, and a new command can be entered. This command does not close the program, it only stops the last process that was running keeping in memory all the data and the results of the operation executed last. New commands can subsequently be entered.

\subsection{Reporting errors}
\poy reports errors in several ways. If there is an error pertaining to wrong syntax (such as a typo in a command name), \poy will indicate the location of an error by underlining the problematic part of the input with ``\texttt{\^}'' in the \emph{Interactive Console} (Figure~\ref{fig:errors}). \poy will also automatically display in the \emph{POY Output} window the description of the command, its syntax, and examples of its usage. As explained above, the Up and Down keys can be used to scroll through the output and determine the source of the error. Certain kinds of errors will be reported explicitly (Figure~\ref{fig:errors}).

\begin{figure}
\centering
\begin{minipage}[c]{0.48\textwidth}
   		\includegraphics[width=\textwidth]{figures/figerror1.jpg}
\end{minipage}%
\quad
\begin{minipage}[c]{0.48\textwidth}
	   	\includegraphics[width=\textwidth]{figures/figerror2.jpg}
   	\end{minipage}
	
\caption{Displaying errors. \poy displays error messages in several ways. In the example in the left panel, the command \commandstyle{build} was entered without parentheses, which is required for a  valid \poy command syntax; The exact place of the error is marked by ``\texttt{\^}'', in this case  following the \commandstyle{build} commands. Examples of the proper usage of the command are automatically displayed in the \emph{POY Output}. In other cases (right panel), error messages are explicitly reported in the \emph{POY Output} window. The first and second error messages indicate that the datafile \texttt{SSU.seq} is not present, which could have been caused either by a mistake in the name of the file, missing file, or the location of the file in a directory, other than the one specified prior to starting the \poy session. The third error message indicates that the valid syntax of \commandstyle{exit} requires the parentheses following the command name (also shown by ``\texttt{\^}'' in  the \emph{Interactive Console}).}
\label{fig:errors}
\end{figure}

\subsection{Exiting}
To finish a \poy session, enter command \commandstyle{exit()} (Figure~\ref{fig:exithelp}) or \commandstyle{quit()}. This will close the \poy interface and resume the Terminal window (Mac) or the Command Prompt window (Windows).

\begin{figure}[]
    \begin{center}
        \includegraphics[width=0.5\textwidth]{figures/exithelp.jpg}
    \end{center}
    \caption{Exiting \poy}
    \label{fig:exithelp}
\end{figure}

\section{Obtaining help} \label{sec:help}
Instructions to run \poy, command descriptions, and the theory behind \poy can be obtained from a variety of sources.
\begin{description}
\item[POY] contains a help file that can be accessed by entering \commandstyle{help()} at \poy prompt in the \emph{POY Output} window. This file contains descriptions and examples of all currently implemented \poy commands. Up and down arrows allow to scroll through the file. To obtain help on a particular command, the name of the command must be specified in the parentheses following \commandstyle{help()}. For example, to learn about the command \commandstyle{exit}, type \commandstyle{help(exit)}. Help will appear in the upper window, as shown in Figure~\ref{fig:exithelp}.
\item[Quick Start and Program Documentation] is a comprehensive and detailed manual on all the aspects of using \poy, from installation to outputting and visualizing the results. It includes a \emph{Quick Start}, a \poy command reference, practical guides and tutorials that make the program immediately accessible for beginners and provide in-depth information for experienced users. The documentation in PDF format can be accessed from the \emph{Help} menu of the graphical user interface or downloaded separately from \poy web site at
\begin{center}
\texttt{http://research.amnh.org/scicomp/projects/poy.php}
\end{center}
\item[POY Book] (Wheeler et al., 2006 \emph{Dynamic Homology and Phylogenetic Systematics: A Unified Approach Using POY}) provides a review of the theory behind \poy, and contains formal descriptions of many algorithms implemented in the program and the descriptions of commands of the earlier version, \texttt{POY3}.
\item[POY4 Mail Group] is an Internet-based forum for discussing all issues related to \poy and provides the best way to communicate with \poy developers on specific issues (see \emph{WWW resources} below). The website is located at \texttt{http://groups.google.com/group/poy4}.
\begin{figure}[htbp]
   \centering
   \includegraphics[width=0.23\textwidth]{figures/figPOYBook.jpg}
   \caption{The \poy Book.}
   \label{fig:figprocess}
\end{figure}
\end{description}

\section{WWW resources}
\poy is an ongoing project and new versions are being continuously developed to include new procedures, improve performance, and eliminate reported bugs. Therefore, it is imperative to keep up with the program's development and check regularly for updates. There are several Internet-based resources that offer this information, and, additionally, provide a forum for discussing specific issues using \poy, and present an efficient way to communicate with \poy developers regarding any technical difficulties, reporting bugs, and obtaining help.

\begin{description}
\item[POY4 Web Site] has downloadable compressed files of \poy binaries, source code, and documentation in PDF format. It also provides a links to the \emph{POY Mail Group}. The website is hosted by AMNH Computational Sciences at 
\begin{center}
\texttt{http://research.amnh.org/scicomp/projects/poy.php}
\end{center}
\item[POY4 Mail Group] informs registered users via email of new developments, such as new versions and updates. It also provides a way for reporting bugs and other problems with \poy and its documentation, as well as an additional resource for obtaining help on specific issues. In addition, it allows users to receive and respond to each other's questions thus providing an open forum to  discuss the methods and applications of \poy. The users who choose not to register, will have access to the archives of the postings but will not be able to either submit or receive emails from other users and \poy developers. The \emph{POY4 Mail Group} is hosted  by Google at
	\begin{center}
	\texttt{http://groups.google.com/group/poy4}
	\end{center}
	
\end{description}

\section{Using \poy}

This section will help you get started using \poy and will prepare you for the
more extensive, technical descriptions in the next chapter, \emph{\poy Commands}. Now that you are acquainted with the program's interface, learned how to initiate, and exit or interrupt) a \poy session, and how to obtain help, you are well prepared to run your first analysis. This chapter will teach
you how to read (import) datafiles, check the data you are analyzing, generate
a set of initial trees, do basic branch swapping to find a local optimum, and, finally, produce
and visualize the resultant trees, their strict consensus, and generate support values.

For the purpose of this exercise, three datalfiles are used. These sample files can be downloaded from \\
\texttt{http://research.amnh.org/scicomp/projects/poy.php}:

\begin{itemize}
	\item {\texttt{18s.fas} contains unaligned DNA sequence data for a single locus (partial 18S ribosomal DNA) in FASTA format.}
	\item {\texttt{28s.fas} contains unaligned DNA sequence data for a single locus (partial 28S ribosomal DNA) in FASTA format.}
	\item {\texttt{morpho.ss} contains a morphological data matrix in Hennig86 format.}
\end{itemize}

Once \poy has been launched and the interface (Figure~\ref{fig:figinterface}) had appeared on the screen, the data can be imported and the analysis can proceed. As you follow the instructions, you are encouraged to consult the help file by using the command \commandstyle{help} (see Section~\ref{sec:help} to learn more about \poy commands and their arguments).

\subsection{Importing data} \label{sec:import}

The basic command to input data in \poy is \commandstyle{read()}, which includes the list of files (in quatation marks and separated by commas) enclosed in parentheses. Suppose that we would like to simultaneously analyze morphological and molecular datasets, contained in separate datafiles, \texttt{morpho.ss} and \texttt{28s.fas}, respectively. We can issue a pair of \commandstyle{read()} commands (Figure~\ref{fig:readingexample}):
\begin{quote}
        \commandstyle{read("morpho.ss")}\\
        \commandstyle{read("28s.fas")}
\end{quote}

\begin{figure}
    \begin{center}
        \includegraphics[width=0.6\textwidth]{figures/reading_example.jpg}
    \end{center}
    \caption{Importing datafiles using the \emph{Interactive Console}. Two consecutive \commandstyle{read} commands import both the morphological datafile in Hennig86 format (\texttt{morpho.ss}), and the molecular datafile in Fasta format (\texttt{28s.fas}) . Note that \poy automatically reports  in the \emph{POY Output} window the names and types of files that have been imported.}
    \label{fig:readingexample}
\end{figure}

The syntax of \commandstyle{read} is not unique; in fact, every command in \poyv contains two elements: the name of the command (e.g. \commandstyle{read}), followed by the list of arguments 
separated by commas and enclosed in parentheses. Typically, the arguments of the command \commandstyle{read()} are names of datafiles, each being enclosed in double quotes (as shown in the example above). Eventhough arguments there might be only one argument or it might be absent or omitted in some commands, parentheses (e.g.\poycommand{read()}) always follow the command name. An exhaustive discussion of \poy command structure and detailed descriptions of all commands with examples of their usage are provided in the \emph{POY Commands Reference} document.

Most of the time users are interested in importing multiple datafiles to analyze on the entire dataset. In this case, multiple datafiles can be specified as arguments for a single command. For example, importing both files, \texttt{morpho.ss} and \texttt{28s.fas}, can be written more succinctly:
\commandstyle{read("morpho.ss", "28s.fas")}. This is equivalent to sequentially importing each file at a time as was shown previously (Figures ~\ref{fig:readingexample} and \ref{fig:reading_example2}).

Figure~\ref{fig:readingexample} also illustrates an important feature that makes \poy
different from many other phylogenetic analysis programs: every
time a file is imported during a \poy session, the input data is \emph{added} to the current data in memory, it \emph{does not replace it}. This allows additional analytical flexibility. For example, if only morphological data are read and trees are built based on these data alone, a subsequently imported molecular character dataset will be used in conjunction with the previously imported morphological data in subsequent operations, despite the fact that the trees were generated only from morphological data (Figure~\ref{fig:reading_example2}):

\begin{quote}
\commandstyle{read("morpho.ss")}\\
\commandstyle{build()}\\
\commandstyle{read("28s.fas")}\\
\commandstyle{rediagnose()}\\
\commandstyle{swap()}
\end{quote}

It must be noted that if the number of terminals differs among datafiles, only the data that corresponds to the terminals used to generate the trees (from the morphological datafile in our example) are used; the rest of the character data are ignored.

Also, because \poy appends trees and data in memory, it is a good practice to empty the memory when starting a new analysis using use the \commandstyle{wipe()} command (see also \commandstyle{clear\_memory()}).

\begin{figure}[]
    \begin{center}
        \includegraphics[width=0.6\textwidth]{figures/reading_example2.jpg}
    \end{center}
    \caption{Building trees with morphological data only but continuing analysis using combined morphological and molecular data. This example shows how we can add data to the analysis incrementally by loading files at different points in the search. First, the morphological data are imported from \texttt{morpho.ss} file using \poycommand{read()} the and trees are built based on these data. Then molecular data from the \texttt{28s.fas} file are loaded into memory in addition to previously imported morphological data. Finally, subsequent analyses, \commandstyle{rediagnose()} and \commandstyle{swap()}, are conducted using the data in memory, that is the trees based on morphological data, and both morphological and molecular character sets.}
    \label{fig:reading_example2}
\end{figure}

Valid input files include nucleotide and amino acid sequence files in many formats,
and morphological data in Hennig 86 format. For information on specific formats supported by \poy and other types of input files see \commandstyle{help(read)}.

\subsection{Inspecting data}

Once a dataset (or multiple datasets) is imported, \poy automatically reports a brief description of contents for each loaded file in the \emph{POY Output} window as shown in Figure ~\ref{fig:readingexample}. However, it may be desirable to inspect the imported data in greater detail to ensure that the format and contents of the files has been interpreted correctly. This practice allows to avoid common errors, such as misspelled terminal names, which may result in bogus results, produce error messages, and aborted jobs.

The basic command for outputting information is \commandstyle{report()}. One of its arguments, \commandstyle{data}, outputs a set of tables showing the list of terminals, the number and type of characters, the list of files that have been imported so far, and the lists of terminals and characters excluded from the analysis. For example, to produce such report of the same datafiles that were used in the previous example (\texttt{morpho.ss} and \texttt{28s.fas}), we import the data and execute \commandstyle{report(data)}:
\begin{quote}
    \commandstyle{read("morpho.ss","28s.fas")}\\
    \commandstyle{report(data)}
\end{quote}
This will generate an extensive, detailed output, partial views of which are shown in Figure ~\ref{fig:reportdata}. Obviously, the entire report will not be visible in the \emph{POY Output} window. Therefore, the Up and Down arrow keys should be used to scroll through it.

\begin{figure}
\centering
\begin{minipage}[c]{0.52\textwidth}
   		\includegraphics[width=\textwidth]{figures/report2.jpg}
\end{minipage}
\quad
\begin{minipage}[c]{0.44\textwidth}
	   	\includegraphics[width=\textwidth]{figures/report3.jpg}
   	\end{minipage}
\caption{Inspecting imported data. The figure shows segments of a data report generated by \commandstyle{report(data)}. The left and right panels demonstrate a typical table output the character and terminal data respectively.}
\label{fig:reportdata}
\end{figure}

In this example, all the imported data is analyzed and, therefore, the report fields that list excluded data will appear empty in the report. One can, however, exclude specific characters or terminals from the analysis using additional commands (see \commandstyle{select()}).

By default, \poy reports the results of executed commands in the \emph{POY Output} window. However, the same output can be redirected to a file simply by adding the name of the output file in the list of argument of the command \commandstyle{report()} \emph{before} the argument that specified the type of the requested report (in this case \commandstyle{data}). For instance, if we would like to output into the file ``data\_analyzed.txt'', we would write \commandstyle{report("data\_analyzed.txt", data)}.

Another useful argument of \commandstyle{report} is \commandstyle{cross\_references}. It displays which terminals are present or absent in each one of the imported files, providing a comprehensive visual overview of the missing data. Building on the previous example, such output can be generated by the following sequence of commands:
\begin{quote}
    \commandstyle{read("morpho.ss", "28s.fas")}\\
    \commandstyle{report(cross\_references)}
\end{quote}

\begin{figure}[]
    \begin{center}
        \includegraphics[width=0.6\textwidth]{figures/crossref.jpg}
    \end{center}
    \caption{Visualizing missing data. The command \commandstyle{cross\_references} displays a table showing whether a given terminal (in the left column) is present (``+'') or absent (``-'') in each datifie. In this example, the data for all the the taxa listed in the \emph{POY Output} window are present both datafiles, \texttt{morpho.ss} and \texttt{28s.fas}.}
    \label{fig:crossref}
\end{figure}

A typical output of \commandstyle{cross\_references} command is shown in Figure ~\ref{fig:crossref}.

\subsection{Building initial trees}

The command to build trees is \commandstyle{build} (already been mentioned in Section~\ref{sec:import}). After importing \texttt{morpho.ss} and \texttt{28s.fas}, executing the command \commandstyle{build()} without specifying any arguments will generate 10 Wagner trees by random addition sequence (the default setting of the command). Make sure that if you plan to build trees based on other data you have to purge the memory first by using \commandstyle{wipe()} command
Many \poy commands operate under default settings when executed without arguments. (To learn what the default settings are for a particular command, use either \commandstyle{help()} command with the command name of interest inserted in parentheses or consult the \emph{POY Commands Reference}; see Section~\ref{sec:help}.) If you would like to build more trees, 100 for instance, an argument \commandstyle{trees} followed by a colon (``:'') and an integer specifying the number of trees must be included in the argument list of the \commandstyle{build} command: \commandstyle{build(trees:100)}. This command has a shortcut that omits the argument \commandstyle{trees}; therefore, \commandstyle{build(trees:100)} is equivalent to \commandstyle{build(100)}. As defaults, the shortcuts are fully described in the \emph{POY Commands Reference}. The entire sequence of commands minimally required to import the data and build 100 trees is the following:

\begin{quote}
 	\commandstyle{read("morpho.ss","28s.fas")}\\
 	\commandstyle{build(100)}
\end{quote}

As the tree building advances, the \emph{Current Job} window displays the current status of the operation (Figure~\ref{fig:building}). It shows how many Wagner builds have been performed out of the total number requested, the number of terminals added in the current build, the cost of a current tree (recalculated after each terminal addition), and the estimated time (in seconds) for the completion of all the builds. When all the trees are generated, the \emph{State of Stored Search} window displays the range of tree costs (the best and worst costs), the number of trees stored in memory, and the number of trees with the best cost.

\begin{figure}
\centering
\begin{minipage}[c]{0.507\textwidth}
   		\includegraphics[width=\textwidth]{figures/building1.jpg}
\end{minipage}
\quad
\begin{minipage}[c]{0.453\textwidth}
	   	\includegraphics[width=\textwidth]{figures/building2.jpg}
   	\end{minipage}
\caption{Generating Wagner trees. During the process of tree building (left panel), the \emph{Current Job} window displays how many builds have been performed so far (\texttt{57 of 100}), the number of terminals added in the current build (\texttt{13 of 17}), a cost of a current tree recalculated after each terminal addition (\texttt{362}), and the estimated time (in seconds) for the completion of the operation (\texttt{4 s}). Because the process is not complete, the \emph{State of Stored Search} window contains no trees. Once tree building is finished, the \emph{State of Stored Search} window displays the best (\texttt{451}) and worst (\texttt{472}) costs, the number of trees stored in memory (\texttt{100}), and the number of trees with the best cost (\texttt{2}).} 
\label{fig:building}
\end{figure}

\subsection{Performing a local search}

Now, that the trees have been generated and stored in memory, a local search can be performed to refine and improve the initial trees by examining additional topologies of potentially better cost.  The command \commandstyle{swap()} implements an efficient strategy by performing SPR and TBR branch swapping iteratively. As with other commands, the arguments of \commandstyle{swap()} allow customization of the performance of the command. In case of \commandstyle{swap()}, additional options specify which algorithms are used in swapping and restrict swapping to certain sections of trees. In the following example, branch swapping is performed under the default settings on each of the 100 trees build in the previous step:

\begin{quote}
 	\commandstyle{read("morpho.ss","28s.fas")}\\
 	\commandstyle{build(100)}\\
	\commandstyle{swap()}
\end{quote}

Branch swapping is performed sequentially on all trees stored in memory. During swapping, the \emph{Current Job} window reports the number of a tree that is currently being analyzed, the method of branch swapping, the specific routine being currently performed, and the cost of the current tree (Figure~\ref{fig:swapping}). When the process is complete, the \emph{State of Stored Search} window displays the range of tree costs (the best and worst costs), the number of trees stored in memory, and the number of trees of the best cost (Figure~\ref{fig:swapping}). Note that the local search had reduced the costs of the initial best (from 451 to 446) and narrowed the range of tree costs.

\begin{figure}
\centering
\begin{minipage}[c]{0.49\textwidth}
   		\includegraphics[width=\textwidth]{figures/swap1.jpg}
\end{minipage}
\quad
\begin{minipage}[c]{0.453\textwidth}
	   	\includegraphics[width=\textwidth]{figures/swap2.jpg}
   	\end{minipage}
\caption{Performing a local search. When searching (left panel), the \emph{Current Job} window reports the number of the tree that is currently being analyzed (\texttt{73 of 100}), a method of branch swapping (\texttt{Alternate}), a function being currently performed (\texttt{SPR search}), and a cost of the current tree (\texttt{456}). When the searching is finished (right panel), the \emph{State of Stored Search} window displays the best (\texttt{446}) and worst (\texttt{463}) costs, the number of trees stored in memory (\texttt{100}), and the number of trees of the best cost (\texttt{9}) recovered from independent tree builds. Note these trees may not necessarily have unique topology.} 
\label{fig:swapping}
\end{figure}

Using different combinations of the arguments of \commandstyle{swap()} allows to design a  large number of search strategies of different levels of complexity. Some simple options allow the choice between SPR and TBR. More complex strategies allow keeping a specific number of best trees per single initial tree (generated during the building step). For example, the command \commandstyle{swap(trees:10)} will keep up to 10 best trees generated during branch swapping on a single initial tree. Consequently, if 100 trees were built initially, this command will produce 1,000 trees. The argument \commandstyle{threshold} allows the retention of suboptimal trees within a specified percent of cost difference from the current best tree. For example, \commandstyle{swap(trees:10, threshold:10)}. Other options provide the means to sample trees as they are evaluated, timeout after certain number of seconds, transform the cost regime, and other perform other functions in conjunction with other \poy commands.

\subsection{Selecting trees}

Having performed the basic steps of importing character data, building initial trees, and conducting a local search, we obtained a set of local optimum trees in memory. Most of the time, a user would like to select only those trees that are optimal and topologically unique and the default setting of the \commandstyle{select()} does exactly that. Adding \commandstyle{select()} to our example of command sequence for the basic analysis 
\begin{quote}
 	\commandstyle{read("morpho.ss","28s.fas")}\\
 	\commandstyle{build(100)}\\
	\commandstyle{swap()}\\
	\commandstyle{select()}
\end{quote}
will select only unique trees of best cost; the remaining trees will be deleted from memory. The \emph{State of Stored Search} window will report the number and the cost of the best tree(s) (Figure~\ref{fig:select}).

\begin{figure}[]
    \begin{center}
        \includegraphics[width=0.6\textwidth]{figures/select.jpg}
    \end{center}
    \caption{Selecting unique best trees. Executing \commandstyle{select()} keeps only unique tees of best cost. The \emph{State of Stored Search} window reports that there is only one unique tree of best cost (\texttt{446}).}
    \label{fig:select}
\end{figure}

\commandstyle{select()} is another multifunctional command the arguments of which are also used to select (include or exclude) specific terminals, characters, and trees.)

Comparing the output reported in the \emph{State of Stored Search} before (Figure~\ref{fig:swapping}) and after (Figure~\ref{fig:select}) executing \commandstyle{select()} shows that swapping on 9 of 100 initial trees produced the trees of best cost (\texttt{446}), but these trees are identical, because only one was retained when filtered using \commandstyle{select()}.

\subsection{Visualizing the results}

There are several ways to visualize results. A quick way to see the tree(s) on screen is to use the command \commandstyle{report(asciitrees)} that will draw a cladogram in the \emph{POY Output} window (Figure~\ref{fig:trees}). The ascii tree can also be reported in a file, if the output file name is specified (in parentheses and separated from the argument \commandstyle{asciitrees} by a comma). However, for reporting trees to a file there are better options. First, the command \commandstyle{report("my\_first\_tree", graphtrees)} will output a cladogram in postscript format (Figure~\ref{fig:trees}) which can be edited using graphics software (such as Adobe Illustrator or Corel Draw). (\poy will also append the ``ps'' extension when generating graphic output to a file.)

\commandstyle{report("my\_first\_trees", trees)} will report the trees in memory to the file \texttt{my\_first\_trees} that can be imported in other programs (such as TNT). Other supported tree tree formats include Newick and Hennig86. \commandstyle{report()} can also generate consensus trees in the graphical (postscript) format when appropriate arguments are specified (for example, \commandstyle{report("strict\_consensus", graphconsensus)}).

\begin{figure}
\centering
\begin{minipage}[c]{0.45\textwidth}
   		\includegraphics[width=\textwidth]{figures/asciitree.jpg}
\end{minipage}
\quad
\begin{minipage}[c]{0.5\textwidth}
	   	\includegraphics[width=\textwidth]{figures/pstree.jpg}
   	\end{minipage}
\caption{Visualizing trees. An ascii tree (left) is generated using the command \commandstyle{asciitrees}. The same tree is reported to a file in a postscript format (right) using \commandstyle{report("my\_first\_tree", graphtrees)}. Note that both representations of trees  are preceded by their costs.}
\label{fig:trees}
\end{figure}

\section{Creating and running \poy scripts}

So far, we have communicated with \poy interactively through the \emph{Graphical User Interface} or by executing commands from the \emph{Interactive Console}. Another way of conducting an analysis is to run a \emph{script}, a simple-text file containing a list of commands to be performed (Figure~\ref{fig:script}). 

Running analyses using scripts has many advantages: not only does it allow for the entire analysis to proceed from the beginning to the end at one click of a button, it also provides means for examining the logical dependency of the commands (see the description of \poyargument{script\_analysis} argument of the command \poycommand{report} in the \emph{POY Commands} chapter). Submitting jobs using scripts typically produces results faster because \poy automatically optimizes the workflow of the entire analysis by taking into account the functional relationships among various tasks and efficiently distributing the resources (such as memory and multiple processors).

Another advantage of using scripts is that it can contain comments that are ignored by \poy but can be helpful to describe the contents of the files used and provide any other annotations. The comments are enclosed in parenthesis \emph{and} asterisks. For example, \texttt{(*this is a comment*)}. Comments can also be entered interactively from the \emph{Interactive Console}. Their utility in that context is, however, limited unless the comments are featured in some output files.

Obviously, using scripts requires the user to design the workflow of the process prior to conducting the analysis. \poy scripts can be created and saved using \emph{Script Editor} window of \poy interface or any conventional text editor or word processor (such as TextPad, TextWrangler, BBEdit, Emacs, MicrosoftWord, WordPad, or NotePad).

The scripts can be imported and executed using the \emph{POY Launcher} of the \emph{Graphical User Interface}, using the \commandstyle{run()} of the \emph{Interactive Console}or, at start up, by entering the names of the script files (without quotes and parentheses). This is extremely useful in cases when operations may take long time to complete eliminating waiting for a part of the analysis to finish in order to proceed to the next step.

There are three ways to import and run a script:
\begin{itemize}
    \item using the \emph{POY Launcher} of the \emph{Graphical User Interface};
    \item using the \commandstyle{run()} of the \emph{Interactive Console} (such as \texttt{run("script.txt")}, where \texttt{script.txt} is the name of the file containing the script);
   \item from the command line used to start \poy by including the filename(s) of the script (or multiple scripts) (such as \texttt{poy script.txt}.
\end{itemize}

It it critical to include the command \commandstyle{exit()} at the end of the script; otherwise \poy will be waiting for further instructions to be entered after executing the script's contents.

\begin{figure}
\centering
\begin{minipage}[c]{0.42\textwidth}
   		\includegraphics[width=\textwidth]{figures/commandlist.jpg}
\end{minipage}
\quad
\begin{minipage}[c]{0.53\textwidth}
	   	\includegraphics[width=\textwidth]{figures/script.jpg}
   	\end{minipage}
\caption{Using \poy scripts. Executing the list of commands the \emph{Interactive Console} (left)  is equivalent to running a script containing the same list (right). Note, that the header of the script is a comment, inclosed in ``(* *)'', that is ignored by \poy. Also note, that commands can either be listed in a row or in a column (compare \commandstyle{build()} and \commandstyle{swap()} in the console and in the script) and different arguments of the same command can either be specified separately or combined in a single argument list (compare \commandstyle{report()} in the console and in the script). (Both conventions are valid for interactive command submission and for scripts.)}
\label{fig:script}
\end{figure}

%%%%%%%%%%%%%%%%%%%%%%%%%%%%%%%%%%%%%%%%
%COMMANDS
%%%%%%%%%%%%%%%%%%%%%%%%%%%%%%%%%%%%%%%%
\chapter{\poy Commands}\label{commands}

\section{\poy command structure}

\subsection{Brief description} \label{commands}

\poy interprets and executes \emph{scripts} issued by the end user.  These can
come from the \emph{Graphical User Interface} and the command line in the \emph{Interactive Console} 
of the program, or from an input file. A script is a list of \emph{commands}, separated by any number of 
whitespace characters (spaces, tabs, or newlines). Each command consists of a name in lower case 
(\poylident), followed by a list of arguments separated by commas and enclosed in parentheses. 
Most of the arguments are optional, in which case \poy has default values.

In \poy, we recognize four types of command arguments: \emph{primitive values},
\emph{labeled values}, \emph{commands}, and \emph{lists of arguments}.

\paragraph{Primitive values} can be either an integer (\poyint), a real number
(\poyfloat), a string (\poystring), or a boolean (\poybool).

\paragraph{Labeled values} are lowercase identifiers (which are referred to as
\emph{label}), and an argument, separated by the colon character (``:'').

\paragraph{List of arguments} are several arguments enclosed in parenthesis and
separated by commas (``,'').

\paragraph{Commands} are standard commands that can affect the behavior of
another command when included in its list of arguments.

Thus, certain commands can function as arguments of other commands. Moreover,
some commands share arguments. Although such compositional use of commands
might seem complex, this structure provides much more intuitive
control and greater flexibility. The fact that the same logical operation that functions
in different contexts maintains
the same name (typically suggestive of its function), substantially reduces the number of
commands without limiting the number of operations. Using a linguistic analogy,
\poy specifies a large number of procedures by a more complex grammar (specific
combinations of commands and arguments), rather than by increasing the vocabulary
(the number of specific commands and arguments). For example, the command
\poycommand{swap} specifies the method of branch swapping. This command is
used to conduct a local search on a set of trees. In addition,
\poycommand{swap} functions as an argument for \poycommand{calculate\_support}
to specify the branch swapping method used in each pseudoreplicate during Jackknife or
Bootstrap resampling. \poycommand{swap} can also be used to set the parameters for
local tree search based on perturbed (resampled or partly weighted) data as an argument
of the command \poycommand{perturb}. Therefore, to take the maximum advantage of
\poy functionality, it is essential to get acquainted with the grammar of  \poy.

\subsection{Grammar specification}

The following is the formal specification of the valid grammar of a script in \poy:

\begin{verbatim}
script: = | command
        | command script

command: = LIDENT "(" argument list ")"

argument list: = |
            | arguments

arguments: = |
            | argument
            | argument "," arguments

argument: = | primitive
            | LIDENT
            | LIDENT ":" argument
            | command
            | "(" argument list ")"

primitive: = | INTEGER
            | FLOAT
            | BOOLEAN
            | STRING

LIDENT: = [a-z_][a-zA-Z0-9_]*

INTEGER: = [0-9]+

FLOAT: = | INTEGER
        | [0-9]+ "." [0-9]*

STRING: = """ [^"]* """


\end{verbatim}



The following examples graphically show a typical structure of valid \poy commands
formally defined above. The Figure \ref{simplecommand} illustrates
the syntax of the command \poycommand{swap}. The name of the
command, \poycommand{swap}, is followed by a list of two arguments,
\poyargument{tbr} and \poyargument{trees:2}, enclosed in parentheses
and separated by a comma. Note that \poyargument{trees:2} is a labeled-value
argument that contains a label (\texttt{trees}) and a value (\texttt{2})
separated by a colon.

\begin{figure}[htbp]
   \centering
   \includegraphics[width=0.60\textwidth]{doc/figures/fig-poycommand1.jpg}
   \caption{The structure of a simple \poy command. The entire command (highlighted
   in blue), consists of  a command name followed by a list of arguments (enclosed in red box).
   See text for details.}
   \label{simplecommand}
\end{figure}

Figure \ref{compositecommand} shows a more complex command structure, using the command 
\poycommand{perturb} as an example. This is a compound command because the list of its arguments 
contains another command, \poycommand{swap}. This means that executing \poycommand{perturb} 
performs a set of specified operations that contains a nested set of operations specified by \poycommand{swap}. 
Note also, that in contrast to the first labeled-values argument \poyargument{iterations}, the second 
labeled-values argument \poyargument{ratchet} has multiple values (a float and an integer). When 
multiple values are specified, they must be enclosed in parentheses and separated by a comma. The 
third argument is a command (\poycommand{swap}), therefore it is syntactically distinguished from 
other arguments, labeled and unlabeled alike, by having parentheses following the command name. It 
must be emphasized that the parentheses always follow the command name even if no arguments are 
specified. If no arguments are specified, a command is executed under its default settings provided it 
has default settings.  If a command has no default settings \emph{e.g.} \poycommand {transform}, then 
typing \poycommand{transform ()} does nothing. 

\begin{figure}[htbp]
   \centering
   \includegraphics[width=1.0\textwidth]{doc/figures/fig-poycommand2.jpg}
   \caption{A structure of a compound \poy command. Note that the list of arguments
   (enclosed in red box) includes a command (highlighted in blue). Also, note that
   \poyargument{ratchet} accepts multiple values, a float and an integer, that are inclosed int
   parentheses and separated by a comma. See text for details.}
   \label{compositecommand}
\end{figure}

\section{Notation}

Some arguments are obligatory, whereas others are not; some commands accept an
empty list of arguments, but others do not; some argument labels have
obligatory values, some have optional values. The \poy commands and arguments are listed alphabetically 
in the next section. In the descriptions of \poy commands below, the elements of \poy 
grammar are defined in the text using the following conventions:

\begin{itemize}
    \item A command that could be included in a \poy script (that is can be entered in the
    	interactive console or included in an input file) is shown in \poycommand{terminal} typeface.
    \item Optional items are inclosed in \poycommand{[square brackets]}.
    \item Primitive values are shown in \poycommand{UPPERCASE}.
\end{itemize}
\bigskip
Each command description entry contains the following sections:

\begin{itemize}
    \item The name of the command.
    \item The valid syntax for the command.
    \item A brief description of the command's function.
    \item The list of descriptions of valid arguments.
    \item Description of default settings.
    \item Examples of the command's usage.
    \item Cross references to related commands. 
\end{itemize}

\begin{statement}
\textbf{Default syntax}. The default syntax for all commands is the same: it includes the command name 
followed by empty parentheses, e.g. \poycommand{swap()}. However, within the descriptions of each 
command the default settings include the entire argument list for illustrative purposes only (i.e. in the case 
of \poycommand{swap()} the entire argument list appears as \poycommand{swap(trees:1, alternate, threshold:0, bfs)}).
\end{statement}

\begin{statement} \label{commandorder}
     \textbf{Command order}. The effect of the order of arguments in a command depends on the context. 
    If arguments are not logically interconnected, their order is not important. For example, the commands 
    \poycommand{build(10,randomized)} and \poycommand{build(randomized,10)} are equivalent. However, 
    executing the commands \poycommand{transform(tcm:(1,1),gap\_opening:4)} and 
    \poycommand{transform(gap\_opening:4,tcm:(1,1))} will produce different results because 
    \poycommand{gap\_opening} \emph{modifies} the values set by
    \poyargument{tcm}, while \poyargument{tcm} \emph{overrides} the values set by \poyargument{gap\_opening}.
\end{statement} 

\begin{statement}
    \textbf{Output files}. When an output file is specified, the file name (in double quotes and
    followed by a comma) must precede the argument, e.g. \poycommand{report("first\_trees", trees)}.
\end{statement}

%%%%%%
%The following statement is not included as there is no need for users to know anything about developer arguments
%\begin{statement}
%\textbf{Developer arguments}.  Certain command arguments are mainly useful to \poy developers, and
%those arguments are preceded by an underscore, e.g. \poycommand{\_breakvsjoin}.
%\end{statement}
%%%%%%



%%%%%%%%%%%%%%%%%%%%%%%%%%%%%%%%%%%%%%%%
%COMMAND REFERENCE
%%%%%%%%%%%%%%%%%%%%%%%%%%%%%%%%%%%%%%%%%

\section{Command reference}
\begin{command}{build}{buildcommand}

    \syntax{\obligatory{(\optional{argument list})}}
	
 	\begin{poydescription}
        Builds Wagner trees~\cite{farris1970}. Building multiple trees with a randomized addition of terminals allows for
        the evaluation of many more possible tree topologies and generates a
        diversity of trees for subsequent analysis. The arguments of the command
        \poycommand{build} specify the number of trees to be generated and
        the order in which terminals are added during a singe tree building procedure. During tree
        building, \poy reports in the \emph{Current Job} window of the ncurses interface
        which of the terminal addition strategies is currently used.
        

        By default \poy replaces the trees stored in memory with those generated
        in a subsequent build. For example, executing \poycommand{build(10)}
        followed by \poycommand{build(20)} will replace the 10 trees generated
        during the first build with 20 new trees. However, it might be desirable
        (for example, if computer memory were limited) to generate a large number of trees by
        appending trees from multiple separate builds. To keep trees from consecutive
        builds, a tree output file must be specified using the command ~\ccross{report} that must 
        precede the \poycommand{build} command. This will produce a file
        containing the trees appended from all builds. If the same file name is used for reporting
         trees for other analysis, the new trees are going to be appended. Alternatively, trees from different
        builds can be redirected to separate files if different file names are specified.
        
        The command \poycommand{build} is also used as
        an argument for the command \poycommand{calculate\_support}.
   	\end{poydescription}

	\begin{arguments}

	\argumentdefinition{as\_is}{} 
            {Indicates that in one of the trees to be built, the terminals are
            added in the order in which they appear in the 
            imported datafiles, and all others are built using a random addition
            sequence.}
            {asis}

        \argumentdefinition{branch\_and\_bound}{\optional{\poyfloat}}
            {Calculates the exact solution using the Branch and Bound algorithm
            ~\cite{hendy1982}. By default only one optimal tree is kept but
            the number of optimal trees to be retained can be specified by the
            argument \poyargument{trees}. The optional float value specifies the
            bound (either tree cost or likelihood score).}
            {branchandbound} 

        \argumentdefinition{constraint}{\optional{\poystring}}
            {Builds trees using the set of constraints specified by the consensus
            tree input file. If no input file is provided, the constraint is calculated as
            the strict consensus of the trees in memory. Every tree built
            using this method is subjected to the same randomization as wagner
            builds within each constraint.}
            {buildconstraint}

        \argumentdefinition{random}{}
            {Generates a tree at random.  All possible trees have equal
            probability.}
            {buildrandom}

		\argumentdefinition{randomized}{} 
            {Indicates that terminals are added in random order on every Wagner
            tree built. This is a default tree-building strategy.}
            {randomized}

        \argumentdefinition{trees}{\obligatory{\poyint}}
            {The integer value specifies the number of independent, individual
            Wagner tree builds. The label \poyargument{trees} is optional: it is
            sufficient to specify only the integer value. Therefore, \poycommand{build(5)} is
            equivalent to \poycommand{build(trees:5)}.  Note that  \poyargument{trees} is
            also used as an argument of the command~\nccross{swap}{swapcommand}
            but with different meaning.
            
            The value \texttt{0} generates no trees but it \emph{retains} all trees in memory.
            This is useful, for example, in the \ccross{bremer} support calculation,
            where instead of generating new trees per each node, the searches are
            performed on the trees in the neighborhood of the current trees in memory.}
            {treesbuild}
            
        \argumentdefinition{INTEGER}{}
            {The integer argument specifies the number of independent, individual
            Wagner tree builds. This is a shortcut of the argument \poyargument{trees}.}
            {}

        \argumentdefinition{of\_file}{\obligatory{\poystring}}
            {Imports tree file included in the file path of the argument. This command is
            useful for importing starting trees for calculating \ccross{bremer} support.
            In other contexts the command \ccross{read} can be used with the same effect.}
            {offile}

        \argumentdefinition{STRING}{}
            {This is a shortcut of the argument \poyargument{of\_file}.}
            {}

        \argumentdefinition{all}{}
            {Turns off all preference strategies for adding branches and simply tries all possible addition
            positions for all terminals.}
            {allbuild}
        
   \end{arguments}
      
   \poydefaults{trees:10, randomized}
       {By default, \poy will build 10 trees using a random addition sequence for
       each of them.}

	\begin{poyexamples}
		\poyexample{build(20)}
            {Builds 20 Wagner trees randomizing the order of terminal
            addition (note that because the argument \poyargument{random} is specified by default, 
            it can be omitted).}

		\poyexample{build(trees:20, randomized)}
            {A more verbose version of the previous example. By default a build
            is randomized, but in this case the addition sequence is explicitly
            set. For the total number of trees, rather than simply specifying \texttt{20},
            the label \texttt{trees} is used. The verbose version might be desirable
            to improve the readability of the script.}

		\poyexample{build(15, as\_is)}
            {Builds the first Wagner tree using the order of terminals in the first
            imported datafile and generates the remaining
            14 trees using random addition sequences.}
            
            	\poyexample{build(branch\_and\_bound, trees:5)}
            {Builds trees using branch and bound method and keeps up to
            5 optimal trees in memory.}
            
	\end{poyexamples}

\end{command}

\begin{command}{calculate\_support}{calculatesupport}

	\syntax{\obligatory{(\optional{argument list})}}

	\begin{poydescription} 
            Calculates the requested support values. \poy implements support
            estimation based on resampling methods (Jackknife~\cite{Farrisetal1996} 
            and Bootstrap~\cite{Felsenstein1985}) and Bremer support~\cite{Bremer1988, Kallersjoetal1992}. 
            The Jackknife and Bootstrap support values are computed as
            frequencies of clades recovered in strict consensus trees built in each resampling
            iteration. The consensus trees are based on best trees recovered in each replicate
            with zero-length branches collapsed.
            All the arguments of \poycommand{calculate\_support} command are optional and
            their order is arbitrary.
            
            \begin{statement}
  	  The placement of the root affects calculation of the support values.
	  Therefore, it is critical to specify the root prior to executing
	  \poycommand{calculate\_support}. See the description of the
	  command \nccross{set}{root} on how to specify the root.
	\end{statement}

            The \poycommand{calculate\_support} command does not output support values by default. The output of
            support values must be requested using the command~\ccross{report}. 
            This is particularly important for
            Jackknife and Bootstrap support values, as these sampling techniques
            do not require the presence of trees in memory. Therefore, it is
            possible to perform the sampling for support values \emph{before}
            the tree of interest has been found.
            
            \begin{statement}
                In the context of dynamic
                homology, the characters being sampled during pseudoreplicates
                are entire sequence fragments, not individual nucleotides.
                Consequently, the bootstrap and jackknife support values
                calculated for dynamic characters are not directly comparable to
                those calculated based on static character matrices. If it is
                desired to perform character sampling at the level of
                individual nucleotides, the dynamic characters must be
                transformed into static characters using \poyargument{static\_approx}
                argument of the command~\ccross{transform}
                prior to executing \poycommand{calculate\_support}.
                Alternatively, an output file in the Hennig86 format can be
                generated based on an implied alignment
                using~\ccross{phastwinclad} that can subsequently be analyzed
                using other programs.
                                
                It is important to remember that the local optimum for the dynamic
                homology characters can differ from that for the static homology characters
                based on the same sequence data. Therefore, it is recommended to perform an extra round of swapping on the
                 transformed data to reach the local maximum for the static
                 homology characters prior to calculating support values.
            \end{statement}
            
            \end{poydescription}

	\begin{arguments}
		\begin{argumentgroup}{Support calculation methods}
            {The following commands allow selecting among several methods for
            calculating support.} 

			\argumentdefinition{bremer}{}
                {Calculates Bremer support~\cite{Bremer1988, Kallersjoetal1992}
                for each tree in memory by performing independent constrained searches for each
                node. The parameters for the searches can be modified using arguments
                described under \emph{Search strategy}.} 
                {}

			\argumentdefinition{bootstrap}{\optional{\poyint}}
                {Calculates Bootstrap support~\cite{Felsenstein1985}. 
                The integer value specifies
                the number of resampling iterations (pseudoreplicates). If the value
                is omitted, 5 pseudoreplicates are performed by default.} 
                {}

			\argumentdefinition{jackknife}{\optional{([argument list])}}
                {Calculates Jackknife support~\cite{Farrisetal1996} using the 
                sampling parameters specified by the arguments. The arguments of
                \poyargument{jackknife} are optional and their order is arbitrary. If
                both values are omitted, the default values of each argument is used.}
                {}
                
                \begin{description}
                    \argumentdefinition{remove}{\obligatory{\poyfloat}}
                        {The value of the argument \poyargument{remove} specifies the
                        percentage of characters being deleted during a pseudoreplicate. The
                        default of \poyargument{remove} is \texttt{36} percent.}
                        {}
                     \argumentdefinition{resample}{\obligatory{\poyint}}
                        {The value of the argument \poyargument{resample} specifies the
                        number of resampling pseudoreplicates. The default of \\
                        \poyargument{resample} is \texttt{5}.}
                        {}
               \end{description}  
		\end{argumentgroup}

        \begin{argumentgroup}{Search strategy}
            {The calculation of the support values requires a local search,
            that is performed under the default settings unless the values
            of the following arguments are specified.}
		 
	     \argumentdefinition{build}{}
             {For calculating Bremer support, the integer value of
             \poyargument{build} specifies the number of independent
             Wagner tree builds per node. The integer value \texttt{0}
             (\texttt{build:0}) specifies that Bremer support values are
             calculated on the starting trees currently
             in memory, rather than on newly generated trees.
             The initial trees for calculating Bremer support
             can also be imported using the argument \poyargument{of\_file}
             of the command ~\nccross{build}{buildcommand}.
             
             For calculating Jackknife
             and Bootstrap supports, it specifies the number of
             Wagner tree builds per pseudoreplicate.  Single best trees from all
             psudoreplicates are used to calculate the support values. If
             multiple best trees are recovered in a pseudoreplicate, one 
             is selected. If \poyargument{build} is
             omitted from the argument list of \poycommand{calculate\_support},
             a single random addition Wagner tree per
             pseudoreplicate is built by default. This is equivalent to 
             \poycommand{build(trees:1, randomized)}. See
             \nccross{build}{buildcommand} for a detailed discussion of
             arguments of the command \poycommand{build}.}
             {buildarg}

        \argumentdefinition{swap}{}
            {Specifies the method and parameters for local tree search. If searching
            parameters are not specified, the search is performed under
            the default settings of ~\nccross{swap}{swapcommand}.} 
            {swaparg}
	     
        		\end{argumentgroup}

	\end{arguments}

    \poydefaults{bremer, build(trees:1, randomized), \\ swap(trees:1)}
    {By default \poy will calculate the bremer support for each tree in memory node by node.
    However, if no trees stored in memory, executing the command
    \poycommand{calculate\_support()} does not have any effect.}

	\begin{poyexamples} 

        \poyexample{calculate\_support(bremer)}
            {Calculates Bremer support values by performing
            independent searches for every node for every tree in memory. This is equivalent to executing \poycommand{calculate\_support()} (the default setting.)}
         
         \poyexample{calculate\_support(bremer, build(trees:0), swap(trees:2))}
            {Calculates Bremer support values by performing swapping on 
            each tree in memory for every node and keeping up to two
            best trees per search round.}
          
          \poyexample{calculate\_support(bremer, build(of\_file:"new\_trees"), \\
          swap(tbr, trees:2))}
            {Calculates Bremer support values by performing TBR swapping on 
            each tree in the file \texttt{new\_trees} located in the current
            working directory for every node and keeping up to two
            best trees per search round.}  
            
         \poyexample{calculate\_support(bootstrap)}
         {Calculates Bootstrap support values under default settings. This command
         is equivalent to \poycommand{calculate\_support(bootstrap:5, \\ build(trees:1,
         randomized), swap(trees:1))}.}
	
        \poyexample{calculate\_support(bootstrap:100, build(trees:5), \\
            swap(trees:1))}
            {Calculates Bootstrap support values performing one random resampling with
            replacement, followed by 5 Wagner tree builds (by random addition sequence)
            and swapping these trees under the default settings of the command 
            \poyargument{swap}, and keeping one minimum-cost tree. The procedure
            is repeated 100 times.}
        
        \poyexample{calculate\_support(jackknife:(resample:1000), 
            build(), \\ swap(tbr, trees:5))}
            {Calculates Jackknife support values randomly removing 36 percent of the
            characters (the default of \poycommand{jackknife}), building 10
            Wagner trees by random addition sequence (the default of
            \poycommand{build}), swapping these trees using \poyargument{tbr},
            and keeping up to 5 minimum-cost tree in the
            final swap per swap (totaling up to 50 stored trees per replicate). 
            The procedure is repeated 1000 times.}

	\end{poyexamples}
            
	\begin{poyalso}
		\cross{report}
        \ncross{supports}{supports}
        \ncross{graphsupports}{graphsupports}
	\end{poyalso}

\end{command}

\begin{command}{clear\_memory}{clearmemory}

	\syntax{\obligatory{(\optional{argument list})}}
	
	\begin{poydescription}
            Frees unused memory. Rarely needed, this is a useful command when the
            resources of the computer are limited. The arguments are optional and
            their order is arbitrary.
	\end{poydescription}
	
	\begin{arguments}
		\argumentdefinition{m}{}
            {Includes the alignment matrices in the freed memory.} 
            {clearmemoryalign}

		\argumentdefinition{s}{}
            {Includes the unused pool of sequences in the freed memory.}
            {clearmemoryseq}
	\end{arguments}
	
    \poydefaults{}{By default \poy clears all memory
    \emph{except} for the pool of unused sequences and the matrices used for the
    alignments.}
	
	\begin{poyexamples}
		\poyexample{clear\_memory(s)}
            {This command frees memory including all alignment matrices but keeping
            unused pool of sequences.}
	\end{poyexamples}

	\begin{poyalso}
		\cross{wipe}
	\end{poyalso}
	
\end{command}

\begin{command}{cd}{cd}

	\syntax{\obligatory{(\poystring)}}

	\begin{poydescription}
            Changes the working directory of the program. This command is useful
            when datafiles are contained in different directories. It also
            eliminates the need to navigate into the working directory before
            beginning a \poy session. To display the path of the current
            directory, use the command~\ccross{pwd}.
	\end{poydescription}

	\begin{arguments}
		\argumentdefinition{STRING}{}
            {The value specifies a path to a directory.}
            {}
	\end{arguments}
	
	\begin{poyexamples}

		\poyexample{cd ("/Users/username/docs/poyfiles")}
            {Changes the current directory to the directory \\
            \texttt{/Users/username/docs/poyfiles}.}

    \end{poyexamples}

    \begin{poyalso}
        \cross{pwd}
    \end{poyalso}

\end{command}

\begin{command}{echo}{}

    \syntax{\obligatory{(\poystring, output class)}} 
	
	\begin{poydescription} 
         Prints the content of the string argument into a specified type of output.
         Several types of output are generated by \poy  which are specified by the
         ``output class'' of arguments (see below). If no output-class arguments are
         specified, the command does not generate any output.
	\end{poydescription}

    \begin{arguments}
           \begin{argumentgroup}{Output class}
        \argumentdefinition{error}{}
            {Outputs the specified string as an error message (\texttt{stderr} in the
            flat interface).}
            {errorecho}

        \argumentdefinition{info}{}
            {Outputs the specified string as an information message (\texttt{stderr} in the
            flat interface).}
            {}

        \argumentdefinition{output}{\optional{\poystring}}
            {Reports a specified string (\texttt{stdout} in the flat interface) to screen or file, 
            if the filename string (enclosed in parentheses) is specified following \texttt{output} 
            and separated from it by a colon, ``:''.}
            {}
           \end{argumentgroup}
    \end{arguments}

	\begin{poyexamples}

        \poyexample{echo("Building with indel cost 1", info)}
            {Prints to the output window in the ncurses interface and to the
            standard error in the flat interface the message \texttt
{Building with indel cost 1}.}

        \poyexample{echo("Final trees", output:"trees.txt")}
            {Prints the string \texttt{Final trees} to the file \texttt{trees.txt}.}

        \poyexample{echo("Initial trees", output)}
            {Prints the string \texttt{Initial trees} to the output window in the
            ncurses interface, and to the standard output (\texttt{stdout} in the flat
            interface).}
    \end{poyexamples}

	\begin{poyalso}
		\cross{report}
	\end{poyalso}

\end{command}

\begin{command}{exit}{} 

	\syntax{\obligatory{()}}

	\begin{poydescription}
         Exits a \poy session. This command does not accept any argument.
         \poycommand{exit} is equivalent to the command \poycommand{quit}.

         \begin{statement}
         To interrupt a process without quitting a \poy session, use Control-C.
         It aborts a currently running operation but keeps all the previously accumulated
         data in memory. It does not abort the current session permitting entering new
         command and continuing the session.
        \end{statement}
	
	 \end{poydescription}
	 
    \begin{poyexamples}
        \poyexample{exit()}
            {Quits the program.}
    \end{poyexamples}

    \begin{poyalso}
        \cross{quit}
    \end{poyalso}

\end{command}

\begin{command}{fuse}{}

    \syntax{\obligatory{(\optional{argument list})}}

    \begin{poydescription}
            Performs Tree Fusing~\cite{goloboff1999} on the trees in memory. Tree Fusing is a
            genetic algorithm technique for avoiding being trapped at a local optimum
            by exchanging clades with identical composition of terminals between
            pairs of trees. Only \emph{one} pair of trees is evaluated during a single iteration.
            The size of the clades being exchanged is not specified.
    \end{poydescription}

    \begin{arguments}
        \argumentdefinition{keep}{\obligatory{\poyint}}
            {Specifies the maximum number of trees to keep between iterations.
            By default, the number of trees retained is the same as the number
            of starting trees.}
            {}

        \argumentdefinition{iterations}{\obligatory{\poyint}}
            {Specifies the number of iterations of tree fusing to be performed. The
            number of iterations is effectively the number of pairwise clade exchanges. 
            The default number of iterations is four times the number of retained
            trees (as specified by \poyargument{keep}).}
            {}

        \argumentdefinition{replace}{\obligatory{argument}}
            {Specifies the method for tree selection. Acceptable arguments
            are:
            \begin{description}
                \item[better] Replaces parent trees with trees of better cost
                produced during a fusing iteration.
                \item[best] Keeps a set of trees of the best cost regardless their origin.
            \end{description}
            The default is \texttt{best}.}
            {}

        \argumentdefinition{swap}{}
            {Specifies tree swapping strategy to follow each iteration of tree fusing.
            No swapping is performed under default settings.
            See the description of the command~\nccross{swap}{swapcommand}.}
            {}

    \end{arguments}
    
    \poydefaults{replace:best}
        {By default \poy performs fusing keeping the same number of trees per
        iterations as the number of the starting trees. The number of iterations is
        four times the number starting trees. During the procedure, only the best
        trees are retained. No swapping is performed subsequent to tree fusing.}
        
    \begin{poyexamples}
	
	\poyexample{fuse(iterations:10, replace:best, keep:100 swap())}
            {This command executes the following sequence of operations. In the
            first iteration, clades of the same composition of terminals are exchanged
            between two trees from the pool of the trees in memory. The cost of the
            resulting trees is compared to that of the trees in memory and a subset of
            the trees containing up to 100 trees of best cost is retained in memory.
            These trees are subjected to swapping under the default settings of
            \poycommand{swap}. The entire procedure is repeated nine more times.}
            
            \poyexample{fuse(swap(constraint))}
            {This command performs tree fusing  
            with modified settings for swapping that follows each iteration. Once
            a given iteration is completed, a consensus tree of the files in memory
            is computed and used as constraint file for subsequent rounds of swapping (see
            the argument \nccross{constraint}{swapconstraint} of the command
            \poycommand{swap}).}

     \end{poyexamples}
        
        \begin{poyalso}
        \cross{swap}
    \end{poyalso}

\end{command}

\begin{command}{help}{}

	\syntax{\obligatory{(\optional{argument})}}
	
	\begin{poydescription}
         Reports the requested contents of the help file on screen.
	\end{poydescription}
	
	\begin{arguments}
        \argumentdefinition{LIDENT}{}
            {Reports the description of the command, the name of which is specified by the
            LIDENT value.}
            {}

        \argumentdefinition{STRING}{}
            {Reports every occurrence in the help file of the expression specified by the string value.}
            {}
	\end{arguments}
	
	\poydefaults{}
        {By default \poy displays the entire content of the help file on screen.}
        
	\begin{poyexamples}
		\poyexample{help(swap)}
            {Prints the description of the command
            \poycommand{swap} in the \emph{POY Output} window of the ncurses
            interface or to the standard error in the flat interface.}

        \poyexample{help("log")}
            {Finds every command with text containing the substring \texttt{log} and
            prints them in the \emph{POY Output} window of the ncurses
            interface or to the standard error in the flat interface.}

	\end{poyexamples}

\end{command}


\begin{command}{inspect}{}

	\syntax{\obligatory{(\poystring)}} 

	\begin{poydescription}
        Retrieves the description of a \poy file produced by the command \ccross{save}. If the description was
        not specified by the user, \poycommand{inspect} reports that the
        description is not available. If the file is not a proper
        \poy file format, a message is printed in the \emph{POY Output}
        window of the ncurses interface or to the standard error of the flat interface.

        \poy files are not intended for permanent storage. They are recommended
        for temporary storage of a \poy session, checkpointing
        the current state of the search (to avoid losing data in case the computer or the
        program fails), or reporting bugs. \poy also automatically
        generates \poy files in cases of terminating errors (important exceptions are
        out-of-memory errors). 

    \end{poydescription}

    \begin{poyexamples}
        \poyexample{inspect("initial\_search.poy")}
            {Prints the description of the \poy file \texttt{initial\_search.poy}
            located in the current working directory in the \emph{POY Output}
            window of the ncurses interface or to the standard error in the flat
            interface. If, for example, the file was saved using
            the command \poycommand{save ("initial\_search.poy", "Results of
            Total Analysis")}, then the output message is: \texttt{Results of
            Total Analysis}.}
    \end{poyexamples}

    \begin{poyalso}
        \cross{save}
        \cross{load}
        \cross{cd}
        \cross{pwd}
    \end{poyalso}

\end{command}

\begin{command}{load}{}

	\syntax{\obligatory{(\poystring)}} 

	\begin{poydescription}
            Reads and imports a \poy file, the name of which (specified by the command
             \poycommand{save}) is included in the string argument.
            All the information of the current \poy session will be replaced
            with the contents of the \poy file. If the file is not in proper \poy file
            format, an error message is printed in the \emph{POY Output}
            window of the ncurses interface, or the standard error in the flat interface.
            See the description of the command \ccross{save} on the \poy file
            and its usage.

            \poy files are not intended for permanent storage: they are recommended
        for temporary storage of a \poy session, checkpointing
        the current state of the search (to avoid losing data in case the computer or the
        program fails), or reporting bugs. \poy also automatically
        generates \poy files in cases of terminating errors (an important exception is
        out-of-memory error).
	
	\end{poydescription}

    \begin{poyexamples}
        \poyexample{load("initial\_search.poy")}
            {Reads and imports the contents of the \poy file
            \texttt{initial\_search.poy}, located in the current working
            directory.}

        \poyexample{load("/Users/andres/test/initial.poy")}
            {Reads and imports the contents of the \poy file \texttt{initial.poy}
             in the absolute path described by the argument.}
    \end{poyexamples}

    \begin{poyalso}
        \cross{save}
        \cross{inspect}
        \cross{cd}
        \cross{pwd}
    \end{poyalso}

\end{command}

	 
\begin{command}{perturb}{}

	\syntax{\obligatory{(\optional{argument list})}}

	\begin{poydescription} 
        Performs branch swapping on the trees currently in memory using a temporarily modified 
        (``perturbed'') characters. Once a local optimum is found for 
        the perturbed characters,
        a new round of swapping using the original (non-modified) characters is
        performed. Subsequently, the costs of the initial and final trees are
        compared and the best trees are selected. If there are $n$ trees in
        memory prior to searching using \poycommand{perturb}, then the  $n$ best
        trees are selected at the end. For example, if there are 20 trees currently in memory,
        20 individual \poycommand{perturb} procedures will be performed (each
        procedure starting with one of the 20 initial trees), and 20
        final trees are produced.        
        This command allows for movement from a local search optimum in the tree space by
        \emph{perturbing} the character space (hence the name). The
        arguments specify the type of perturbation (\poyargument{ratchet},
        \poyargument{resample}, and \poyargument{transform}), the parameters of the
        subsequent search (\poyargument{swap}), and the number of iterations 
        of the \poycommand{perturb} operation (\poyargument{iterations}).
        
        No new Wagner trees are generated following the perturbation of the
        data; the search is performed by local branch swapping (specified by
        \poyargument{swap}). If \poycommand{perturb} is executed with no
        trees in memory, an error message is generated. The arguments of
        \poycommand{perturb} are optional and their order is arbitrary. 
 	\end{poydescription}

	\begin{arguments}

         \argumentdefinition{iterations}{\obligatory{\poyint}}
            {Repeats (iterates) the \poycommand{perturb} procedure for the
            number of times specified by the integer value. The number of iterations
            is reported in the \emph{Current Job} window of the ncurses interface
            and to the standard error in the flat interface.}
            {}

        \argumentdefinition{ratchet}{\optional{(\poyfloat, \poyint)}}
            {Perturbs the data by implementing a variant of the parsimony
            ratchet~\cite{Nixon1999}. For unaligned data,
            \poyargument{ratchet} randomly selects and reweighs a fraction of
            sequence fragments (\emph{not} individual nucleotides) specified
            by the float
            (decimal) value, upweighted by a factor specified by the integer
            value (severity). For static matrices, such as those obtained
            using the command~\nccross{transform}{transformcommand}, 
            \poycommand{ratchet} randomly selects and
            reweights individual nucleotide positions (column vectors), as in Nixon's
            original implementation ~\cite{Nixon1999}.
            
            Under default settings,
            \poyargument{ratchet} selects 25 percent of characters and upweights
            them by a factor of 2.  Unless \poyargument{ratchet} is performed
            under default settings (that does not require the specification of the
            fraction of data to be reweighted and the severity value), both
            values must be specified in the proper order and separated by a comma.
            This argument is only used as an argument for \poycommand{perturb}.}
            {}

        \argumentdefinition{resample}{\obligatory{(\poyint, \poylident)}}
            {Resamples the data (characters or terminals) in random order with
            replacement. The \poyargument{resample} string consists of an 
            integer value
            specifying the number of items to be resampled (followed by a comma)
            and a lident value specifying whether characters or terminals
            (values \poycommand{characters} and \poycommand{terminals}, respectively)
            are to be resampled. Specifying both values
            is required. No default settings are available for \poyargument{resample}. This
            command is only used as an argument of \poycommand{perturb}.}
            {}

         \argumentdefinition{swap}{}
            {Specifies the method of branch swapping for a local tree search
            based on perturbed data. If the argument \poyargument{swap}
            is omitted, the search is
            performed under default settings of the
            command~\nccross{swap}{swapcommand}.}
            {swaparg}

        \argumentdefinition{transform}{}
            {Specifies a type of character transformation to be performed
            \emph{before} executing a \poycommand{perturb} procedure.
            See the command~\nccross{transform}{transformcommand} for
            the description of the methods of character type transformations
            and character selection.}
            {}

	\end{arguments}

    \poydefaults{ratchet, swap (trees:1)}
        {When no arguments specified, \poy performs the ratchet procedure under default
        settings.}
	
	\begin{poyexamples}
	
	\poyexample{perturb(resample:(50,terminals), iterations:10)}
	{Performs 10 successive repetitions of random resampling of 50 percents of
	terminals with replacement. Branch swapping is performed using
	alternating SPR and TBR, and and keeping one minimum-cost
	tree (the default of \poycommand{swap}).}
	
	\poyexample{perturb(iterations:20, ratchet:(0.18,3))}
            {Performs 20 successive repetitions of a variant of the ratchet (see
            above) by randomly selecting 18 percent of the characters (sequence
            fragments) and upweighting them by a factor of 3. Branch swapping is
            performed using alternating SPR and TBR, and keeping one
            optimal tree (the default of \poycommand{swap}).}

           \poyexample{perturb(iterations:1, transform (tcm:(4,3)))}
            {Transforms the cost regime of all applicable characters (\emph{i.e.} molecular 
            sequence data) to the new cost regime specified by
            \poyargument{transform} (cost of substitution 4 and cost of indel 3).
            Subsequently a single round of branch swapping is
            performed using alternating SPR and TBR, and and keeping one
            optimal tree (the default of \poycommand{swap}).}
            
            \poyexample{perturb(ratchet:(0.2,5), iterations:25, swap(tbr, trees:5))}
            {Performs 25 successive repetitions of a variant of the ratchet (see
            above) by randomly selecting 20 percent of the characters (sequence
            fragments) and upweighting them by a factor of 5. Branch swapping is
            performed using TBR and keeping up to 5 optimal trees in each iteration.}
            
            \poyexample{perturb(transform(static\_approx), ratchet:(0.2,5), \\
            iterations:25, swap(tbr, trees:5))}
            {Transforms all applicable (\emph{i.e.} dynamic homology sequence characters) using
            \poyargument{transform} into static characters. 
            Therefore, the subsequent ratchet is performed at the level of
            individual nucleotides (as in the original implementation), \emph{not}
            sequence fragments. Thus, ratchet is performed by selecting 20 percent of
            the characters (individual nucleotides) and upweighting them by a factor of 5.
            Branch swapping is performed using TBR and keeping up to 5 optimal trees in 
            each iteration as in the example above.}

	\end{poyexamples}
               
	\begin{poyalso}
		\ncross{swap}{swapcommand}
        		\ncross{transform}{transformcommand}
	\end{poyalso}
	
\end{command}

\begin{command}{pwd}{pwd}

	\syntax{\obligatory{()}}
	
	\begin{poydescription}
         Prints the current working directory in the \emph{POY Output} window of
         the ncurses interface and the standard error (stderr) of the flat interface.
         The command \poycommand{pwd} does not have arguments. The default
         working directory is the shell's directory when \poy started.
	\end{poydescription}
	
	\begin{poyexamples}
		\poyexample{pwd()}
            {This command generates the following message: ``The current
            working directory is /Users/myname/datafiles/''. The actual reported
            directory will vary depending on the directory of the shell when
            \poy started, or if it has been changed using the command
            \poycommand{cd()}.}
    \end{poyexamples}

    \begin{poyalso}
        \cross{cd}
    \end{poyalso}

\end{command}

\begin{command}{quit}{}
	
	\syntax{\obligatory{()}}
	
	\begin{poydescription}
        Exits \poy session. This command does not have any arguments
        \poycommand{quit} is equivalent to the command \poycommand{exit}.
        \end{poydescription}

	\begin{statement}
	 To interrupt a process without quitting a \poy session, use Control-C.
	 It aborts a currently running operation but keeps all the previously accumulated
	 data in memory. It does not abort the current session permitting entering new
	 commands and continuing the session.
	\end{statement}

    \begin{poyexamples}
        \poyexample{quit()}{Quits the program.}
    \end{poyexamples}
    
    \begin{poyalso}
        \cross{exit}
    \end{poyalso}
\end{command}

\begin{command}{read}{}

    \syntax{\obligatory{(\optional{argument list})}}  

	\begin{poydescription} 
        Imports data files and tree files.  Supported formats are ASN1, Clustal, FASTA,
        GBSeq, Genbank, Hennig86, Newick, NewSeq, Nexus, PHYLIP, POY3,
        TinySeq, and XML. Filenames should be enclosed in quotes and, if multiple
        filenames are specified, they must be separated by commas.
        \poycommand{read} automatically detects the type of the input file.
        \poycommand{read} can use wildcard expressions (such as *) to
        refer to multiple files in a single step. For example, \poycommand{read("biv*")} 
        imports all data files the names of which start
        with \texttt{biv} or \poycommand{read("*.ss")} imports all files with
        the extension \texttt{.ss} (given that the data files are in the current directory).
        Specifying filename(s) is
        obligatory: an empty argument string, \poycommand{read()}, results in no
        data being read by \poy. The list of imported files and their content
        can be reported on screen or to a file using \poycommand{report(data)}.
        
        If a file is loaded twice, \poy issues an error message but this will not
        interfere with subsequent file loading and execution of commands.
        
       \poy automatically reports in the \emph{POY Output} window of the ncurses
            interface or to the standard error in the flat interface the names
            of the imported files, their file type, and a brief description of
            their contents. A more comprehensive report on the contents of the imported
            files can be requested (either on screen or to a file) using the argument
            \poyargument{data} of the command \ccross{report}.

        \begin{statement}
            Although \poy recognizes multiple data file formats, it does not
            interpret all of their contents. Instead, it will recognize and import
            only character data and ignore other content (such as blocks of
            commands, \emph{etc.}). For certain data file formats, \poy will interpret
            additional information as detailed for each file type below.
            It is important, however, to verify that the data was interpreted properly (using
            the command \poycommand{report}).
        \end{statement}
          
        \begin{statement}
            Unlike many phylogenetic programs, \poy does not clear the memory
            upon reading a second file. Instead, any subsequently read files
            will be added to the total data being analyzed.  If a \emph{new} taxon
            appears in a file, then it is be assigned missing data for all
            previously loaded characters. If a taxon does \emph{not} appear in a
            file, missing data are assigned for the characters that appearing in it. 
            
            To eliminate the imported data and then to input a new data
            the \poycommand{wipe()} command must be issued
            first. 
        \end{statement}
        
        \begin{statement}
             If one of the terminal names in an imported molecular file contains
             a space, `` '', \poy issues a warning. This also occurs if a
             taxon name appears to match a nucleotide sequence.

             If one of the terminal names in an imported molecular file contains
             an ``at'' (\texttt{@}) or a percentage (\%) symbols, the file will not be loaded because
             it may cause the program to crash when reporting results.
        \end{statement}
	\end{poydescription}

	\begin{arguments}

	  \begin{argumentgroup}{Data file types}
	  To import data files, individual data file names must be included in the list of
	  \poycommand{read} arguments, enclosed in quotes, and separated
	  by commas. If no data file types are specified, the types of the imported
	  files are recognized automatically. To specify the data type,
	  an additional argument explicitly denoting the data type,
	  is included; it is followed by a colon (``:'') and the
	  list of data file names (enclosed in parentheses), separated by commas and enclosed in quotes. This
	  format prevents any ambiguity in importing multiple data file types
	  simultaneously (\emph{i.e.} included in an argument list of a single \poycommand{read})
	  command.

          \argumentdefinition{STRING}{}
              {Reads the file specified in the path included in the string argument.
              A path can be absolute or relative to the current
              working directory (as printed by \poycommand{pwd()}). The file type
              is recognized automatically.

              Molecular files are assumed to contain nucleotide sequences. Valid
              files to read using this command are: tree files using
              parenthetical notation (newick, \poy trees), Hennig86 files, Nona
              files, Sankoff character files as used in POY 3, FASTA files (and
              virtually any file generated by Genbank), and NEXUS files. Only
              taxon names, trees, characters, and cost regimes will be imported
              from each one of this files, no other commands are currently
              recognized.}
              {}

        \argumentdefinition{aminoacids}{\obligatory{(\poystring list)}}
            {Specifies that the data listed in the string argument
            are amino acid sequences in FASTA format.} 
            {}
            
	\begin{statement}
            Currently, IUPAC ambiguity codes for aminoacids are \emph{not} supported 
            and inputing files that contain aminoacid data with ambiguities results in
            an error message.
        \end{statement}
        
        \argumentdefinition{annotated}{\obligatory{(\poystring list)}}
            {Specifies that the data listed in the string argument are chromosomal
            sequences with pipes (`` $\vline$ '') separating individual
            loci. This data type allows for locus-level rearrangements specified by
            the argument \nccross{dynamic\_pam}{dynamicpam} of the command
            \nccross{transform}{transformcommand}. Locus homologies are
            determined dynamically, but based on annotated regions~\cite{vinh2006}.} 
            {}

        \argumentdefinition{breakinv}{\obligatory{(STRING, STRING, 
        [orientation:\poybool, init3D:\poybool])}}
            {An \\ enhancement of the data file type \poyargument{custom\_alphabet} allowing
            rearrangement events specified using \poycommand{dynamic\_pam()}. Syntactically,
            breakinv data type is identical to custom\_alphabet data type.} 
            {breakinv}

        \argumentdefinition{chromosome}{\obligatory{(\poystring list)}}
            {Specifies that the data in the files listed in the string argument
            are chromosomal sequences without predefined locus boundaries.
            Specifying that imported sequences are chromosome type data enables
            the application of parameter options that optimize chromosome-level
            events such as rearrangements, inversions, and large-scale
            insertions and deletions (including duplications). These parameter
            options (\emph{e.g.} inversion cost) are specified using the
            argument \poycommand{dynamic\_pam} in the
            command~\nccross{transform}{transformcommand}.  
            Unlike when using \poycommand{annotated} data type,
            both locus-level and nucleotide-level homologies
            are determined dynamically~\cite{vinh2007}. If chromosome sequences were imported as
            nucleotide type data, they can be converted to chromosome type data
            using the argument \poyargument{seq\_to\_chrom} of
            \nccross{transform}{transformcommand}.} 
            {}
            
            \argumentdefinition{custom\_alphabet}{\obligatory{(STRING, STRING, 
        [orientation:\poybool, init3D:\poybool])}}
            {Reads the data in the user-defined alphabet format. The first string argument is
            the name of a datafile that contains custom-alphabet sequences in FASTA format. 
            The characters can be (but are not required to be) separated by spaces.
            
            The second string argument is the name of a custom-alphabet input file that contains two parts:
            an alphabet itself, where the alphabet elements are separated by spaces, and a
            transformation cost matrix. The elements in an alphabet can be letters, digits, or
            both, as long as one element is not a prefix of another  (``prefix-free''). For
            example, the following pairs of custom-alphabet elements are \emph{not} valid
            because the first is a prefix of the second (which would prevent the proper parsing of
            an input file): \texttt{AB} and \texttt{ABBA} or \texttt{122} and \texttt{122X}.
            The transformation cost matrix contains the rows and columns in which the
            positions from left to right and top to bottom correspond to the sequence of the
            elements as they are listed in the alphabet. Aa extra rightmost column and lowermost
            row correspond to a gap. It is impotant that the cost matrix must be symmetrical. Here
            is an example of a valid custom alphabet input file:
       	  
	  \texttt{alpha beta gamma delta \\
            0 2 1 2 5 \\
            2 0 2 1 5 \\
            1 2 0 2 5 \\
            2 1 2 0 5 \\
            5 5 5 5 0}
            
           In this example, the cost of transformation of \texttt{alpha} into \texttt{beta} is \texttt{2},
           and cost of a deletion or insertion of any of the four elements costs \texttt{5}.
           
           An example of a corresponding input file:
       
           \texttt{$>$Taxon1\\
	alphabetagammadelta\\
	$>$Taxon2\\
	alphabetabetagammadelta\\
	$>$Taxon3\\
	alphabetabetadelta}
	
	The optional arguments of \poyargument{custom\_alphabet} include \poyargument{orientation}
	and \poyargument{init3D}, both of which require obligatory boolean values. The argument
	\poyargument{orientation} allows the user to specify the orientation of custom-defined alphabet
	characters. The \emph{tilde} symbol (``$\sim$'') preceding an alphabet character indicates
	the negative orientation. The options are \poyargument{orientation:true}
	or \poycommand{orientation:false}. The default option is \texttt{true}.
	
	The argument \poyargument{init3D} indicates that if program will calculate in advance
	the medians for all triplets of characters (a, b, c). The options are \poyargument{init3D:true} or
	\poyargument{init3D:false}. The default option is \texttt{true}.
	
	\poycommand{custom\_alphabet}
	can be transformed into \poycommand{breakinv} using \\ \poycommand{transform()}.}
	 {customalphabet}
        
        \argumentdefinition{genome}{\obligatory{(\poystring list)}}
            {Specifies that the data listed in the string argument are
            multichromosomal nucleotide sequences with the ``\atsymbol'' sign 
            separating individual chromosomes. This data type
            allows for chromosome-level rearrangements specified by
            the argument
            \nccross{dynamic\_pam}{dynamicpam} of the command
            \nccross{transform}{transformcommand}. Chromosome
            homologies are determined dynamically using distance
            threshold levels specified by the argument
            \nccross{chrom\_hom}{chromhom}
            of \nccross{transform}{transformcommand}.} 
            {}
            
        \argumentdefinition{nucleotides}{\obligatory{(\poystring list)}}
            {Specifies that the data in the list of files hold nucleotide
            sequences in FASTA format.} 
            {}
            
	\begin{statement}
            By default, upon importing prealigned sequence data, all the gaps are
             removed and the sequences are treated as dynamic homology characters.
             To preserve the alignment the data must be imported using the
             \poyargument{prealigned} argument of the command \poycommand{read}.
          \end{statement}
        
        \argumentdefinition{prealigned}{\obligatory{(read argument,
        tcm:\poystring)}}
            {Specifies that \\ the input sequences are prealigned and
            should be assigned the transformation cost matrix from the 
            input file defined by the string argument. (See the argument~\nccross{tcm}{transformtcm} 
            of the command \poycommand{transform}.)}
            {}
        
        \argumentdefinition{prealigned}{\obligatory{(read argument,
        tcm:(\poyint, \poyint))}}
            {Specifies that\\ the input sequences are prealigned and should be
            assigned substitution and indel costs as defined by the
            \poyargument{tcm} argument. (See the argument~\nccross{tcm}{transformtcm} 
            of the command \poycommand{transform}.)}
            {prealigned2}

	\end{argumentgroup}
		
	\end{arguments}

    \poydefaults{}{If no data files are specified, \poy does nothing. If however,
    data files are listed but character type is not indicated, \poy automatically
    detects data file types and interprets sequence files as nucleotides-type data.}

	\begin{poyexamples}
	
        \poyexample{read("/Users/andres/data/test.txt")}
            {Reads the file \texttt{test.txt} located in the path
            ``/Users/andres/data/''.}

        \poyexample{read("28s.fas", "initial\_trees.txt")}
            {Reads the file \texttt{28s.fas} and loads the trees in parenthetical notation
            of the file \texttt{initial\_trees.txt}.}

        \poyexample{read("SSU*", "*.txt")}
            {Reads all the files the names of which start with \texttt{SSU}, and all the
            files with the extension \texttt{.txt}. The types of the datafiles are determined
            automatically.}
        
        \poyexample{read(nucleotides:("chel.FASTA", "chel2.FASTA"))}
            {Reads the files \texttt{chel.FASTA} and \texttt{chel2.FASTA}, containing nucleotide
            sequences.}

        \poyexample{read(aminoacids:("a.FASTA", "b.FASTA", "c.FASTA"))}
            {Reads the amino acid sequence files \texttt{a.FASTA}, \texttt{b.FASTA}, and
            \texttt{c.FASTA}.}

        \poyexample{read("hennig1.ss", "chel2.FASTA", aminoacids:("a.FASTA"))}
            {Reads the Hennig86 file \texttt{hennig1.ss}, the FASTA file \texttt{chel2.FASTA}
            containing nucleotide sequences, and the amino acid
            sequence file \\ \texttt{a.FASTA}.}
            
        \poyexample{read(custom\_alphabet:("my\_data", "alphabet"))}
        {Reads the first file, \texttt{my\_data}, containing data in the format of a custom
        alphabet, which is defined in the second input file, \texttt{alphabet}. By default, the
        forward and reverse orientation (\texttt{orientation:true}) of custom-alphabet
        characters is considered and prior calculation of medians for their triplets
        (\texttt{init3D:true}) is performed.}
            
           \poyexample{read(annotated:("filea.txt", "fileb.txt"), \\ chromosome:("filec.txt"))}
           {Reads three files containing chromosome-type sequence data.
           The sequences in two files,
            \texttt{filea.txt} and \texttt{fileb.txt}, contain pipes (``~$\vline$~'') separating
            individual loci, whereas the sequences in the third, are without
            predefined boundaries.}
            
            \poyexample{read(genome:("mt\_genomes", "nu\_genomes")}
            {Reads two files containing genomic (multi-chromosomal) sequence data.}

	   \poyexample{read(prealigned:("18s.aln", tcm:(1,2)))}
	    {Reads the prealigned data file \texttt{18s.aln} generated from the nucleotide file \texttt{18s.FASTA}
	    using the the transformation costs \texttt{1} for substitutions and \texttt{2} for indels.}
	
	\poyexample{read(prealigned:(nucleotides:("*.nex"), tcm:"matrix1"))}
	    {Reads character data from all the Nexus files as prealigned data using the the transformation cost
	    matrix from the file \texttt{matrix1}.}

	\end{poyexamples}

	\begin{poyalso}
          \cross{report}
	\end{poyalso}

\end{command}

\begin{command}{rediagnose}{rediagnose}

	\syntax{\obligatory{()}}

	\begin{poydescription}
        Performs a re-optimization of the trees currently in memory. This
        function is only useful for sanity checks of the consistency of the data.
        Its main usage is for the \poy developers. This command does not have
        arguments.
	\end{poydescription}

    \begin{poyexamples}
        \poyexample{rediagnose()}{See the description of the command.}
    \end{poyexamples}

\end{command}

\begin{command}{recover}{}
    \syntax{\obligatory{()}}

    \begin{poydescription}
            Recovers the best trees found during swapping, even if the swap was
            cancelled. This command functions only if the argument \nccross{recover}{recoverarg} 
            was included in a previously executed 
            (in the current \poy session) command \poycommand{swap}. Otherwise, it has no effect.
	
	The trees imported by \poycommand{recover} are appended to those currently
	stored in memory.
	
	Note that using recovered trees is not intended for temporary storage of trees.
	It is useful only as an intermediary operation in a given part of a \poy session. When
	other commands that require clearing memory are executed (such as
	\poycommand{build}, \poycommand{calculate\_support}, or another
	\poycommand{swap}),
	the trees stored by \poyargument{recover} can no longer be retrieved.
            
    \end{poydescription}

    \begin{poyexamples}
        \poyexample{recover()}{If the command \poycommand{swap} (executed
        earlier in the current \poy session) contained the argument \poyargument{recover},
        for example, \poycommand{swap(tbr,recover)}, this command will restore the best
        trees recovered during swapping.}
    \end{poyexamples}

    \begin{poyalso}
        \ncross{swap}{swapcommand}
        \ncross{recover}{recoverarg}
    \end{poyalso}
\end{command}

\begin{command}{redraw}{}

	\syntax{\obligatory{()}}

	\begin{poydescription}
        Redraws the screen of the terminal. This command is only used in the ncurses
        interface, other interfaces ignore it. \poycommand{redraw} clears the
        contents of the \emph{Interactive Console} window but retains the contents
        of the other windows. It does not affect the state of the search and the data
        currently in memory.
	\end{poydescription}

    \begin{poyexamples}
        \poyexample{redraw()}{See the description of the command.}
    \end{poyexamples}

\end{command}

\begin{command}{rename}{}

	 \syntax{\obligatory{(\optional{argument list})}}  
	
	\begin{poydescription} 
        Replaces the name(s) of specified item(s) (characters or terminals). This command allows 
        for substituting taxon names and helps merging multiple datasets without modifying the original
        datafiles. More specifically, it can be used, for example, (1) for housekeeping purposes,
        when it is desirable to maintain long verbose taxon names (such as catalog or GenBank
        accession numbers) associated with the original datafiles but avoid reporting these 
        names on the trees; (2) to provide a single name for a terminal in cases where the corresponding
        data is stored in different files under different terminal names; and (3) to simply change an
        outdated or invalid terminal name.
        
        The command consists of a terminal or character identifier followed by a comma and then by
        either a string containing a synonymy file or a pair (or pairs) of strings containing the names of
        items being renamed.
	\end{poydescription}  
          
	\begin{statement}
            In order to change taxon names, the command \poycommand{rename} must be
            executed \emph{before} importing the datafiles (see command \nccross{read}{read})
            that contain character data the taxa to be renamed.
          \end{statement}
          
          \begin{statement}
            Once the command \poycommand{rename} is applied, subsequent commands 
            must refer to the terminals using the new, substitute names. This is critical, for example,
            when importing a terminals file using the command~\nccross{select}{select} or specifying
             a root using the command~\ccross{set}.
          \end{statement}
          
          \begin{arguments}
		\begin{argumentgroup}{Identifiers}
		{The identifiers specify whether terminals or characters are being renamed. An identifier
		must precede the subsequent arguments.}
		
		\argumentdefinition{characters}{}
                {Specifies that the subsequently subsequently items to be renamed are characters.} 
                {}
		\argumentdefinition{terminals}{}
                {Specifies that the subsequently subsequently items to be renamed are terminals.} 
                {}
		\end{argumentgroup}
	      
	      \begin{argumentgroup}{Specifying items to be renamed}
	      {These arguments allow to specify the items to be renamed either individually (by 
	      using a pair of string arguments) or in a group (by importing a \emph{synonymy} file.
	      The latter is useful when there are multiple items to be renamed and/or when it is
	      desirable to substitute a single name for  multiple ones.}
	      
	      \argumentdefinition{STRING}{}
                {Specifies the name of the file (a \emph{synonymy} file) that contains the list of
                terminals or characters to be renamed. The synonymy file has the following structure:
                each line contains a list of synonyms (two or more) separated by spaces. The name of the
                item listed first is going to be substituted for all the subsequently listed names. Consider,
                for example, a two-line synonymy file below:
                
                \texttt{alpha beta gamma \\
                delta 1\\}
                 
                 When this file is imported, the items \texttt{beta} and \texttt{gamma} will be
                 renamed as \texttt{alpha} and the item \texttt{1} will be renamed as \texttt{delta}
                 in all subsequently imported datafiles.}
                {}
                \argumentdefinition{(STRING, STRING)}{}
                {Specifies the names of individual items to be renamed. The first item is renamed
                as the second item: specifying \texttt{("alpha",\\"beta")} renames the character or taxon
                \texttt{alpha} to \texttt{beta}. To specify multiple pairwise name substitution, several 
                name pairs can be listed: \texttt{("alpha","beta"),("gamma","delta").}}
                {}
                
                \begin{statement}
                Note that when \poycommand{rename} is applied by specifying pairs of
                 synonyms in the command's argument,
                the substitute name is listed \emph{second}. However,  the substitute name 
                appears \emph{first} in a synonymy file, followed by one or more synonyms.
                \end{statement}
	      \end{argumentgroup}
	      
          \end{arguments}
          
          \begin{poyexamples}
        \poyexample{rename(terminals,"synfile")}{This command renames terminal names
            contained in the synonymy file \texttt{synfile} in all subsequently imported datafiles.}
            \poyexample{rename(terminals,("Mytilus\_sp","Mytilus\_edulis"))}{This command
            renames terminal file \texttt{Mytilus\_sp} as \texttt{Mytilus\_edulis} in all subsequently
            imported datafiles.}
    \end{poyexamples}

\end{command}
        
\begin{command}{report}{}

	\syntax{\obligatory{(\optional{argument list})}}

	\begin{poydescription} 
        Outputs the results of current analysis or loaded data in the \emph{POY Output}
        window of the ncurses interface, the standard output of the flat
        interface, or to a file. To redirect the output to a file, the file name in 
        quotes and followed by a comma must be included in the argument list
        of \poycommand{report}. All arguments for \poycommand{report} are
        optional. 
	\end{poydescription}

	\begin{arguments}

        \begin{argumentgroup}{Reporting to files}{}

            \argumentdefinition{STRING}{}
                {Specifies the name of the file to which all the specific types of report outputs,
                designated by additional arguments, are printed. If no additional arguments
                are specified, the data, trees, and diagnosis are reported to that file by
                default.
                
                A string (text in quotes) argument is interpreted as a filename.
                Therefore, \texttt{"/Users/andres/text"} represents the file \texttt{text} in
                the directory \texttt{/Users/andres} (in Windows
                \texttt{C:$\backslash$users$\backslash$andres}). If no path is given, the path
                is relative to the current working directory as printed by
                \poycommand{pwd()}.
                usage.} 
                {}
        \end{argumentgroup}
                
	\begin{argumentgroup}{Terminals and characters}
            {This set of arguments reports the current status of terminals and
            characters from the imported data files. }
		
            \argumentdefinition{compare}{\obligatory{(\poybool, identifiers,
            identifiers)}}
                {If the boolean argument is set to \texttt{false}, the command
                reports the ratios of all pairwise distances to their maximum length
                for the characters specified by character identifiers. If the boolean
                argument is set to \texttt{true}, the complement sequences for
                the characters specified by the second identifier are computed prior
                to reporting the distance.}{}

            \argumentdefinition{cross\_references}{\optional{identifiers\optional{STRING}}}
                {Reports a table with terminals being analyzed in rows, and the
                data files in columns. A plus sign (``+'') indicates that data for a given
                terminal is
                present in the corresponding file; a minus sign (``-'') indicates that it is
                not. \poyargument{cross\_references} is a very useful tool for visual
                representation of missing data.
                
                Under default settings, cross-references are reported for
                all imported datafiles. To report cross-references for some of
                the fragments within a given file, a single character, or a subset
                of characters, optional arguments must be specified. A combination of
                a character identifier (see command  \nccross{select}{select}) and
                the file names (specified in the the string value) is used to select specific
                datafiles to be cross-referenced. For example, if a
                command \poycommand{cross\_references:names:\\("file1")} is
                executed, the output is produced only for \texttt{file1}.
                
                The argument \poyargument{cross\_references:all} generates
                a table that shows presence and absence of fragments contained
                within each file. If each datafile contains a
                single fragment,  executing \poyargument{cross\_references:all}
                is equivalent to executing \poyargument{cross\_references}.
                
                By default, the cross-reference table is printed on screen or to an
                output file, if specified.}
                {crossreferences}

	     \argumentdefinition{data}{}
                {Outputs a summary of the input data.
                More specifically, \poy will report the number of
                terminals to be analyzed, a list of included terminals with
                numerical identification numbers, list
                of synonyms (if specified), a list of excluded terminals, a
                number of included characters in each character-type category
                (\emph{i.e.} additive, non-additive, Sankoff, and molecular) with the corresponding
                cost regimes, a list of excluded
                characters, and a list of input files.} 
                {data}

            \argumentdefinition{seq\_stats}{\obligatory{identifiers}}
                {Outputs a summary of the sequences specified in the argument
                value, for all taxa. The summary includes the maximum, minimum,
                and average length and distance for all terminals.}
                {seqstats}

            \argumentdefinition{terminals}{}
                {Reports a list and number of terminals included and excluded
                per input file. Use the command~\ccross{select} for including and excluding
                terminals .}
                {}

            \argumentdefinition{treestats}{}
                {Reports the number of trees in memory per cost.}
                {treestats}

            \argumentdefinition{treescosts}{}
                {Reports the cost of each tree separated by colons. The output
                contains no formatting for easy processing by any scripting
                language.}
                {}

		\end{argumentgroup}

		\begin{argumentgroup}{Trees}
            {This set of arguments outputs tree representations
            in parenthetical, ascii (simple text), or postscript formats.
            The arguments specify the types of tree outputs. They include
            actual trees resulting from current searches, or imported from
            files, their consensus trees, or trees displaying support values.
            
            To select the root terminal in the tree representation, the command~\ccross{set} is used.
            
            Most analyses produce more than a single tree and it is
            often desirable to report only some of them. To
            report particular trees (for instance all optimal trees,
            randomly-selected trees, or all unique trees, \emph{etc.}), first the
            command~\ccross{select} must be applied to specify (select)
             the desired trees from all those stored in memory.} 

             \argumentdefinition{all\_roots}{}
                {In a tree with $n$ vertices (and therefore $n - 1$ edges),
                calculates the cost of the $n - 1$ rooted trees as implied by a
                root located in the subdivision vertex at each edge in the unrooted
                tree in memory.}
                {allroots}

            \argumentdefinition{asciitrees}{\optional{collapse\optional{\poybool}}}
                {Draws ascii character representations of trees stored in memory. The
                argument \poyargument{collapse} collapses the zero length branches if
                the boolean value is \texttt{true} (the default); if the boolean value is
                \texttt{false}, the zero length branches are not collapsed.}
				{}

			\argumentdefinition{clades}{}
	     {Output a set of Hennig86 files. Each file, named \texttt{file.hen},
                where ``file'' is whatever string you pass to this function
                contains information on each clade for one of the trees
                currently stored. This is similar to the utility jack2hen 
                of \texttt{POY3}.}
				{}

			\argumentdefinition{consensus}{\optional{\poyint}}
                {Reports the consensus of trees in memory in parenthetical notation.
                If no integer value is
                specified, a strict consensus is calculated~\cite{rohlf1982};
                if integer value is specified,
                a majority rule consensus is computed, collapsing nodes with
                occurrence frequencies less than the specified integer~\cite{margush1981}.
                If a value less
                than \texttt{51} is specified, \poy reports an error.} 
                {}

            \argumentdefinition{graphconsensus}{\optional{\poyint}}
                {Same as \poyargument{consensus} except for consensus trees are
                reported in graphical format, either in the ascii format on
                screen or in the postscript format if redirected to a file.}
                {}

            \argumentdefinition{graphsupports}{\optional{argument}}
                {This command outputs a tree with support values that have
                been previously calculated using the
                \nccross{calculate\_support}{calculatesupport} either on screen
                in ascii format, or, if specified, to a file in postscript
                format. The argument values are the same as for 
                \poyargument{supports} (\emph{i.e.} \poyargument{bremer},
                \poyargument{jackknife}, and \poyargument{bootstrap}).} 
                {graphsupports}

            \argumentdefinition{graphtrees}{\optional{collapse\optional{\poybool}}}
                {If \poy has been compiled with graphics support, it 
                will display a window in which you can
                browse graphical representations of all the trees in memory.
                When working in this window, using ``j'' and ``k'' keys displays the
                previous or next tree respectively. If no graphical support available, the output 
                is similar to that generated by the \poyargument{asciitrees} argument. Pressing ``q'' key 
                returns to the \emph{Interactive Console} window. The argument
                \poyargument{collapse} will collapse the zero length branches if
                true, otherwise not (default is \texttt{true}.)} 
	     {}

            \argumentdefinition{supports}{\optional{argument}}
                {Outputs a parenthetical representation of a tree with the
                support values has previously been calculated using the
                command~\nccross{calculate\_support}{calculatesupport},
                either to the screen or to a file (if specified). If no argument
                is given, all calculated support values are printed. The arguments
                \poyargument{bremer}, \poyargument{jackknife}, and
                \poyargument{bootstrap} specify which type of support tree to
                report. 
                
                \poyargument{bremer} accepts an optional string argument
                (as in \poycommand{report(supports:\\bremer:"file.txt")}, 
                which specifies a file containing a list of trees and costs (as
                those generated by \ccross{visited}), which should be used with
                their annotated cost to assign the bremer support values. 
                If no input file is given, or if \poyargument{bootstrap} or
                \poyargument{jackknife} are requested,
                then the necessary information must have been calculated 
                using \nccross{calculate\_support}{calculatesupport}.
                
                \poyargument{jackknife} and \poyargument{bootstrap} accept an
                optional argument with two possible values:
                \poyargument{individual} or \poyargument{consensus}.
                \poyargument{individual}
                reports the support value for each tree held in memory: if there
                are a hundred trees stored in memory, for each one, the support
                values for each tree are reported. \poyargument{consensus}
                generates a ``consensus'' tree, with the clades that have
                support higher than 50 percent. The default behavior, when no
                \poyargument{individual} or \poyargument{consensus} value is
                provided, is \poyargument{individual}.}
                {supports}

			\argumentdefinition{trees}{\obligatory{(argument list)}}
                {Outputs the trees in memory in parenthetical notation. The argument
                \poyargument{trees} receives an optional list of values
                specifying the format of the tree that has to be generated. The
                valid optional arguments are:  
                
                \begin{description}
                    \argumentdefinition{total}{}
                        {Includes the total cost of a tree in square brackets
                        after each tree.}
                        {total}

                    \argumentdefinition{\_cost}{}
                        {Include the cost in square brackets for every subtree in the tree. (These 
                        are \emph{not} branch lengths.}
                        {cost}

                    \argumentdefinition{hennig}{}
                        {Prepends the \poycommand{tread} command to the list of
                        trees and separates them with a star; this format is
                        suitable for Hennig86, NONA, and TNT files.}
                        {}
                        
                    \argumentdefinition{newick}{}
                        {Outputs the trees in the Newick format, with the
                        terminals separated with commas, and trees separated
                        with semicolons.}
                        {}
	    \begin{statement}
	     The \poyargument{hennig} and \poyargument{newick} arguments are 
	     mutually exclusive.
	     \end{statement}
	     
                    \argumentdefinition{margin}{\obligatory{INTEGER}} 
                        {Sets the margin width of the generated trees.}
                        {}

                    \argumentdefinition{nomargin}{}
                        {Outputs the trees in a single line. This is useful for
                        some programs (such as TreeView) that cannot read
                        trees broken in several
                        lines.}
                        {}

                    \argumentdefinition{collapse}{\optional{\poybool}}
                        {If \poycommand{true}, zero length branches are collapsed (the
                        default), but if \poycommand{false}--no branches are
                        collapsed.}
                        {}
                    \end{description}}
                {treesreport}

		\end{argumentgroup}

		\begin{argumentgroup}{Implied alignments}
            {This set of arguments outputs implied alignments~\cite{wheeler2003}.} 

            \argumentdefinition{fasta}{\obligatory{identifiers}}
            {The same as \nccross{implied\_alignments}{impliedalignment} but no additional headers
                are added, producing a valid FASTA file. Intended for easy
                automation, by producing a file that other programs can read
                immediately.}
                {fasta}

            \argumentdefinition{implied\_alignments}{\optional{identifiers}}
                {Outputs the implied alignments of the specified
                set of characters in FASTA format. The optional value of the
                argument specifies the characters included
                in the output, using the same identifiers described for the
                character specification in the entry for the command~\ccross{select}. If no
                characters are specified, then the implied alignment of all the
                sequence characters is generated. The output is reported on
                screen unless a name of an output file (in parentheses) is
                specified, preceding the command name and separated from it by a
                comma. This argument is synonymous with the argument
                \poyargument{ia}.}
                {impliedalignment}

            \argumentdefinition{ia}{\optional{identifiers}}
                {Synonym of \poyargument{implied\_alignments}.}
                {}

        \end{argumentgroup}

        \begin{argumentgroup}{Exporting static homology data}
            {The following commands export the static homology characters
            currently in memory.}

            \argumentdefinition{phastwinclad}{}
                {Produces a file in the Hennig86 format that contains the
                additive and nonadditive characters currently in memory.  In
                order to export an implied alignment as a Hennig86 file, the
                characters must first be transformed into static characters
                using the \poycommand{transform} command: 
                \begin{flushleft}
                    \poycommand{transform ((all, static\_approx))} \\
                    \poycommand{report ("report.ss", phastwinclad)}
                \end{flushleft}}
                {}
	\begin{statement}
	      To generate a file that contains implied
                alignments only for a subset of fragments, an identifier must be
                included in the argument list of \poycommand{transform}. For
                example, 
                \begin{flushleft}
                \poycommand{transform ((names:("fragment\_1", "fragment\_2"),
                static\_approx))} \\
                \poycommand{report ("myfile.ss", phastwinclad)}
                \end{flushleft}
                will produce Hennig86 files only for
                \texttt{fragment\_1} and \texttt {fragment\_2}. The resulting file can be imported into other programs,
                such as WinClada.  This is equivalent to the
                \poycommand{phastwincladfile} command in \texttt{POY3}.
	\end{statement}
	
		\end{argumentgroup}

		\begin{argumentgroup}{diagnosis}
			{This set of arguments will output the diagnosis.} 

			\argumentdefinition{diagnosis}{}
                {Outputs the diagnosis of each tree on screen or redirects it to a file, if
                specified. If the extension \emph{.xml} is appended to the name of the
                output file, the diagnosis is reported in XML format, rather than in
                simple text format.} 
                {}

		\end{argumentgroup} 

		\begin{argumentgroup}{Other arguments}
			{} 

	     \argumentdefinition{ci}{}
	      {Calculates the ensemble consistency index (CI ~\cite{farris1989,
	      klugeandfarris1969}) for additive, nonadditive, and Sankoff
	      characters. Dynamic homology characters are ignored in calculating
	      the CI, therefore, the dynamic homology characters must be converted
	      to static homology characters using the argument \poyargument{static\_approx} 
	      of the command \nccross{transform}{transformcommand}.}
	      {ci}
	      
	      \argumentdefinition{memory}{}
                {Reports on screen, the statistics of
                the garbage collector. For a precise description of each memory parameter, see
                the Objective Caml documentation.}
                {}
	     
	      \argumentdefinition{ri}{}
	      {Calculates the ensemble retention index (RI; ~\cite{farris1989}) for additive,
	      nonadditive, and Sankoff characters. Dynamic homology characters are ignored in calculating
	      the RI, therefore, the dynamic homology characters must be converted
	      to static homology characters using the argument \poyargument{static\_approx} 
	      of the command \nccross{transform}{transformcommand}.}
	      {ri}
	      
	     \argumentdefinition{script\_analysis}{\obligatory{\poystring}}
                {Reports the order in which commands listed of the imported
                script (specified by the string argument) are going to be executed.
                Unlike executing individual commands interactively, when commands are submitted in a 
                script, \poy determines the logical interdependency of operations
                and processes the commands in the order that yields the same
                results as if they were executed sequentially. This substantially
                optimizes parallelization and reduces memory consumption.
                
                The colored output in the \emph{POY Output} window of the ncurses
                interface facilitates reading the output of \poyargument{script\_analysis}:
                red lines mark hard constraints that allow neither
                parallelization nor memory optimizations, blue lines mark 
                constraints that allow the program to pipeline commands in
                parallel, and green lines mark fully parallelizable commands. When \poy
                is compiled with parallel off, all the operations are
                sequential, therefore, each potentially parallel operation is
                done as sequential repetitions of the subscripts described in
                the output of the command, reducing memory consumption.}
                {scriptanalysis}
                
                \argumentdefinition{timer}{\obligatory{STRING}}
                {Reports the value and the user time (in seconds) elapsed between
                two consecutive timer reports. The string value provides a label
                (typically a textual description) that precedes the time report
                in the output produced.
                The first timer report displays the time elapsed since the beginning of the
                \poy session. This command is useful for monitoring the execution time
                of specific tasks.}
                {}
                
                \argumentdefinition{xslt}{\obligatory{(STRING, STRING)}}
	     {Applies a user-defined xslt stylesheet to the XML output. The first string is
	     the filename of the output, the second string is the name of the stylesheet
	     requested to generate it.}
	     {}
	     
	     \begin{statement}
	     Extensible Stylesheet Language Transformations (XSLT) are used
	     for the transformation of XML output into other formats. Because the XML output contains all the information regarding data and trees, using XSLT stylesheets greatly expand the capabilities of \poy to use and display results.
	     Examples of potential applications includes graphical display of trees with proportional branch lengths,
	     integration of tree topologies with geographical coordinate data for spatial mapping, and
	     generating input files for other programs.
	     \end{statement}
	     
	     \end{argumentgroup}
	\end{arguments}

	\poydefaults{data, diagnosis, trees}
        {By default, \poy will print on screen the following items: the tree(s)
        in parenthetical notation with corresponding tree cost(s), diagnosis of
        each tree, and a graphical representation on the tree(s) in ascii
        format. This output can be re-directed to a file by specifying a file
        name enclosed in quotation marks, for example:
        \poycommand{report("filename")}.}

	\begin{poyexamples} 

		\poyexample{report("my\_results")}
		{This commands outputs the data, trees, and diagnosis (the default) to the
		file \texttt{my\_results}. Because no path is specified, the
		file is located in the current working directory.}
		
		\poyexample{report(data)} 
            {This command displays on screen a list of included and excluded terminals, their
            names and codes, gene fragments, file names, and other relevant data.}
            
		\poyexample{report(treestats)}
            {This example displays on screen the costs of all trees in memory and the
            number of trees for each cost.}

		\poyexample{report("filename", treestats)} 
            {This commands outputs the costs of all trees in memory and the
            number of trees for each cost to a file \texttt{filename}.}

		\poyexample{report(cross\_references:names("file1", "file3"))}
		{This command produces a table showing presence
		and absence of data corresponding to all terminals contained
		in files \texttt{file1} and \texttt{file3}. Because an output
		file is not specified, the table is displayed on screen.}
		
		\poyexample{report("taxa", terminals)}
		{This command generates a file \texttt{taxa} that contains the
		lists and numbers of excluded and included terminals for each of the previously
		imported datafiles.}
		
		\poyexample{report(trees)}
            {This command displays on screen the trees in memory in parenthetical
            notation with zero-length branches collapsed and terminals
            separated by spaces.}

        	\poyexample{report(trees:(total))}
            {This command produces the same output as the example above
            but also includes the total tree cost in square brackets
            following each tree.}

	\poyexample{report("filename", trees:(collapse:false, newick))} 
            {This command produces a file \texttt{filename} that contains
            all trees in Newick format with zero-length branches \emph{not}
            collapsed.}
		
		\poyexample{report("filename", graphtrees)} 
            {This command saves all trees in memory in
            postcript format to the file \texttt{filename.ps}.}

		\poyexample{report(asciitrees, "file1", trees:(newick, nomargin), \\ "file2", graphtrees)}
		{This command displays a tree in ascii format on screen and outputs
		to \texttt{file1} trees with zero-length branches collapsed in Newick format
		in a single line (using no margin, the format compatible with \emph{TreeView}). It
		also writes to \texttt{file2} the graphical representation of these trees in
		postscript format.}

        \poyexample{report("hennig.ss", phastwinclad, trees:(hennig, total))}
            {This command outputs all the static homology characters, including their cost
            regime, in the file \poycommand{hennig.ss}; then append to the same
            file the trees currently in memory using the Hennig format, 
            including the total cost of each tree in square brackets. The
            generated \poycommand{hennig.ss} is compatible with NONA, TNT, and
            Hennig86.
            \index{general}{export!hennig}\index{general}{export!nona} \index{general}{export!tnt}}
            
         \poyexample{report("my\_results", data, diagnosis, consensus, \\consensus:75,
         "consensus", graphconsensus)}
         {This command reports the requested types of outputs (\emph{i.e.}
        reports on the data, diagnosis, and strict consensus and 75 percent
         majority-rule consensus trees in parenthetical notation) to the file
         \texttt{my\_results}. In also outputs a strict consensus tree to the file
         \texttt{consensus}.}
         
         \poyexample{report(graphsupports, "bremertree", graphsupports:bremer)}
         {This command reports on screen all previously calculated support values
         placed at the nodes of ascii trees and outputs to file the \texttt{bremertree}
         only the tree(s) with bremer support values.}
         
         \poyexample{report(implied\_alignments)}
         {This command reports the implied alignments for all dynamic homology
         characters on screen.}
         
          \poyexample{report("align\_file", ia:names:("SSU", "LSU"))}
          {This command generates the file \texttt{align\_file} that contains
          the implied alignments only for characters contained in datafiles
          \texttt{SSU} and \texttt{LSU}.}
          
          \poyexample{report("scipt1\_analysis", script\_analysis:"/users/datafiles/\\script1.poy")}
          {This command produces the file \texttt{scipt1\_analysis} that lists the commands from
          the input script file \texttt{script1.poy} in the order that optimizes parallelization and
          memory consumption. In this example the complete path (\texttt{/users/datafiles/script1.poy})
          is provided, which is not necessary if the directory containing the file \texttt{script1.poy}
          has already been assigned using the command \ccross{cd} in the same \poy session.}
          
          \poyexample{report("swapping", timer:"swap end")}
          {This command generates the file \texttt{swapping} that contains
          the string \texttt{swap end} followed by the number of seconds (in
          decimals) elapsed since the execution of the previous \poyargument{timer}
          argument.}
          
          \poyexample{report("new\_tree\_diagnosis.xml", diagnosis)}
         {This command reports the diagnosis to the \texttt{new\_tree\_diagnosis.xml}
         file in XML format.}

	\end{poyexamples}

	\begin{poyalso}
        \ncross{calculate\_support}{calculatesupport}
	 \end{poyalso}

\end{command}

\begin{command}{run}{}

	\syntax{\obligatory{(\poystring)}}

	\begin{poydescription}
        Runs \poy script file or files. The filenames must be included in
        quotes and, if multiple files are included, they must be separated by commas.
        The script-containing files are executed in the order in which they are listed
        in the string argument.
        Executing scripts using \poycommand{run} is useful in cases when
        operations take take long time or many scripts need to be executed automatically,
        for example, when conducting sensitivity analysis\cite{wheeler1995}.
        There are no default settings of \poycommand{run}.
        \end{poydescription}
        
        \begin{statement}
  	Note that if any of the scripts contain the commands \poycommand{exit()} or
	\poycommand{quit()}, \poy will quit after executing that file. Therefore, if
	multiple files are submitted, only the last one must contain \poycommand{exit()}
	or \poycommand{quit()}.
	\end{statement}
	
	\begin{poyexamples}
        \poyexample{run("script1", "script2")}
            {This command executes \poy command scripts contained in the files \texttt{script1}
            and \texttt{script2} in the same order as they are listed in the list of arguments of
            \poycommand{run}.}
          \end{poyexamples}
          
          \begin{poyalso}
        		\cross{exit}
		\cross{quit}
	\end{poyalso}

\end{command}

\begin{command}{save}{}

	\syntax{\obligatory{(\poystring \optional{, \poystring})}}

	\begin{poydescription}
            Saves the current \poy state of the program to a file (\poy file). The first, obligatory string argument
            specified the name of the \poy file. The second, optional string argument specifies a
            string included in the \poy file, that can be retrieved using the command \ccross{inspect}. 

            \poy files are not intended for permanent storage: they are recommended
            for temporary storing of a \poy session by a user, checkpointing the
            current state of a search to avoid loss work in case the computer or the
            program itself fails, or to report bugs. \poy will also automatically
            generate the file in many cases when a terminating error occurs (an
            important exception is out-of-memory errors). The format of these files might differ among different versions of \poy; consequently, these files might not be interchangeable between all the versions of the program.
	\end{poydescription}

    \begin{poyexamples}
        \poyexample{save("alldata.poy")}
            {This command stores all the memory contents of the program in the file
            \texttt{alldata.poy} located in the current working directory, as
            printed by \poycommand{pwd()}.}

        \poyexample{save("alldata.poy", "My total evidence \\data")}
            {This command performs the same operation as described in the example above,
            but, in addition, it includes the string \texttt{My total
            evidence data} with the file \texttt{alldata.poy},
            which can later be retrieved using the command \ccross{inspect}.}

        \poyexample{save("/Users/andres/test/alldata.poy", "My total evidence \\data")}
            {This command performs the same operation as the command described above
            with the important difference that the file \texttt{alldata.poy} generated in the
            directory \texttt{/Users/andres/test/} instead of the current working directory.}
    \end{poyexamples}

    \begin{poyalso}
        \cross{inspect}
        \cross{load}
    \end{poyalso}

\end{command}

\begin{command}{search}{}

	\syntax{\obligatory{(\optional{argument list})}}

	\begin{poydescription}
            The command \poycommand{search} implements a default strategy for a
             search. The command integrates the \poycommand{build},
            \poycommand{transform}, and \poycommand{swap} commands for an
            efficient initial search. Tree building and swapping are executed
            under default settings; the \poyargument{transform} provides an
            option of making sequential transformations of characters that
            substantially speeds up the search, however, at the expense of
            accuracy in calculating tree cost. The arguments are optional and
            their order is arbitrary. Even though the entire sequence of the commands
            can also be specified by setting \poycommand{build},
            \poycommand{transform}, and \poycommand{swap} individually to
            corresponding values, the advantage of using \poycommand{search}
            command is that these steps are already predefined.
	\end{poydescription}

	\begin{arguments}

		\argumentdefinition{build}{\optional{\poybool}}
            {Specifies either to build new trees (boolean value
            \texttt{true}) or to use the trees stored in memory (that is do not
            build new trees; boolean value \texttt{false}). The default for
            \poyargument{build} is \texttt{false}. Therefore, executing
            \poycommand{search} under default settings produces no result if
            there are no trees in memory.}
            {buildsearch}

		\argumentdefinition{transform}{\optional{\poybool}}
            {Specifies to either transform characters as part of the search by
            sequential execution of commands
            \poyargument{auto\_sequence\_partition} and
            \poyargument{auto\_static\_approx} (both arguments of the command
            \nccross{transform}{transformcommand}) (boolean
            value \texttt{true}) or not (boolean value \texttt{false}). 
            This combination of character transformations substantially accelerates the
            search but at the expense of accuracy in calculating the exact tree cost. The
            default is \texttt{false} (characters are not transformed).}
            {transformarg2}

	\end{arguments}

    \poydefaults{build:true, transform:false}{Under default settings, 10 trees
    are built, then subjected
     to branch swapping using alternating SPR and TBR, keeping one tree per
     swap (the default of \poycommand{swap}), and without transforming characters.}
        
    	\begin{poyexamples}
        	
	\poyexample{search(build:true, transform:false)}
            {This command builds 10 Wagner trees by random addition sequence (the default
            of \poycommand{build}), performs alternating SPR and TBR branch swapping,
            and keeps one tree per swap (the default of \poycommand{swap}). No characters are 
            transformed during the search. Because the 
            default of the argument \poycommand{transform}
            is \texttt{false}, it can be omitted from the list of argument. Therefore this command
            is equivalent to \poycommand{search(build:true)}. Note that if there are
            trees in memory, the new trees generated by \poycommand{search} replace
            them.}
	
	\poyexample{search(transform:true)}
	{This command performs branch swapping on the existing trees in memory
	under default parameters of \poycommand{swap}, as shown in the examples
	above. The searches are performed on characters transformed using
	sequential application of \poycommand{transform} arguments
	\poyargument{auto\_sequence\_partition} and \poyargument{auto\_static\_approx}
	to speed up the swapping procedure.}
	
            \end{poyexamples}

	\begin{poyalso}
  		\ncross{build}{buildcommand}
		\ncross{swap}{swapcommand}
		\ncross{transform}{transformcommand}
	\end{poyalso}

\end{command}

\begin{command}{select}{}

	\syntax{\obligatory{(\optional{argument})}}

	\begin{poydescription} 
            Specifies a subset of terminals, characters, or trees from those
            currently loaded in memory to use in subsequent analysis.
	\end{poydescription}
	

	\begin{arguments}
		
		\begin{argumentgroup}{Select terminals and characters}
            {Specifies terminals and characters to use in subsequent
            analysis. 
            The arguments in this group specify whether terminals or characters
            are being selected.
            \emph{Identifiers} are used to specify which characters or
            terminals are being selected (see
            the \emph{Character and terminal identifiers} argument group below
            for the description of methods for selecting specific terminals or characters).}
 
            \argumentdefinition{terminals}{}
                {Specifies that the subsequently listed identifiers
                refer to \emph{terminals} to be selected. By default, \poy
            assumes that the specification refers to terminals. For example, to
            analyze only those terminals listed in the file \texttt{opiliones} using
            the character data currently loaded in memory, use the command 
            \poycommand{select(files:("opiliones"))}. This  command is
            equivalent to \poycommand{select(terminals,files:("opiliones"))}.
            
            When the command is executed, the list of selected terminals is
            printed on screen.  \poycommand{terminals} is only valid as an
            argument of commands \poycommand{select} and \ccross{rename}.} 
                {}
	
	\begin{statement}
  	Note that once specific terminals and/or  characters are selected, the excluded
	data cannot be restored. To be able to reconstitute the original data set or to
	experiment with various character and terminal selections within a given \poy
	session, use the commands ~\ccross{store}{} and \ccross{use}.
	\end{statement}
	
	\argumentdefinition{characters}{}
                {Specifies that the subsequently listed identifiers
                refer to \emph{characters} to be selected.}
                {}

            \argumentdefinition{STRING}{}
                {Selects terminals listed in the file specified by the string argument.}
                {}

		\end{argumentgroup}
		
        \begin{argumentgroup}{Character and terminal identifiers}\label{identifiers}
        {\emph{Identifiers} specify which characters or terminals are analyzed.
        In addition to the command \poycommand{select}, identifiers are used as
        arguments for other commands that require selection of specific terminals or
        characters, such as commands \ccross{report} and
        \nccross{transform}{transformcommand}.}

            \argumentdefinition{all}{}
                {Specifies all characters or terminals.}
                {allidentifier}

            \argumentdefinition{names}{\obligatory{(\poystring list)}}
                {Specifies the names of the characters or terminals.}
                {}

            \argumentdefinition{codes}{\obligatory{(\poyint list)}}
                {Specifies the codes of characters or terminals. The codes are unique
                numbers that are generated by \poy when data files are first imported.
                The codes can be reported using the argument \ccross{data}
                of the command \poycommand{report}. The codes are generated anew
                when a given data file is reloaded; therefore, they can effectively be used
                only within a current \poy session.}
                {}

            \argumentdefinition{files}{\obligatory{(\poystring list)}}
                {Specifies the filename list containing lists of terminals or
                characters.}
                {}

            \argumentdefinition{missing}{\obligatory{\poyint}}
                {Selects terminals or characters to be included in the analysis
                based on the proportion of missing data. The
                integer value sets the minimum percentage of missing
                data. Terminals or characters that have \emph{more} missing data
                than defined by the value are included in the analysis.}
                {}
               
                \argumentdefinition{not\_missing}{\obligatory{\poyint}}
                {Selects terminals or characters to be included from the analysis
                based on the proportion of missing data. The
                integer value sets the minimum percentage of missing
                data. Terminals or characters that have  \emph{less} missing data
                than defined by the value are included in the analysis.
                In effect, this selects a complement of data to the argument \poyargument{missing}.}
                {notmissing}

                \begin{statement}
                For dynamic homology characters, the missing data refers to
                sequence fragments, whereas for static characters it refers to
                individual nucleotide positions. Therefore, when excluding
                terminals with missing data, the resulting set of selected
                terminals depends on the character type and might, or
                might not, be identical. For example, if a data file (containing
                sequences corresponding to a single fragment) includes
                a very short sequence, this sequence is not treated as
                missing data regardless of its length. This is because in the
                context of dynamic homology a fragment, rather than an
                individual nucleotide position, constitutes a character.
                On the other hand, if the same data are treated as static characters,
                the taxon represented by a very short sequence
                might be excluded if the length of the sequence exceeds the
                threshold defined by the value of \poyargument{missing}.
                \end{statement}

            \argumentdefinition{static}{}
                {Specifies the static homology characters.}
                {}

            \argumentdefinition{dynamic}{}
                {Specifies the dynamic homology characters.}
                {}

            \argumentdefinition{not names}{\obligatory{(\poystring list)}}
                {Specifies the characters or terminals other than those the
                names of which are listed in the string list.}
                {notnames}

            \argumentdefinition{not codes}{\obligatory{(\poystring list)}}
                {Specifies the characters or terminals other than those the
                codes of which are listed in the string list.}
                {notcodes}

          
        \end{argumentgroup}

		\begin{argumentgroup}{Select trees}
			{Selects trees from the pool of trees currently in memory.}

			\argumentdefinition{optimal}{}
				{Selects all trees of minimum cost.} 
                			{}
			
            \argumentdefinition{best}{\obligatory{\poyint}}
				{Selects the number of best trees specified by the integer value.
				Best trees are not equivalent to optimal trees because best trees
				can include suboptimal trees in case the value of
				\poyargument{best} exceeds the number of optimal (minimal-cost)
				trees. If the number of optimal trees exceeds the value of
				\poyargument{best}, only a subset of optimal trees (equal to the
				value of \poyargument{best} is selected in unspecified order).} 
                			{}
	
	\begin{statement}
               There is no special command in \poy to clear trees from memory. However,
               selecting zero best trees using the command \poycommand{select(best:0)}
               effectively removes all trees currently stored in memory.
          \end{statement}
            
            \argumentdefinition{within}{\obligatory{\poyfloat}}
                {Selects all optimal and suboptimal trees the costs of which do not exceed
                the current optimal cost by the float value. For example, if the current
                optimal cost is 507 and the float value of \poyargument{within} is
                \texttt{3.0}, all trees with costs 507--510 are selected.} 
                {}

            \argumentdefinition{random}{\obligatory{\poyint}}
	{Randomly selects the number of trees specified by the integer
	value irrespective of cost.} 
                {}

	\argumentdefinition{unique}{}
	{Selects only topologically unique trees (after collapsing zero-length
	branches) irrespective of their cost.} 
                {}

		\end{argumentgroup}
	
	\end{arguments}
	 	 	 	 	  
    \poydefaults{unique, optimal}
        {By default \poy selects all unique trees of optimal (best) cost. The rest of
        the trees are removed from memory.}

	\begin{poyexamples}
        
        
        \poyexample{select(terminals,names:("t1", "t2", "t3", "t4", "t5"), \\ 
            characters, names:("chel.aln:0"))}
            {This command selects only terminals \texttt{t1},  \texttt{t2},
             \texttt{t3},  \texttt{t4}, and  \texttt{t5} and use data only from the
              fragment  \texttt{0} contained in the file \texttt{chel.aln}.}
	
	\poyexample{select(terminals, missing:50)}
	{This command excludes from subsequent analysis all the terminals that
	have more than 50 percent of characters missing. The list of included and excluded
	terminals is automatically reported on screen.}
	
	\poyexample{select(optimal)}
            {Selects all optimal (best cost) trees and discards suboptimal trees from
            memory. The pool of optimal trees might contain duplicate trees (that can
            be removed using \poyargument{unique}).}
            
	\poyexample{select(unique, within:2.0)}
	{This command selects all topologically unique optimal and suboptimal trees
	the cost of which does not exceed that of the best current cost by more than
	2. For example, if the best current cost is 49, all unique trees that fall within
	the cost range 49--51 are selected.}
	
	\end{poyexamples}

	\begin{poyalso}
        \cross{characters}
        \ncross{transform}{transformcommand}
	\end{poyalso}

\end{command}

\begin{command}{set}{}

	\syntax{\obligatory{(\optional{argument list})}}

	\begin{poydescription}
            Changes the settings of \poy. This command performs diverse auxiliary 
            functions from setting the seed of the random number generator to
            selecting a terminal for rooting output trees.
            
            There is no default setting for \poycommand{set} and the order of its
            arguments is arbitrary.
            
	\end{poydescription}

	\begin{arguments}

        \begin{argumentgroup}{Application settings}
            {Some generic application settings. Have no effect in the analyses
            themselves.}

            \argumentdefinition{history}{\obligatory{\poyint}}
                {Sets the size of the \poy output history displayed in the
                \emph{POY Output} window to the number of lines specified by the
                integer value. The size of the history must be greater than
                zero. This command has effect only in the ncurses interface. The
                default size of the output history is 1000 lines.}
                 {}

            \argumentdefinition{log}{\obligatory{\poystring}}
                {Directs a copy of a partial output to the file specified by the
                string  argument. The output includes the  information in the
                \emph{POY Output}, \emph{Interactive Console}, and \emph{State
                of Stored Search} windows of ncurses interface.  Timers and
                current state of the search are not included in the log. If the
                log
                file already exists, \poy will append the text to it; if the log
                file does not exist, then \poy creates a new file. If the user
                would like to delete the contents of a pre-existing file, then
                the argument \poyargument{log:new:"logfile"} creates a new
                initially empty file named \texttt{logfile}.}
                {log}

            \argumentdefinition{nolog}{}
                {Stops outputting the log to any previously selected
                file. See the description of the argument \poyargument{log}
                above.}
                {}

            \argumentdefinition{root}{\obligatory{\poylident}}
                {Specifies the terminal to root output trees.
                 The terminal can either be indicated as a taxon name (a
                \poystring, which must appear in quotes, such as
                \texttt{"Genus\_species"}) or the code, that is automatically
                assigned to the taxon by \poy at the beginning of each \poy
                session (for example, \poycommand{set(root:45)}. The codes can
                be obtained using the command \poycommand{report(data)}).  The
                terminal codes, however, are unique only within a current
                session.}
                {root}

        \end{argumentgroup}

        \begin{argumentgroup}{Cost calculation}
            {These arguments set the tree cost estimation routines and
             are applied to all character types. The arguments
            are mutually exclusive: only the argument of 
            \poycommand{set} specified last is used.}

            \argumentdefinition{normal\_do}{}
                {Applies a standard Direct Optimization algorithm for the tree
                cost estimation. This is the default and fastest technique.}
                {normaldo}

            \argumentdefinition{exhaustive\_do}{}
                {Applies a standard Direct Optimization algorithm for the tree
                cost estimation. The difference with \poyargument{normal\_do} is
                that the calculation of the tree costs during a search is much
                more intense, always looking for the best possible alignment 
                for every single topology (instead of a lazy and greedy strategy
                used by the \poyargument{normal\_do}).}
                {exhaustivedo}

            \argumentdefinition{iterative}{}
                {Applies the Iterative Pass optimization for the tree cost calculations. This improves
                the tree cost estimation but at the expense of a tremendous execution time.  The iterative argument 
                can be applied to all character types including chromosome, genome, and custom\_alphabet characters.  
                When implementing the iterative pass option for these complex character types the parameter 
                \poyargument{max\_3d\_len} can be used within the command \poycommand{transform}
                ((all, \poyargument{dynamic\_pam}:(\poyargument{max\_3d\_len:} {\obligatory{\poyint}}))) 
                to reduce execution time.  Another hueristic strategy is to implement  \poyargument{iterative} at the 
                very end of an analysis to polish the final set of trees and perform a final search.}
                {}
                
	\begin{statement}
  	  Due to the complexity of heuristics of the iterative pass optimization, there is no
	  guarantee that the tree cost recovered from the search would be exactly the same
	  as produced by the diagnosis of the same tree. However, the cost of the tree found during
	  the search can be verified by outputting the medians from the diagnosis
      (see the description of the argument \nccross{diagnosis}{diagnosis})
      of the command \poycommand{report} and determining edge costs by hand. The cost of the tree found during the search might
	  differ from that obtained by the rediagnosing the same tree (see \nccross{rediagnose}{rediagnose}), but will
	  recover the same tree cost in subsequent rediagnoses.

  \end{statement}
	
        \end{argumentgroup}

        \begin{argumentgroup}{Randomized routines}
            {}

            \argumentdefinition{seed}{\obligatory{\poyint}}
                {Sets the seed for the random number generator using the integer 
                value. If unspecified, \poy uses the system's time as seed.}
                {}
	\begin{statement}
  	 It is critical to set a seed value to insure reproducibility of the results
	 of the analyses that require randomization routines (such as tree
	 building).
	\end{statement}
	
            \end{argumentgroup}
	\end{arguments}

    \poydefaults{history:1000, normal\_do}{Under default settings the size of
    the history buffer is limited to 1,000 lines, the Direct Optimization is used for
    tree cost calculation, and the current time is used to specify the seed.}

    \begin{poyexamples}
    	\poyexample{set(history:1500, seed:45, log:"mylog.txt")}
            {This command increases the size of the history in the ncurses
            interface to 1,500 lines, sets the random number generator to 45,
            and initiates a log file \texttt{mylog.txt}, located in the current
            working directory.}
            
            \poyexample{set(root:"Mytilus\_edulis")}
            {This commands selects terminal \texttt{Mytilus\_edulis} as a root
            for the output trees.}
            
     \end{poyexamples}
     
	\begin{poyalso}
		\cross{report}
	\end{poyalso}

\end{command}

\begin{command}{store}{}

	\syntax{\obligatory{(\obligatory{\poystring})}} 

	\begin{poydescription}
            Stores current state of \poy session in memory. The stored information
           includes character data, trees, selections, \emph{everything}. Specifying the
           name of the stored state of the search (using the string argument)  does
           \emph{not}, however, generate a file under this name that can be examined;
           the name is used only to recover the stored state using the command \poycommand{use}.

            In combination with \poycommand{use}, the command \poycommand{store}
            is extremely useful when exploring alternative  cost regimes and terminal sets
            within a single \poy session.
	\end{poydescription}
	
	\begin{arguments}
		\argumentdefinition{STRING}{}
                {Specifies the name of the stored search state of the current \poy session.}
                {}
	\end{arguments}

    \begin{poyexamples}
        \poyexample{store("initial\_tcm") \\ transform(tcm:(1,1)) \\ use("initial\_tcm")}
            {The first command, \poycommand{store}, stores the current
            characters and trees under the
            name \texttt{initial\_tcm}. The second command,
            \poycommand{transform}, changes the cost regime of molecular characters,
            effectively changing the data being analyzed. However, the third
            command, \poycommand{use}, recovers the initial state stored under the
            name \texttt{initial\_tcm}.}
    \end{poyexamples}

    \begin{poyalso}
        \cross{use} 
        \cross{transform}
    \end{poyalso}

\end{command}

\begin{command}{swap}{swapcommand}

	\syntax{\obligatory{(\optional{argument list})}}

	\begin{poydescription} 
            \poycommand{swap} is the basic local search function in \poy. This
            command implements a family of algorithms collectively known as
            branch swapping in systematics and as hill climbing in combinatorial
            optimization. They proceed by clipping parts of a given tree and
            attaching them in different positions of the same tree.  It can be
            used to perform a local search on a set of trees loaded in memory.

            Swapping is performed on all trees in memory. During a search,
            \poycommand{swap} collects information about the
            visited trees and perform various kinds of checkpoints to reduce
            information loss in case if \poy crashes.

            \poycommand{swap} is also used as an argument for other
            commands to specify a local search strategy in other contexts, for example,
            in calculating support values using the command
            \nccross{calculate\_support}{calculatesupport}.
            
            All arguments of \poycommand{swap} are optional and their order
            is arbitrary.
            
        \end{poydescription}

	\begin{arguments}

	\begin{argumentgroup}{Neighborhood}
	{The basic standard procedures for local search in phylogenetic
            analysis are SPR and TBR~\cite{swofford1990}. The arguments in this group define the
            parameters of these methods.
            
            The nearest-neighbor interchanges (NNI) swapping strategy is implemented by combining 
            the arguments \poyargument{spr} and \poyargument{sectorial} (see \emph{Join method} group of
            arguments): \poycommand{swap(spr, sectorial:1)}.
            \index{general}{NNI|see{swap}}
            \index{general}{nearest-neighbor interchanges|see{swap}}}
            \label{swap_neigh}

	      \argumentdefinition{alternate}{}
	     {Performs \poyargument{spr} and \poyargument{tbr}
                swapping iteratively until a local optimum is found.
                This is a specific strategy of performing \poyargument{tbr},
                as the trees visited by \poyargument{spr} are a subset
                of those visited by \poyargument{tbr}.}
                {}

            \argumentdefinition{spr}{\optional{once}}
	{This argument performs \poyargument{spr} swapping, starting
                from the current trees in memory and subsequently repeating
                the SPR procedure until  a local optimum is found. If the optional value
                \poyargument{once} is specified, \poyargument{spr} 
                stops once the first tree with better cost is found.} 
                {}

            \argumentdefinition{tbr}{\optional{once}}
	      {This argument performs \poyargument{tbr} swapping, starting
                from the current trees in memory and subsequently repeating
                the TBR procedure until  a local optimum is found.  If the optional value
                \poyargument{once} is specified, \poyargument{tbr} 
                will stop once the first tree with better cost is found.}
                {}

        \end{argumentgroup}

    \begin{argumentgroup}{Trajectory}
        {The following arguments define the direction of the search in the defined
        neighborhood.}

        \argumentdefinition{around}{}
            {Similar to \poyargument{current\_neighborhood}, this
            argument changes the trajectory of a search by
            completely exploring the neighborhood of the current
            tree in memory and choosing the best swap position
            in its neighborhood first before continuing.
            The default in \poy is to choose the first one
            available that shows a better cost than the current
            best cost.}
            {}

        \argumentdefinition{annealing}{\obligatory{(\poyfloat, \poyfloat)}}
            {Uses simmulated annealing~\cite{Kirkpatrick1983}. If the argument's value is $(a, b)$, 
            \poy accepts a tree with cost $c$ when the best known tree has
            cost $d$ with probability $\exp{(- (c - d) / t)}$, where
            $t = a \times \exp{- i / b}$ and $i$ is the number of tree
            evaluated in the local search.}
            {}

        \argumentdefinition{drifting}{\obligatory{(\poyfloat, \poyfloat)}}
            {Uses \poy drifting function~\cite{goloboff1999}. If the argument's value is
            $(a, b)$, then \poy always accepts a tree with better cost than
            the current best cost, with probability $a$ a tree with equal cost,
            and with probability $1 / b + d$ a tree with cost $d$ greater
            than the current best cost.}
            {}

	\end{argumentgroup}

	\begin{argumentgroup}{Branch break order}
		{During the local search, a branch is broken and local branch swapping
		is performed (see \emph{Neighborhood} group of arguments), the
		precise choice of which branches
        should be broken first can affect both the speed and the local
        optimum found by the program. The following arguments select among
        the different strategies available in \poy.}
        
        \argumentdefinition{once}{}
            {Breaks each edge only once during a local search; that is, if a
            broken edge does not yield a better tree, it is never broken again,
            no matter how many changes occur in the tree.}
            {once}

        \argumentdefinition{randomized}{}
            {Chooses edges uniformly at random for breakages.}
            {}

        \argumentdefinition{distance}{}
            {Gives higher priority to those edges with the greatest length.}
            {}

    \end{argumentgroup}

    \begin{argumentgroup}{Join method}
        {After breaking a tree (using SPR or TBR), the following 
        arguments control the selection of the positions to join the broken
        clades.}

            \argumentdefinition{constraint}{\optional{\poyint | 
            (depth:\poyint, file:\poystring)}}
                {The constraint argument for the \poycommand{swap} command sets
                constraints on the join locations during the search using both a tree and an optional maximum distance
                from the break branch. Only sets defined either in the input
                file, or in the strict consensus of the files in memory to consider during swapping. An integer value of
                \poyargument{depth} specifies the maximum distance from the
                break branch to attempt joins. The string value for
                \poyargument{file} specifies an input file containing a singe
                tree that defines topological constraints. Under default settings,
                \poyargument{constraint}
                will use a consensus tree from the files in memory and perform
                swapping with the value of \poyargument{depth} set to \texttt{0}
                (no maximum distance is specified).}
                {swapconstraint}

            \argumentdefinition{all}{\optional{\poyint}}
                {Turn off all preference strategies to make a join, simply try
                all possible join positions for each pair of clades generated
                after a break.}
                {}

            \argumentdefinition{sectorial}{\optional{\poyint}}
                {Do not join in edges at distance greater than the value of the argument
                from the broken edge, where the distance is the number of edges
                in the path connecting them. If no argument is given, then no
                distance limit is set.}
                {}
		
    \end{argumentgroup}

    \begin{argumentgroup}{Reroot order}
        {During TBR, the following options control the order of the rerooting.}

        \argumentdefinition{bfs}{\optional{\poyint}}
            {Reroots using breath first search~\cite{cormen2001} from the broken edge, within the
            arguments value distance from the root of the clade. If no value is
            given, there is no limit distance for the rerooting. By default, \poyargument{bfs}
            is used with no limit distance for the rerooting.}
            {}

    \end{argumentgroup}

	\begin{argumentgroup}{Trajectory samples}
	{During the search, \poy visits a large number of trees. For some applications
	 it might be desirable to  collect information about the trees examined during a
	 search: for example, to provide backups of the state of a
        search (in an unlikely crash), or to examine the characteristics of the alignments.
        The difference from the \poycommand{swap} arguments is that the user can
        choose any combination of trajectory samples, and that can be 
        used during the search. None of the trajectory samples is used by
        default.}

	\argumentdefinition{recover}{}
	{Stores the current best tree in memory that can be recovered in
            case of failure. If it is necessary to recover such
            trees after an aborted command, use the command~\ccross{recover}. 
            If the program terminates normally, the stored trees are exactly
            those produced at the end of the \poycommand{swap}. Using
            \poyargument{recover}, however, requires twice as much memory
            compared to swapping without it.}
            {recoverarg}

         \argumentdefinition{timeout}{\obligatory{\poyint}}
	{Specifies the number of seconds after which branch swapping is
           stopped. Use this argument in association with \poycommand{recover}
           to keep the best trees found up to \emph{n} seconds after
           starting the search.}
                {}

        \argumentdefinition{timedprint}{\obligatory{\poyint, \poystring}}
	{\poyargument{timedprint:(n, "trees.txt")} will print the current
            best tree in memory to the file \texttt{trees.txt}, at least every 
            \texttt{n} seconds. However, \poy typically underestimates the amount of
            time and, therefore, the samples can be slightly sparcer. \poyargument{timedprint} 
            can only be used in combination with the argument \poyargument{recover}. This
            argument requires the argument \poyargument{recover} to be specified.}
            {}

		\argumentdefinition{trajectory}{\optional{\poystring}}
			{\poyargument{trajectory:"better.txt"} will store every new tree
                found with a better score during the local search in the file
                \texttt{better.txt}. The string is the filename where the
                trajectory is to be stored, which is optional (indicated by
                brackets); if not added, the trees are printed in the standard
                output (flat interface) or the output window (ncurses
                interface).}
            {} 

        \argumentdefinition{visited}{\optional{\poystring}}
            {\poyargument{visited:"visited.txt"} will store every visited tree
            and its cost during the local search in the file
            \texttt{visited.txt}. The (optional) string is the filename where the
            trajectory is to be stored. If not included, the trees are printed
            in the standard output (flat interface) or the output window (ncurses
            interface).}
            {}
    
    \end{argumentgroup}

    \begin{argumentgroup}{Tree selection}
        {As the tree search proceeds, a tree may or may not be selected to
        continue the search or to return as a result. The following
        arguments determine under what conditions can a tree be acceptable
        during the search.}

        \argumentdefinition{threshold}{\obligatory{\poyfloat}}
            {Sets the percentage cost for suboptimal
            trees that are more exhaustively evaluated during the swap,
            meaning that trees within the threshold are subject to an extra
            round of swapping. For example, if the current
            optimal tree has cost 450, and \poyargument{threshold:10} is specified, trees
            with cost at most 495 are swapped.  \poyargument{threshold} is
            equivalent to \emph{slop} of  \texttt{POY3}.}
            {thresholdswap}

        \argumentdefinition{trees}{\obligatory{\poyint}}
            {Maximum number of best trees that are retained in a search round,
            per tree in memory.}
            {treesswap}

    \end{argumentgroup}
    
	\end{arguments}

    \poydefaults{trees:1, alternate, threshold:0, bfs}{By default, current trees are
    submitted to a round of SPR followed by TBR using breath first search under
    default setting, and  keeping one best tree per each starting tree.}

	\begin{poyexamples} 
		\poyexample{swap()}
            {This command performs swapping under default settings.}

		\poyexample{swap(trees:5)}
            {Submits current trees to a round of SPR followed by TBR. It keeps
            up to 5 minimum cost trees for each starting tree.}

		\poyexample{swap(transform ((all, static\_approx)))}
            {Submits current trees to a round of SPR followed by TBR, using
            static approximations for all sequence characters.}
            
		\poyexample{swap(trees:4, transform ((all, static\_approx)))}
            {Submits current trees to a round of SPR followed by TBR, using
            static approximations for all characters, keeping up to 4 minimum
            cost trees for each starting tree.}
            
        \poyexample{swap(constraint:(depth:4))}
            {Calculates a consensus tree of the files in memory and uses it as
            constraint file, then joins at distance at most 4 from the breaking
            branch. This is equivalent to \texttt{swap(constraint:(4))}.}
	
        \poyexample{swap(constraint:(file:"bleh"))}
            {Reads the tree in file \texttt{bleh} and use it as constraint for the
            search. This is equivalent to \texttt{swap(constraint:("bleh"))}.}	
		
        \poyexample{swap(constraint:(file:"bleh", depth:4))}
            {Uses the tree in the file \texttt{bleh} as a constraint tree and joins at
            distance at most 4 from the breaking branch during the swap.}
        
         \poyexample{swap(recover, timedprint:(5, "timedprint.txt"))}
            {Saves the current best tree to file \texttt{timedprint.txt} every 5 seconds.}
            
	\end{poyexamples}

	\begin{poyalso}
        \ncross{transform}{transformcommand}
	\end{poyalso} 

\end{command}

\begin{command}{transform}{transformcommand}

	\syntax{\obligatory{(\optional{argument list})}} 

	\begin{poydescription} 
            Transforms a character or a list of characters from one type into
            another type. This includes changing in costs for indels and substitution,
            modifying character weights, converting dynamic into static homology characters,
            and transforming nucleotide into chromosomal (and vise versa) characters
            among other operations.
            
            The essential arguments of the command
            \poycommand{transform} include identifiers and methods. The methods
            specify what type of transformation is applied to the set of characters
            specified by identifiers as defined in the description of the command~\ccross{select}.
            Identifiers and methods are included in parentheses and separated by
            a comma. It is important to remember that only identifiers of
            \emph{characters} (such as \poyargument{names}, \poyargument{codes}, among
            others) can be used. The parentheses separate these essential
            arguments from all other optional arguments that might be included
            in the list. Thus, if only identifiers and methods are specified,
            the argument list of \poycommand{transform} is included in double
            parentheses. For example, the command \poycommand{transform((all,
            gap\_opening:1))} contains only an identifier (\poyargument{all}) and a
            method (\poyargument{gap\_opening}).  Minimally, only methods can be
            specified; in that case, the transformation is applied to all
            characters to which the transformation method can be applied and only a
            single set of parentheses is used. For instance,
            \poycommand{transform(gap\_opening:1)}, where
            \poyargument{gap\_opening} defines
            the transformation method.

            There are no default values for \poycommand{transform}, that is if
            no methods are specified (\poycommand{transform()}), the command does nothing.
	\end{poydescription}

	\begin{arguments}
	
        \begin{argumentgroup}{Identifiers}
            Identifiers specify which characters are transformed. Only
            identifiers of characters (\emph{not} terminals) can be used. If
            identifiers are omitted, the transformation to is applied to all
            applicable characters. For example,
            \poycommand{transform((all,tcm:(1,1)))} is equivalent to
            \poycommand{transform((tcm:(1,1)))}. See the command~\ccross{select}
            for detailed description of identifiers.
        \end{argumentgroup}
           
           \begin{argumentgroup}{Methods}
            This set of arguments specifies different transformations that can be applied
            to selected characters. If multiple transformation methods are applied
            sequentially in the same list of arguments, the effect of the methods listed
            earlier might be altered or canceled by methods listed after that. Thus, caution
            must be used in designing complex strategies with multiple character
            transformations. See the note on command order (Section~\ref{commandorder}).

        \argumentdefinition{auto\_static\_approx}{}
            {Evaluates each loaded fragment and, if the number of indels
            appear to be low and stable between topologies, then the character
            is transformed to the equivalent character using static homologies
            with the implied alignment~\cite{wheeler2003}. If no characters are
            specified (using identifiers), all sequence fragments are evaluated.
            This method greatly accelerates searching.}
            {autostaticapprox}

        \argumentdefinition{auto\_sequence\_partition}{}
            {Evaluates each fragment and if a long region appears 
            to have no indels, then the fragment is broken inside that region.
            Any number of partitions can occur along a fragment. Fragmenting
            long sequences greatly accelerates searching.}
            {autosequencepartition}

        \argumentdefinition{fixed\_states}{}
          {Transforms the characters specified in fixed state characters~\cite{wheeler1999a}
          with distances equal to the edition distance between their observed values.
          By default, the application of \poyargument{fixed\_states} transforms
          all molecular characters. To specify a subset of characters, an identifier
          must be used in conjunction with \poyargument{fixed\_states}.}
          {fixedstates}
        \argumentdefinition{gap\_opening}{\obligatory{\poyint}}
            {Sets the cost of opening a block of gaps to the specified value. Note that
            this cost is in addition to the standard cost of the insertion as
            specified by a given transformation cost matrix.
            The default in \poy is not to have extension
            gap cost (\poyargument{gap\_opening:0}). If the gap
            opening cost is
            $a$, and $indel(x)$ is the cost cost of inserting (or deleting) a
            base $x$ according to the tcm assigned to the character, the total
            cost of inserting (or deleting) the sequence $s[0...n]$ is $a +
            tcm(s[0]) + tcm(s[1]) + ... + tcm(s[n - 1]) + tcm(s[n]).$} 
            {gapopening}

        \argumentdefinition{multi\_static\_approx}{}
            {Calculates the implied alignment for each tree in memory
            and convert them to static homology characters using the alignment's
            cost regime. The new character set will be the union of all those
            characters generated for all the trees~\cite{wheeler1995a}. This option is intended only
            for heuristic search purposes.}
            {multistaticapprox}

            \argumentdefinition{prealigned}{}
                {Treats the sequences as prealigned and uses the
                cost regime according to the specified transformation cost
                matrix. All other cost parameters are ignored (including affine
                gap costs).}
                {prealignedtransform}
            
        \argumentdefinition{sequence\_partition}{\obligatory{\poyint}}
            {Partition the sequences in the argument's value number of
            fragments.}
            {sequencepartition}

        \argumentdefinition{static\_approx}{\optional{\poylident}}
            {Transforms the sequences to the static homology characters
            corresponding to their implied alignments and their transformation
            cost matrix~\cite{wheeler2003}. The resulting characters and their number will vary
            depending on the characteristic of transformation cost matrix
            assigned to each sequence. For example, if the cost of both substitutions
            and indels is \texttt{1}, then one non-additive character is created per
            each homologous position in the implied alignment. If the cost of
            substitutions is \texttt{1} and the cost of indels  is \texttt{2}, then
            one character is created for each homologous position, and one extra character for
            each homologous position with gaps. In more complex cases, a Sankoff character is
            created.
            
            The lident value \texttt{remove} excludes all uninformative characters
            information (except autapomorphies), whereas the value \texttt{keep}
            retains these characters. The default is \texttt{remove}.}
            {staticapprox}
            
            \begin{statement}
  	  The transformation of dynamic into static homology characters cannot be reverted.
	  Therefore, caution must be taken when the transformation is applied. For example,
	  if sequence characters  have been transformed into static characters to calculate
	  jackknife or bootstrap support values based on sampling of individual nucleotides,
	  all commands executed subsequently will be applied to the transform data.
	\end{statement}
	
	\begin{statement}
  	  It is important to remember that the local optimum for the dynamic homology
	  characters can differ from that for the static homology characters based on the
	  same sequence data. Therefore, performing additional searches on the transformed
	  data (for example, in calculating support values based on individual nucleotides
	  rather than on sequence fragments) can produce a discrepancy in tree costs.
	\end{statement}

        \argumentdefinition{trailing\_insertion}{\obligatory{\poystring /(\poyint list)}}
            {The tail and prepend costs specify the cost of having an insertion of
            each element in the alphabet at the beginning or end
            of a sequence. The string is the name of a file containing the cost of
            a trailing insertion corresponding to each of the elements
            in the alphabet separated by spaces. The last element in the list is the
            cost of the indel of a gap (should be 0). Instead of a file, the list can
            be the input of the argument, in the same order, separated by commas.
            Synonym of the argument \poyargument{ti}.} 
            {trailinginsertion}

        \argumentdefinition{trailing\_deletion}{\obligatory{\poystring /(\poyint list)}}
            {The tail and prepend costs specify the cost of having a deletion
            of each element in the alphabet at the beginning or end
            of a sequence. The string is the name of a file containing the cost of
            a trailing deletion corresponding to each of the elements
            in the alphabet separated by spaces. The last element in the list is the
            cost of the indel of a gap (should be 0). Instead of a file, the list can
            be the input of the argument, in the same order, separated by commas.
            Synonym of the argument \poyargument{td}.} 
            {trailinginsertion}

             \argumentdefinition{tcm}{\obligatory{(\poyint, \poyint)}}
            {Defines transformation cost matrix. The first integer value specifies
            substitution cost, the second integer value defines indel cost. By default,
            the cost of substitution is \texttt{1}, and the cost of an indel is \texttt{2}
            (\poyargument{tcm:(1,2)}).}
            {transformtcm}

            \argumentdefinition{tcm}{\obligatory{\poystring}}
            {Defines transformation cost matrix by importing a file (specified by
            the string value) that contains a user defined nucleotide
            transformation cost matrix. 
            The transformation cost matrix file contains five rows and columns
            with values listed in the following order (left to right and top to
            bottom): adenine, cytosine, guanine,
            thymine/uracil, and indel. The costs must be symmetrical (that is, the
            cost of the A to T substitution is equal to the cost of T to A
            substitution). For example:
	        \begin{center}
            \texttt{0 2 1 2 4 \\
            2 0 2 1 4 \\
            1 2 0 2 4 \\
            2 1 2 0 4 \\
            4 4 4 4 0} 
            \end{center}
            }
            {transformtcmmatrix}

            \argumentdefinition{weight}{\obligatory{argument}}
            {Changes the cost of specified characters by a
            constant value (weight) which is specified by either a
            float or an integer value.} 
            {weight}

            \argumentdefinition{weightfactor}{\obligatory{argument}}
            {Changes the cost of specified characters by a
            multiplicative factor (weight factor) which is specified by either a
            float or an integer value.} 
            {weightfactor}

      \end{argumentgroup}
           
       \begin{argumentgroup}{Chromosomal transformation methods}
           For chromosome and genome character types, \poy optimizes nucleotide-, 
           locus-, and chromosome-level variation simultaneously. The arguments in this group
           transform nucleotide characters into chromosomal character
           to allow for translocations, inversions, and indel events at the locus-level in a
           chromosome and chromosome level in a genome.
           
          Functions to calculate breakpoint and inversion distances between two
	sequences of gene orders are taken from GRAPPA, Genome
	Rearrangements Analysis under Parsimony and other Phylogenetic Algorithms,
	available at \\ \texttt{http://www.cs.unm.edu/\~{}moret/GRAPPA/}.
    
         \argumentdefinition{breakinv\_to\_seq}{}
            {Transforms \poyargument{breakinv} character type into \poyargument{custom\_alphabet} character
            type. This transformation prevents the use of rearrangement operations.} 
            {breakinvtoseq}

        \argumentdefinition{seq\_to\_breakinv}{\obligatory{([argument list])}}
            {Transforms \poyargument{custom\_alphabet} character type into \poyargument{breakinv} character type to allow for rearrangement operations
            (translocations and inversions; duplications are not currently
            supported).} This argument is useful, for example, when
            custom\_alphabet characters are used to define a sequence of
            individual genes and one is interested in detecting potential
            change in their order on a chromosome. See the
            command~\ccross{read} for the description on how to load \poyargument {custom
            alphabet} and \poyargument{breakinv} character types. The optional list of arguments
            includes the arguments of the \poyargument{dynamic\_pam} that can also
            be specified subsequently, as a separate step.
            
        \argumentdefinition{seq\_to\_chrom}{\obligatory{([argument list])}}   
            {Transforms nucleotide type data into \\ chromosome type data to allow
            rearrangements, inversions, and locus-level indel operations.} The
            chromosome-specific options (\emph{e.g.} breakpoint,
            locus--insertion, and locus-deletion costs) can be specified by the argument
            \poyargument{dynamic\_pam}. If no \poyargument{dynamic\_pam} values
            are specified, its default values are applied.
            
        \argumentdefinition{dynamic\_pam}{\obligatory{([argument list])}}
            {Specifies parameters for creating chro\-mosome- and genome-level HTUs (medians).
            The arguments of \poyargument{dynamic\_pam} specify the
            method of calculating distance between pairs of chromosomes
            (\poyargument{inversion} and \poyargument{breakpoint}), costs
            of locus-level events (\poyargument{inversion}, \poyargument{breakpoint},
            \poyargument{locus\_indel}), take into
            account whether the chromosome is linear or circular
            (\poyargument{circular}), and implement a number of heuristic
            procedures to accelerate computations when working with chromosome
            data type (\poyargument{seed\_length}, \poyargument{median},
           \poyargument{swap\_med}, \poyargument{rearranged\_len}, \poyargument{approx}).
           If the\poycommand{set} argument \poyargument {iterative} is implemented in the analysis, the 
           \poyargument{dynamic\_pam} parameter \poyargument {max\_3d\_len} can be used to set 
           a maximum sequence length for the alignment of three sequences to reduce the search execution time.  
           If this parameter is not specified the default value of \poyargument {max\_3d\_len} is set to infinity.
            Under default settings, the distance between two chromosomes is calculated using
            \poyargument{breakpoint} and the rest of the arguments are executed
            under their default settings.}
            {dynamicpam}
	
	\begin{statement}
  	Note that the arguments \poyargument{breakpoint} and \poyargument{inversion}
	are \emph{alternative} methods of calculating distance between two chromosomes.
	Therefore, they cannot be used simultaneously. If both arguments are specified,
	the latter will be executed. The order of other arguments of
	\poyargument{dynamic\_pam} is arbitrary. 
	\end{statement}
	
            \begin{description}
            
            	\argumentdefinition{approx}{\obligatory{\poybool}}
                        {Approximates chromosome medians using a fixed-states
                        approach. This is most useful to accelerating tree
                        building and searching operations for large chromosomal
                        data sets. The boolean value \texttt{true} applies the
                        fixed-states optimization. The default value is
                        \texttt{false}.}
                        {approx}
                    
                    \argumentdefinition{locus\_breakpoint}{\obligatory{\poyint}}
                        {Calculates the breakpoint distance~\cite{blanchetteetal1997}
                        between two pairs
                        of chromosomes given the cost for rearrangement
                        specified by an integer value. The breakpoint distance
                        takes into account rearrangements but not inversions.
                        Note, that this argument \emph{cannot} be used in
                        conjunction with \poyargument{inversion}. The default
                        value of \poyargument{breakpoint} is \texttt{10}.} 
                        {breakpoint}
                        
                        \argumentdefinition{circular}{\obligatory{\poybool}} 
                        {Specifies if chromosome is circular (boolean value 
                        \texttt{true}) or linear (boolean value \texttt{false}).
                        The default value of \poyargument{circular} is
                        \texttt{false} (linear chromosome).}
                        {circular}
                        
                      \argumentdefinition{chrom\_breakpoint}{\obligatory{\poyint}} 
      		 {Calculates the breakpoint distance~\cite{blanchetteetal1997}
                        between two sequences of multiple chromosomes given the cost for
                        rearrangement specified by an integer value. The breakpoint distance
                        takes into account locus rearrangements between non-homologous
                        chromosomes (translocations) but not inversions. The default value of
                        \poyargument{chrom\_breakpoint} is \texttt{100}.}
                        {chrombreakpoint}
		
		\argumentdefinition{chrom\_hom}{\obligatory{\poyfloat}}
                        {Specifies the lower limit of distance between two chromosomes
                        beyond which the chromosomes are not considered to be
                        homologous. The default value of \poyargument{chrom\_hom }
                        is \texttt{0.75}.}
                        {chromhom}
                        
                       \argumentdefinition{chrom\_indel}{\obligatory{(\poyint, \poyfloat)}}
                        {Specifies the cost for insertion and deletion of a chromosome in analysis of
                        multiple chromosomes. The integer value sets gap opening
                        cost ($o$), whereas the float value sets gap extension
                        cost ($e$).  The indel cost for a fragment of length $l$ is
                        specified by the following formula:
                       $o + l \times e$. The default values are $o=10, e=1.0$.}
                        {chromindel}
                        
                    \argumentdefinition{inversion}{\obligatory{\poyint}}
                        {Calculates the inversion distance~\cite{hanenhalliandpevzner1995}
                        between
                        two chromosomes given the cost for inversion
                        specified by the integer value. The inversion distance
                        takes in consideration rearrangements and
                        inversions.
                        Note, that this argument \emph{cannot} be used in
                        conjunction with \poyargument{breakpoint}.} 
                        {inversion}  

                    \argumentdefinition{locus\_indel}{\obligatory{(\poyint, \poyfloat)}} 
                        {Specifies the cost for insertion/deletion of a
                        chromosome segment. The integer value sets the gap opening
                        cost ($o$), whereas the float value sets the gap extension
                        cost ($e$).  The indel cost for a fragment of length $l$ is
                        specified by the following formula:
                       $o + l \times e$. The default values are $o=10, e=1.0$.}
                        {locusindel}

                    \argumentdefinition{median}{\obligatory{\poyint}}
                        {Specifies the number alternative locus and chromosome
                        rearrangements of the best cost selected (randomly) for
                        each HTU (median). Limiting the number of rearrangements
                        stored in memory (smaller value of \poyargument{median})
                        is heuristic strategy to accelerate calculations at the
                        expense of thoroughness of the search. By default, only 1
                        rearrangement is retained (the first one found). If more than
                        one rearrangement is specified, the selected number of
                        rearrangements is selected in random order from the pool of
                        all generated rearrangements.}
                        {median}

                    \argumentdefinition{seed\_length}{\obligatory{\poyint}}
                        {Specifies the minimum length of identical (invariant,
                        completely conserved) contiguous sequence fragments
                        during comparison between two chromosomes. The integer
                        value of \poyargument{seed\_length} is the number of
                        nucleotides. Correct identification of such fragments
                        facilitates detecting chromosome rearrangement events and
                        accelerates other operations (such as tree building and
                        swapping). However, if \poyargument{seed\_length} value
                        is set too low (allowing for detection of short, multiple
                        fragments that are likely to occur frequently in a genome) or
                        if it is set too high (that might result in no identical fragments
                        detected), the speed of subsequent procedures can potentially
                        decrease. The optimal parameters depend on specifics of a
                        given dataset.
                        The default value of \poyargument{seed\_length} is \texttt{9}.}
                        {seedlength}
                        
                      \argumentdefinition{sig\_block\_len}{\obligatory{\poyint}}
                          {Creates a pairwise alignment between two chromosomes and
                          detecting conserved areas (``blocks''). However, only blocks of
                          lengths (in number of nucleotides) greater or equal to the integer
                          value of \poyargument{sig\_block\_len} are considered as hypothetically
                          homologous blocks and used as anchors to divide chromosomes into
                          fragments. Increasing the value of \poyargument{sig\_block\_len} decreases
                           the chance of inferring small-size rearrangements. The default value is \texttt{100}.} 
                          {sigblocklen}
                        
  		   \argumentdefinition{rearranged\_len}{\obligatory{\poyint}}
           	           {Two seeds are said to be \emph{non-rearranged}, 
               	           if their distance is less than  the predefined 
              	           threshold value set for $rearranged\_len$. 
                             In other words, it is unlikely that rearrangement 
              	           operations can occur between two non-rearranged seeds 
                             if they are connected. The default value is \texttt{100}}{rearrangedlen} 
 


                    \argumentdefinition{swap\_med}{\obligatory{\poyint}}
                        {Specifies the maximum number of swapping iterations
                        to search for best pairwise
                        alignment of two chromosomes taking into account locus-level
                        rearrangement events. Limiting the number of swapping
                        iterations accelerates the search at the expense of
                        thoroughness. The default value is \texttt{1}.}
                        {swapmed}

            \end{description}	

	\end{argumentgroup}
	\end{arguments}
	
    \poydefaults{}{If no arguments are given, this command does nothing.}

	\begin{poyexamples} 
		\poyexample{transform((all, tcm:(1,1)))} 
            {Applies the transformation cost matrix (1,1) to all characters,
            meaning that substitutions and gaps receive the same weight.}

		\poyexample{transform((all, tcm:"molmatrix"))} 
            {Applies the character transformation matrix "molmatrix" to all
            characters.}
            	
		\poyexample{transform((all, tcm:(1,1)))}{This command
		is equivalent to \poycommand{transform((dynamic, tcm:(1,1)))}.}
		
		\poyexample{transform(tcm:(1,1), gap\_opening:1)}
		{Applies the transformation cost matrix and the gap opening cost
		to all characters. In this example the cost for substitutions is \texttt{1},
		the gap opening cost is \texttt{2} (\texttt{1} set by \poyargument{gap\_opening}
		+ \texttt{1} set by \poyargument{tcm}), and the gap extension cost is \texttt{1}
		(set by \poyargument{tcm}).}
		
		\poyexample{transform(tcm:(2,2), ti:(1,1,1,1,0), td:(1,1,1,1,0))}
		{Assigns to all characters the symmetric transformation cost
		matrix with cost \texttt{2} for every indel and substitution, but for those
		insertions and deletions at the ends of the sequences, the cost
		assigned will only be \texttt{1}.}
		
		\poyexample{transform((static, weightfactor:2))}
            {This command reweights all the static homology characters
            by a multiplicative factor of \texttt{2}, while keeping the weighting
            scheme that has been specified before.}
		
		\poyexample{transform((static, weight:4.2))}{Applies the same
		weight (a float value \texttt{4.2}) to all static homology characters.}
		
		\poyexample{transform((dynamic, weight:4))}{Applies the same
		weight (an integer value \texttt{4}) to all dynamic homology characters.}

        \poyexample{transform((all, tcm:(1,1)), (names:("gen1",
        "gen2"), \\ static\_approx), (names:("gen3"), tcm:"molmatrix"))}  
            {Applies tcm (1,1) to all characters, then applies static approx
            using that tcm to characters in files gen1 and gen2, and for file
            gen3, it invokes a different transformation cost matrix, contained
            in the file molmatrix. Beware that the file name should be exactly
            as it was reported with report (data), which differs from the actual
            file name (\poycommand{report (data)} reports files as fileX:N).}

        \poyexample{transform((all, tcm:(1,1)), (names:("gen1:3",
        "gen2:10", \\"gen3:1", "gen4:5"), static\_approx), (names:("gen5", \\
        "gen6"), tcm:"Molmatrix1"))}
            {Applies \poyargument{tcm (1,1)} to all characters, then applies
            static approximation to the sequence data contained in files \texttt{gen1}, \texttt{gen2},
            \texttt{gen3}, and \texttt{gen4} according to this transformation cost
            matrix, and applies the custom transformation cost matrix contained in the file
            \texttt{Molmatrix1} to the sequence data contained in files \texttt{gen5} and
            \texttt{gen6}.}
         
         \poyexample{transform(fixed\_states)}
         {Transformed all sequence characters into fixed states characters.}
            
          \poyexample{transform((names:("gen1", "gen4"), fixed\_states))}
           {Transformed only specified sequence characters (\texttt{gen1} and
           \texttt{gen4}) into fixed states characters.}
           
          \poyexample{transform((all, seq\_to\_breakinv:()))}
          {In this example all sequence data is transformed into breakinv data
          type under default settings of \poyargument{dynamic\_pam}.}
          
          \poyexample{transform(seq\_to\_chrom:(circular:true, locus\_indel:(50, \\1.0)))} 
          {All applicable (\emph{i.e.} sequence) data is transformed into chromosome
          data, which is treated as a circular chromosome, and settings locus-level gap
          opening cost at \texttt{50} and gap extension cost at \texttt{1.0}.}
              
          \poyexample{read (chromosome:("mito")) transform((all,
          dynamic\_pam:\\(breakpoint:10, rearranged\_len:60, median:1, circular:\\false)))} 
              {This example shows a file read (``mito'') containing
              mitochondrial chromosome sequences that is transformed to set the
              breakpoint cost at 10, 60 or more nucleotides are necessary to allow 
              rearrangement between 2 identified seeds, the number of median swap passes at 1,
               and the chromosomes are linear. }
            
	\end{poyexamples}	    

\end{command}


\begin{command}{use}{}

	\syntax{\obligatory{(\poystring)}}

	\begin{poydescription}
         Restores from memory the state of a \poy session (that includes character data,
         selections, trees, all other data and specifications) that had previously been
         saved during the session using the command~\ccross{store}{}. The recalled
         session replaces the current session. The string argument specifies the name
         of the stored state.
         
         In combination with ~\ccross{store}{}, the command \poycommand{use}
         is very useful for exploring alternative  cost regimes and terminal sets
         within a single \poy session.
            
	\end{poydescription}
	
	\begin{poyexamples}
        \poyexample{store("initial\_tcm") \\ transform(tcm:(1,1)) \\ use("initial\_tcm")}
            {The first command, \poycommand{store}, stores the current
            characters and trees under the
            name \texttt{initial\_tcm}. The second command,
            \poycommand{transform}, changes the cost regime of molecular characters,
            effectively changing the data being analyzed. However, the third
            command, \poycommand{use}, recovers the initial state stored under the
            name \texttt{initial\_tcm}.}
    \end{poyexamples}

     \begin{poyalso}
        \cross{store}
        \cross{transform}
    \end{poyalso}

\end{command}

\begin{command}{version}{}

	\syntax{\obligatory{()}}

	\begin{poydescription}
            Reports the \poy version number in the output window of the ncurses
            interface, or to the standard error in the flat interface.
	\end{poydescription}

    \begin{poyexamples}
        \poyexample{version ()}{}
    \end{poyexamples}
\end{command}

\begin{command}{wipe}{}

	\syntax{\obligatory{()}}

	\begin{poydescription}
        Elminates the data stored in memory (all character data, trees, \emph{etc.}).
	\end{poydescription}

    \begin{poyexamples}
        \poyexample{wipe ()}{}
    \end{poyexamples}
\end{command}
           



\chapter{\poy Heuristics: A Practical Guide}
\section{Introduction}

As the level of phylogenetic analysis increases---from individual loci to chromosomes to genomes containing multiple chromosomes---so does computational complexity. In \poy, a significant increase in computational time results from combining in a single process cladogram searching with co-optimization of nucleotide pairwise alignments, rearrangements of loci within a chromosome, and rearrangements of chromosome fragments within the genome . As a result, a phylogenetic analysis involves a set of nested computationally ``hard'' (NP-complete) problems that makes finding the exact solution impossible. In addition, the increasing sequence length heterogeneity (at the levels of nucleotides, loci, and chromosomes) and the ever-growing sizes of datasets further contribute to computational complexity making it impossible to obtain an exact solution in a reasonable time.

To circumvent the problem computational intractability, and, hence, the speed of the analyses, \poy employs a battery of approximate, or heuristic, methods that function at different levels of analysis. As with all heuristic procedures, a tradeoff is involved: a substantial decrease in execution time comes at a price of obtaining possibly less accurate  and less precise results (however, the extent of the tradeoff is difficult to evaluate in the analyses of real large datasets). Therefore, it becomes important to understand the combined effect of different heuristic methods, so that the chosen search strategy balances the computational time with a ``reasonable'' accuracy of the result.

Here we provide general guidelines for using different heuristic methods, explore their combined effect, and suggest the choice of parameters that can be explored to provide the best result for specific cases. Real datasets differ greatly in size and complexity, so that no single optimal strategy can be suggested. These guidelines, however, should enable the investigator to design an efficient strategy that will tailor to the peculiarities of a given dataset.

In addition to heuristic methods, this chapter attempts to assist with the selection of transformation cost regimes. Alternative cost regimes can significantly affect the outcome of the analysis, that becomes particularly apparent in dealing with large, genome-level datasets, where multiple cost regimes are used simultaneously to specify transformations at different levels of analysis. Most difficulties stem from selecting the most reasonable combination of parameters that affect optimization of DNA sequence data at the levels of nucleotides, loci, and chromosomes.

\section{Data treatment}

Direct optimization (see \emph{Character optimization} section below) involves comparing all potential nucleotide homologies between two sequences. Consequently, the time it takes is proportional to the product of the lengths of the sequences compared. This procedure can be time consuming for long and greatly differing in length DNA fragments. In cases where unambiguous (such as long completely conserved regions) sequence fragments can be identified, partitioning the long sequences into smaller fragments delimited by these regions can significantly reduce computational time. Such economy is reached because nucleotide homologies are not examined over the separate partitions. This strategy assumes that the fragments are mutually exclusive and are putatively homologous across terminals.

At the level of nucleotides, individual fragments in a locus can be separated by the pound symbols (``\#'') or contained as individual files (that is, treated as partitions). When ``\#'' are used, their number must be the same across homologous sequences. Alternatively, the argument of \poyargument{auto\_sequence\_partition} of the command \ccross{transform}. At the chromosome level, individual loci can be separated by pipes (``$\vert$'').

\begin{center}
\begin{tabular}{| l  l  p{.35\textwidth}|}
	\hline
	Level of analysis & Heuristic & Implementation \\ \hline \hline
	Nucleotides & Fragment sequences & Manually separating fragments or use
	\poycommand{trnsform(auto\_sequence\_partition)}\\
	Locus & Fragment chromosome & Manually insert pipes separating loci \\
	Chromosomes & NA & NA \\
	\hline	
\end{tabular}
\end{center}

\section{Character optimization}
Minimizing overall cladogram cost is an NP hard problem dependent on the lowest cost assignment of HTU sequences.  POY implements direct optimization (DO; ~\cite{wheeler1996}) and fixed-states optimization (FSO; ~\cite{wheeler1999a}) heuristics to determine the set of HTU sequences comprising the internal nodes of each cladogram constructed.  Direct Optimization decomposes the problem into a series of two-node comparisons, calculating locally optimal solutions, which generates the total cladogram cost.  An advantage of direct optimization is that it allows for the exploration of a large diversity of putative homologies and selects the scheme that yields the most optimal solution. This is useful in analyzing sequences of different length, where site-to-site homologies are uncertain.  Because the procedure is based on a greedy algorithm, it requires multiple iterations (independent initial cladogram builds) and extensive tree searches to reach a potentially global minimum.  In contrast, fixed-states optimization does not calculate HTU sequences but rather optimizes those observed in terminal taxa. These internal node sequences then are diagnosed using dynamic programming based on a matrix of edit costs between sequences.  In the fixed-states implementation cladogram optimization is independent of sequence lengths, and as the number of sequences increase so to does the pool from which the HTU sequences are drawn, thereby improving cladogram cost estimation. Because of these properties fixed-states optimization is recommended as an initial approximation strategy for large data sets of variable length sequences.  

\begin{center}
\begin{tabular}{| l  l  p{.35\textwidth}|}
	\hline
Level of analysis&Heuristic&Implementation \\ \hline \hline
Nucleotides&DO&Default strategy\\
Nucleotides&FSO&\poycommand{transform(fixedstates)}\\
Loci&FSO&\poycommand{transform(dynamic\_pam:(approx))}\\
Chromosomes&NA&NA\\
\hline	
\end{tabular}
\end{center}

Further approximations and economies can be achieved by varying parameters of commands, such as selecting a limited subset of trees for subsequent analysis limiting the number of replicates, and examining intermediary results from an interrupted analysis.

\section{Tree searching}
The heuristic approaches to cladogram searching include random addition of taxa, branch swapping (TBR and SPR), simulated annealing (the ratchet and tree-drifting), and genetical algorithms (tree fusing). These techniques, frequently used in combination, allow a more efficient exploring of tree space and provide the means of finding more globally optimal solutions. These methods are widely used in phylogenetics \cite{felsenstein2004a, wheeleretal2006}, although \poy implements additional modifications of these procedures.

Typical search strategy in \poy involves consecutive application of tree search algorithms that begin with generating multiple, randomly selected starting points [Random Addition Sequences (RAS) or Wagner trees]. The importance of multiple starting trees cannot be overemphasized and a successful search shall maximize the number of RAS. However, making a tree search more exhaustive by increasing the number of starting trees comes at a price of longer computation time. Therefore, it is advised here to estimate the amount of time it takes to complete a single replicate and takes this information in consideration when designing a more exhaustive strategy. The  number of replicates used by \poy practitioners for datasets of moderate size (70-100 terminals) ranges from 100 to 250. Here are some examples of search strategies:
\begin{description}
\item[RAS+SPR/TBR+Ratchet] The strategy is for a thorough search for a data set of 100 or fewer taxa. A diversity of starting points is generated by multiple RAS, each refined by a local search (TBR or a combination of SPR and TBR, the latter is an efficient default strategy in \poy). Ratcheting is used to examine tree space that potentially has not been explored by the local searches.
\item[RAS+SPR/TBR+Ratchet+Tree Fusing]  Adding tree fusing step allows for combining the best sectors of cladograms that can potentially yield a tree of shorter length. Empirical studies showed that adding tree fusing after replicate rounds enhances the results only when dealing with data sets with more than 50 taxa.
\item[RAS+SPR/TBR+Tree Drifting + Tree Fusing]
\item[RAS+SPR/TBR+Ratchet+Tree Drifting+Tree Fusing] Tree Drifting can be used in place of or in addition to the Ratchet.
\item[Input Trees+SPR/TBR+Ratchet+Tree Drifting+Tree Fusing] For more exhaustive searches, the best trees obtained from the initial searches using the strategies outlines above, can be used as input trees for subsequent analyses. In doing so, the RAS step can be omitted because searching starts with trees approximating the globally optimal tree(s).
\end{description}
The aggressiveness of searches can be adjusted by varying parameters of the branch swapping, ratchet, tree fusing, and tree drifting commands.

Further economies can be reached by using a combination of different character optimization methods. For example, initial searches can be conducted under faster static approximation (that converts sequence data into static homology characters; see \emph{Character optimization} section), whereas the final refinement can be performed using direct optimization.

\section{Chromosomal heuristics}
Analysis of chromosomal data requires heuristic procedures to estimate
rearrangement events in addition to nucleotide transformations.  In unannotated
chromosomal sequences \poy uses \poycommand{seed\_length},
\poycommand{sig\_block\_len} and connecting sequence length to identify
homologous fragments (e.g. loci) within nucleotide strings. A seed is a
contiguous nucleotide sequence that is identical in composition between two
chromosomes.  Once detected seeds are used to partition chromosomes into
fragments, or blocks.  If blocks are considered to be homologous and their
relative positions differ between the chromosomes, a rearrangement is
inferred. The command \poycommand {seed\_length} specifies the minimum length of
the seed.  The higher the seed\_length value, the fewer seeds are detected,
that, in turn influence the number of blocks recognized. Conversely, if the
value of seed\_length is low, an increased number of seeds and, consequently, a
greater number of short blocks are detected. The command
\poycommand{sig\_block\_len} sets the minimum nucleotide length of the blocks,
beyond which the block are not considered homologous.If the value of
\poycommand{sig\_block\_len} is low, small-size rearrangements are allowed;
whereas if the value of \poycommand{sig\_block\_len} is high�larger
rearrangements are detected. The connector length parameter sets a threshold
value under which homologous blocks separated by non-homologous regions can be
considered as a single block. 
The default for this parameter is \texttt{100}.  Therefore if two inferred
homologous blocks are separated by less 
than 100 nucleotides they will be treated as a single block in 
calculating of rearrangement events. Thus, the combination of parameters \poycommand{seed\_length}, \poycommand{sig\_block\_len}, and connector length significantly influence the estimation of inferred rearrangements.

Once blocks have been designated, heuristic solutions estimating the number of rearrangement events among chromosomal strings can be implemented using \poycommand{breakpoint} distance, \poycommand {inversion} distance, or \poycommand {approx}. The command \poycommand{breakpoint} calculates HTU medians that minimize the number of adjacency breaks (between ordered block pairs) in terminals.  Estimating rearrangements using inversion distance \poycommand {inversion} is more computationally costly (time consuming) but considers locus inversion events as well as adjacency breaks.  Static approximation of chromosomal blocks (\poycommand {approx}) provides the fastest median calculation but is the least exhaustive of the rearrangement search options. The efficiency of each rearrangement heuristic is also influenced by the median parameter specifying the number of alternate rearrangements considered in the search.  The choice of heuristic and thoroughness with which rearrangement estimation is conducted is contingent upon the properties of chromosomal data being analyzed and the cost assigned to rearrangement events relative to that of locus insertions and deletions (\poycommand {locus\_indel}) and nucleotide transformations.

Further approximations and economies can be achieved by varying parameters of commands, such as selecting a limited subset of trees for subsequent analysis limiting the number of replicates, and examining intermediary results from an interrupted analysis (expand).

%Even though using such imprecise relative terms is �high� and �low� might give an idea of the effect of %combination of different parameter values, the more explicit advice on actual values is desirable. The %following sets of parameters were empirically established to provide reasonable (biologically
%meaningful) results judging from a posteirori diagnosis of the inferred rearrangements. 

\begin{center}
\begin{tabular}{| l  l  p{.35\textwidth}|}
	\hline
Heuristic&Default Cost &Suggested Cost  \\ \hline \hline
Locus & & \\
\poycommand{seed\_length}&9&5-15\\
\poycommand{sig\_block\_len}&100&60-150\\
connector length&100&50-1000\\
\poycommand{breakpoint}&10&10-50\\
\poycommand {inversion}&none&15-35\\
 \poycommand {approx}&false&large data sets\\
  \poycommand {median}&1&1-2\\
   \poycommand {swap\_med}&1&1-2\\
  \poycommand {locus\_indel})&opening 10, extension 1&opening 10, extension 1\\
  %Chromosome\\
  %\poycommand{chrom\_breakpoint}&?&?\\
  %\poycommand{chrom\_indel}&?&?\\
\hline
\end{tabular}
\end{center}

%&DO&Default strategy\\
%Nucleotides&FSO&\poycommand{transform(fixedstates)}\\
%Loci&FSO&\poycommand{transform(dynamic\_pam:(approx))}\\
%Chromosomes&NA&NA\\
%\hline	
%\end{tabular}
%\end{center}

		
%Nucleotide		
%Locus	locus\_breakpoint	
%	locus\_indel	
%	inversion	
%	[seed\_length	
%	swap\_med	
%	[sig\_block\_len	
%Chromosome	chrom\_breakpoint	
%	chrom\_indel	


		
%Nucleotide		
%Locus	locus\_breakpoint	
%	locus\_indel	
%	inversion	
%	[seed\_length	
%	swap\_med	
%	[sig\_block\_len	
%Chromosome	chrom\_breakpoint	
%	chrom\_indel	

\section{Transformation cost regimes}
In the analysis at the level of nucleotides, there are three general approaches to selecting transformation cost regimes most commonly used by \poy practitioners.
\begin{description}
\item[Equal costs] This approach assigns the same cost to all substitutions and indels, and does not take into account gap extension cost. For rationale for using this cost regime see Frost et al. \cite{frost2001} and for other examples of its application see.
\item[Homology maximization] This approach, developed by De Laet \cite{delaet2005}, assigns costs \texttt{2, 3, and 1} to transformations, gap opening, and gap extension respectively. For examples using this methods see
\item[Parameter sensitivity analysis] This method, suggested by Wheeler \cite{wheeler1995}, explores the effect of varying transformation costs by comparing results of analyses conducted under different cost regimes. Partition inconguence can subsequently be computed  for each cladogram and the parameter set that minimizes incongruence is selected as optimal. For examples using this methods see
\end{description}
More specifically, it depends on relative costs of nucleotide- and locus-level transformations. Nucleotide-level transformations are specified by tcm argument, the locus-level rearrangements are specified by locus\_breakpoint or inversion costs. If locus\_level rearrangement costs are extremely high, the rearrangements are not going to be counted. On the other hand, if their cost is very low (equal or slightly above that of the nucleotide-level rearrangements), rearrangements can be frequent (depending on the seed\_block\_len and seed\_length settings).

When DNA sequence data is combined with morphological data, the cost for morphological character transformations is customarily is set to be the same as for substitutions.



\chapter{\poy Tutorials}
These tutorials are intended to provide guidance for more sophisticated applications of \poy that involve multiple steps and a combination of different commands. Each tutorial contains a \poy script that is followed by detailed commentaries explaining the rationale behind each step of the analysis. Although these analyses can be conducted interactively using the \emph{Interactive Console} or running separate sequential analyses using the \emph{Graphical User Interphace}, the most practical way to do this is to use \poy scripts (see \emph{ POY4 Quick Start} for more information on \poy scripts).

It is important to remember that the numerical values for most command arguments will differ substantially depending on type, complexity, and size of the data. Therefore, the values used here should not be taken as optimal parameters.

The tutorials use sample datasets that are provided with \poy installation but can also be downloaded from the \poy site at
\begin{center}
\texttt{http://research.amnh.org/scicomp/projects/poy.php}
\end{center}
The minimal required items to run the tutorial analyses are the \poy application and the sample datafiles. Running these analyses requires some familiarity with \poy interface and command structure that can be found in the preceding chapters.

\section{Combining  search strategies}{\label{tutorial1}}
The following script implements a strategy for a thorough search. This is accomplished by generating a large number of independent initial trees by random addition sequence and combining combining different search strategies that aim at thoroughly exploring local tree space and escape the effect of composite optima by effectively traversing the tree space. In addition, this script shows how to output the status of the search to a log file and calculate the duration of the search. 

\begin{verbatim}
(* search using all data *)
read("*.seq", "morph.ss", aminoacids:("myosin.aa"))
set(seed:1, log:"all_data_search.log", root:"taxon1")
report(timer:"search start")
transform (tcm:(1,2), gap_opening:1)
build(250)
swap(threshold:5.0)
select()
perturb(transform(static_approx), iterations:15,ratchet:(0.2,3))
select()
fuse(iterations:200, swap())
select()
report("all_trees", trees:(total) ,"constree", graphconsensus,
"diagnosis", diagnosis)
report(timer:"search end")
set(nolog)
exit()
\end{verbatim}

\begin{itemize}
\item \texttt{(* search using all data *)} This first line of the script is a comment. While comments are optional and do not affect the analyses, they provide are useful for housekeeping purposes.
\item \texttt{read("*.seq", "morph.ss", aminoacids:("myosin.aa"))}
This command imports all the nucleotide sequence datafiles (all files with the extension \texttt{.seq}), a morphological datafile \texttt{morph.ss} in Hennig86 format, and an aminoacid datafile \texttt{myosin.aa}.
\item \texttt{set(seed:1, log:"all\_data\_search.log", root:"taxon")} The \poycommand{set} command specifies conditions prior to tree searching. The \poyargument{seed} is used to ensure that the subsequent randomization procedures (such as tree building and selecting) are reproducible. Specifying the log produces a file, \texttt{all\_data\_search.log} that provides an additional means to monitor the process of the search. The outgroup (\texttt{taxon1}) is designated by the \poyargument{root}, so that all the reported trees have the desired polarity. By default, the analysis is performed using direct optimization.
\item \texttt{report(timer:"search start")} In combination with \texttt{report(timer:"search end")}, this commands reports the amount of time that the execution of commands enclosed by \poyargument{timer} takes. In this case, it reports how long it takes for the entire search to finish. Using timer is useful for planning a complex search strategy for large datasets that can take a long time to complete: it is instructive, for example, to know how long a search would last with a single replicate (one starting tree) before starting a search with multiple replicates.
\item \texttt{transform (tcm:(1,2), gap\_opening:1)} This command sets the transformation cost matrix for molecular data to be used in calculating the cost of the tree. Note, that in addition to the substitution and indel costs, the \poycommand{transform} specifies the cost for gap opening.
\item \texttt{build(250)} This commands begins tree-building step of the search that generates 250 random-addition trees. A large number of independent starting point insures that thee large portion of tree space have been examined.
\item \texttt{swap(threshold:5.0)} \poycommand{swap} specifies that each of the 250 trees is subjected to alternating SPR and TBR branch swapping routine (the default of \poy). In addition to the most optimal trees, all the suboptimal trees found within 5\% of the best cost are thoroughly evaluated. This step ensures that the local searches settled on the local optima.
\item \texttt{select()} Upon completion of branch swapping, this command retains only optimal and topologically unique trees; all other trees are discarded from memory. 
\item \texttt{perturb(transform(static\_approx), iterations:15,ratchet:(0.2,3))} This command subjects the resulting trees to 15 rounds of ratchet, re-weighting 20\% of characters by a factor of 2. During ratcheting, the dynamic homology characters are transformed into static homology characters, so that the re-weigted characters nucleotides. This step, that begins at multiple local maxima, is intended to further traverse the tree space in search of a global optimum.
\item \texttt{fuse(iterations:200, swap())} In this step, up to 200 swappings of branches identical in terminal composition but different in topology, are performed between pairs of best trees recovered in the previous step. This is another strategy for further exploration of tree space. Each resulting tree is further refined by local branch swapping under the default parameters of \poycommand{swap}.
\item \texttt{select()} Upon completion of branch swapping, this command retains only optimal and topologically unique trees; all other trees are discarded from memory.
\item \texttt{report("all\_trees", trees:(total) ,"constree", graphconsensus, "diagnosis", diagnosis)} This command produces a series of outputs of the results of the search. It includes a file containing best trees in parenthetical notation and their costs (\texttt{all\_trees}), a graphical representation (in PostScript format) of the strict consensus (\texttt{constree}), and the diagnoses for all best trees (\texttt{diagnosis}).
\item \texttt{report(timer:"search end")} This command stops timing the duration of search, initiated by the command \texttt{report(timer:"search start")}.
\item \texttt{set(nolog)} This command stops reporting any output to the log file, \texttt{all\_data\_search.log}.
\item \texttt{exit()} This commands ends the \poy session.
\end{itemize}

\section{Searching under iterative pass}{\label{tutorial2}}
The following script implements a strategy for a thorough search under iterative pass optimization. The iterative pass optimization is a very time consuming procedure that makes it impractical to conduct under this kind of optimization (save for very small datasets that can be analyzed within reasonable time). The iterative pass, however, can be used for the most advanced stages of the analysis for the final refinement, when a potential global optimum has been reached through searches under other kinds of optimization (such as direct optimization). Therefore, this tutorial begins with importing an existing tree (rather than performing tree building from scratch) and subjecting it to local branch swapping under iterative pass.

\begin{verbatim}
(* search using all data under ip *)
read("*.seq", "morph.ss", aminoacids:("myosin.aa"))
read("inter_tree.tre")
transform (tcm:(1,2), gap_opening:1)
set(iterative)
swap()
select()
report("all_trees", trees:(total) ,"constree", graphconsensus,
"diagnosis", diagnosis)
exit()
\end{verbatim}

\begin{itemize}
\item \texttt{(* search using all data under ip *)} This first line of the script is a comment. While comments are optional and do not affect the analyses, they provide are useful for housekeeping purposes.
\item \texttt{read("*.seq", "morph.ss", aminoacids:("myosin.aa"))} This command imports all the nucleotide sequence datafiles (all files with the extension \texttt{.seq}), a morphological datafile \texttt{morph.ss} in Hennig86 format, and an aminoacid datafile \texttt{myosin.aa}.
\item \texttt{read("inter\_tree.tre")} This command imports a tree file, \texttt{inter\_tree.tre}, that contains the most optimal tree from prior analyses. 
\item \texttt{transform (tcm:(1,2), gap\_opening:1)} This command sets the transformation cost matrix for molecular data to be used in calculating the cost of the tree. Note, that in addition to the substitution and indel costs, the \poycommand{transform} specifies the cost for gap opening.
\item \texttt{set(iterative)} This command sets the optimization procedure to iterative pass.
\item \texttt{swap(around)} This commands specifies that the the imported tree is subjected to alternating SPR and TBR branch swapping routine (the default of \poy) following the trajectory of search that completely evaluates the neighborhood of the tree (by using \poyargument{around}).
\item \texttt{select()} Upon completion of branch swapping, this command retains only optimal and topologically unique trees; all other trees are discarded from memory.
\item \texttt{report("all\_trees", trees:(total) ,"constree", graphconsensus, "diagnosis", diagnosis)} This command produces a series of outputs of the results of the search. It includes a file containing best trees in parenthetical notation and their costs (\texttt{all\_trees}), a graphical representation (in PostScript format) of the strict consensus (\texttt{constree}), and the diagnoses for all best trees (\texttt{diagnosis}).
\item \texttt{exit()} This commands ends the \poy session.
\end{itemize}

%\section[Fusing and Ratcheting]{Advanced Search II: Tree Fusing and Fragment
%Ratcheting}{\label{tutorial3}}

%This tutorial builds on Tutorial~\ref{tutorial2} to use tree fusing and fragment ratcheting 
%that help to escape suboptimal islands. This tutorial also introduces matrix and 
%tree output in NONA format.

%\begin{enumerate}
%\item Type: 
%\begin{verbatim}
%    build (100) [enter]
%    select (unique) [enter]
%    fuse () [enter]
%    select () [enter]
%    perturb (iterations:10, ratchet:(0.2,5)) [enter]
%    select () [enter]
%    swap (threshold:10) [enter]
%    select () [enter]
%    report ("tutorial3", graphtrees) [enter]
%    report ("alignment3.ss", phastwinclad) [enter]
%\end{verbatim}

%The above commands will perform the following tasks using default parameters:
%\begin{itemize}
%\item Generate 100 random addition sequence Wagner trees.
%\item Discard duplicate trees.
%\item Perform cladogram searching by fusing one pair of trees in memory. Note: Tree 
%fusing requires at least two trees; if there was only one tree in memory, 
%an error message will be generated. 
%\item Discard suboptimal and duplicate trees (\emph{i.e.}, retain only optimal trees).
%\item Perform 10 successive repetitions of a fragment-based ratchet by randomly 
%selecting 20\% of the fragments and upweighting them by a factor of five.
%\item Discard suboptimal and duplicate trees (i.e., retain only optimal trees).
%\item For each tree in memory, alternate SPR and TBR branch swapping of the tree 
%and all trees found within 10\% of the cost of the tree.
%\item Discard suboptimal and duplicate trees (\emph{i.e.}, retain only optimal trees).
%\item Output publication-quality cladogram(s) of optimal tree(s) to a postscript 
%file tutorial3.ps. (Note: \poy automatically adds the \texttt{.ps} extension.)
%\item Output NONA file with the implied alignment for one (if multiple) optimal 
%trees and all optimal trees in the file alignment3.ss.
%\end{itemize}

%\item View publication-quality cladogram(s) by opening tutorial3.ps in your chosen 
%postscript viewer.
%\item View implied alignment, optimal tree(s), and manipulate data by opening the 
%file alignment3.ss in WinClada or MacClade.

%\end{enumerate}

%\section[Trees and Step Matrices]{Advanced Search III: Input Trees and Step
%Matrices}{\label{tutorial4}}

%This tutorial builds on Tutorials~\ref{tutorial2} and~\ref{tutorial3}, and introduces the use of input trees 
%and complex differential cost step matrices.

%\begin{enumerate}
%\item Type: 
%    \begin{verbatim}
%    read ("trees.txt") [enter]
%    transform ((all, tcm:"g4ts1tv2.txt"), 
%    (static, weightfactor:4)) [enter]
%    swap (threshold:10) [enter]
%    select () [enter]
%    report ("tutorial5", graphtrees) [enter]
%    \end{verbatim}

%The above commands will perform the following tasks using default parameters:
%\begin{itemize}
%\item Read cladograms from the input file trees.txt.
%\item Apply specified transformation costs to data. The transformation cost 
%matrix in the file g4ts1tv2.txt is applied to the dynamic homology 
%characters (i.e., unaligned DNA sequences) to assign indel events a cost 
%of 4, transitions 1, and transversions 2. All static homology characters 
%(i.e., the phenotypic characters in the file morph.txt) are upweighted by 
%a factor of 4, with additivities and previous relative weights unchanged 
%(e.g., a character with transformations costs of 2 would be now have a 
%transformation cost of 8).
%\item For each tree in memory, alternate SPR and TBR branch swapping of the tree 
%and all trees found within 10\% of the cost of the tree.
%\item Discard suboptimal and duplicate trees (i.e., retain all most optimal 
%trees).
%\item Output publication-quality cladogram(s) of optimal tree(s) to a postscript 
%file tutorial5.ps. (Note: \poy automatically adds the .ps extension.)  
%\end{itemize}

%\item View publication-quality cladogram(s) by opening tutorial4.ps in your chosen 
%postscript viewer.
%\item View cladogram(s) in parenthetical format by opening tutorial4\_trees.txt in 
%your chosen text editor.
%\item View basic tree statistics by opening tutorial4\_stats.txt in your chosen 
%text editor.
%\end{enumerate}

%\section[Bremer Support]{Support I: Bremer Support}

%This tutorial builds on the previous tutorials to illustrate Bremer support 
%calculation.
%\begin{enumerate}
%\item Type: 
%    \begin{verbatim}
%    transform ((all, tcm:(1,1)), (static, weightfactor:1)) [enter]
%    swap () [enter]
%    calculate_support (bremer, build (trees:5), swap (trees:2)) [enter]
%    report (supports) [enter]
%    \end{verbatim}

%The above commands will perform the following tasks using default parameters:
%\item Apply a weight of one to all transformations.
%\item For each tree in memory, alternate SPR and TBR branch swapping of the 
%optimal tree(s).
%\item Estimate Bremer support by using inverse constraints, doing five 
%independent searches for every group, holding a maximum of two trees. 
%\item Output support values for each group in parenthetical notation to \poy 
%output window.

%\end{enumerate}

%\section[Bootstrap Support]{Support II: Bootstrap Support Using Dynamic Homology}

%This tutorial builds on the previous tutorials to illustrate the calculation of 
%bootstrap frequencies. As discussed in the \poy Commands Reference, the 
%characters sampled during pseudoreplicates are entire fragments of DNA 
%sequences, not individual nucleotide characters. Tutorial 7 shows how to 
%estimate bootstrap frequencies using static homology, which allows nucleotide-
%level characters to be sampled. 

%\begin{enumerate}
%\item Type:
%    \begin{verbatim}
%    calculate_support (bootstrap: 100, build(trees:2), 
%    swap(trees:1)) [enter]
%    report (supports) [enter]
%    \end{verbatim}

%The above commands will perform the following tasks using default parameters:
%\begin{itemize}
%\item Perform 100 pseudoreplicates by sampling characters with replacement, 
%doing two independent searches for each pseudoreplicate and holding a 
%maximum of one tree. 
%\item Output bootstrap frequencies for each group in parenthetical notation to 
%\poy output window.
%\end{itemize}
%\end{enumerate}

%\section[Bootstrap Support with Static Homologies]{Support III: Bootstrap Support Using Static Homology}

%This tutorial builds on the previous tutorials to illustrate the calculation of 
%bootstrap frequencies. Here, bootstrap frequencies are obtained from analysis of 
%the implied alignment of static homologies, which permits individual nucleotide-
%level characters to be sampled instead of whole fragments, as is done using 
%dynamic homology. 
%\begin{enumerate}
%\item Type:
%    \begin{verbatim}
%    transform ((all, static_approx)) [enter]
%    calculate_support (bootstrap: 100, build(trees:2), 
%    swap(trees:1)) [enter]
%    report (supports) [enter]
%    \end{verbatim}

%The above commands will perform the following tasks using default parameters:
%\begin{itemize}
%\item Generate the alignment implied by the optimal tree and given the assumed 
%transformation costs. 
%\item Perform 100 pseudoreplicates by sampling characters with replacement, 
%doing two independent searches for each pseudoreplicate and holding a 
%maximum of one tree. 
%\item Output bootstrap frequencies for each group in parenthetical notation to 
%\poy output window.
%\end{itemize}
%\end{enumerate}

%\section[Chromosome Analysis]{Chromosome Analysis I: Unannotated Sequences}

%This tutorial illustrates the analysis of chromosome-level transformations using 
%unannotated sequences, i.e., contiguous strings of sequences without prior 
%identification of independent regions.
%\begin{enumerate}
%    \item Type
%    \begin{verbatim}
%    wipe () [enter]
%    read (chromosome:("mit5.txt"))  [enter]
%    transform ((all, dynamic_pam:(inversion:15, locus_indel:(10, 1.5), 
%    median:  3, swap_med:5, circular:true, approx:true))) [enter]
%    build (5) [enter]
%    swap ()[enter]
%    select () [enter]
%    report (asciitrees, diagnosis) [enter]
%    transform ((all, dynamic_pam:(inversion:15, locus_indel:(10, 1.5), 
%    median:3, swap_med:5, circular:true, approx:false))) [enter]
%    swap ()[enter]
%    select ()[enter]
%    report (asciitrees, diagnosis) [enter]
%    \end{verbatim}

%The above commands will perform the following tasks using default parameters:
%\begin{itemize}
%    \item Clear all data and trees from memory.
%    \item Import datafile mit5.txt.
%    \item Treat all data as pertaining to unannotated chromosome data, setting the 
%        following parameters: inversion distance and cost 15, locus indel 10 + 1.5 
%        times the length (number of nucleotides) of the locus, keep three 
%        candidate medians, swap on medians for 5 rounds, treat as circular 
%        chromosome, and use fixed-states optimization to approximate chromosome 
%        medians.
%    \item Generate 5 random addition sequence Wagner trees.
%    \item Alternate SPR and TBR branch swapping of each tree in memory.
%    \item Discard suboptimal and duplicate trees (i.e., retain only optimal trees).
%    \item Draw optimal trees in \poy Output window (ncurses version only) or output 
%        file (non-ncurses version), reporting the cost of each tree.
%    \item Output the optimal median states and edge costs.
%    \item Treat all data as pertaining to unannotated chromosome data with 
%        parameters as above but using optimization alignment (not fixed-states) to 
%        approximate chromosome medians.   
%    \item Alternate SPR and TBR branch swapping of each tree in memory.
%    \item Draw optimal trees in \poy Output window (ncurses version only) or output 
%        file (non-ncurses version), reporting the cost of each tree.
%    \item Output optimal median states and edge costs.
%\end{itemize}
%\end{enumerate}


\chapter{\poy Frequently Asked Questions}
This FAQ is supplementary documentation that aims to answer the
most frequently poised questions to the \poy developers and on the
\poy google groups.

\renewcommand{\cftdotsep}{\cftnodots} % This gets rid of the dots locally
\cftpagenumbersoff{questions} % This turns page numbers off for just the questions
\listofquestions
\newpage


%\section{Getting to know \poy}
\question{What does POY stand for?}
{POY is a meta-acronym, which comes from an older program YAPP (Yet
Another Phylogeny Program) that was written in \texttt{C}.  This
program, which was an extension of \texttt{MALIGN}, was the first
designed around direct optimization.  This program was rewritten
in Ocaml (Ocaml YAPP), which was shortened to OY.  The subsequent
parallelization of this program yielded POY.}

\question{I would like to visualize the implied alignment generated from 
my analysis, how do I do this?} 
{The implied alignment can be exported from \poy in two ways:  \\

The first is to export the implied alignment, using the command report.
The implied alignment will be output in FASTA format 
(see \nccross{implied\_alignments}{impliedalignment}).  \\

The second is to transform using static\_approx and report using
the phastwinclad option (see \nccross{phastwinclad}{phastwinclad}). 
This will produce a file Hennig86 format.\\
\\
These files can subsequently be imported into other programs such as 
Winclada or Mesquite, for visualization.}

\question{Looking at the implied alignment generated from my \poy
analysis, it looks very `gappy'. Why?} 
{POY is not, nor has it ever been an alignment program. Non-homologous, 
independent insertions are assigned their own columns with \poy, hence, 
the number of columns will expand with the number of insertions.}

%\section{Errors}
\question{I encountered a problem while running an analysis that I
think might be a bug in the program, how should I report this?}
{All error and bug reports should be made directly to the \poy Mail
Group. Before posting to this group, it is advised that the user
search the history of previously posed questions, to make sure that
it has not been answered previously.  When reported to the Mail
Group, users should include the following information: \\
\\
-- What steps will reproduce the problem; \\
-- What is the expected output and what do you see instead; \\
-- What version of the program are you using and on which operating system?}

\question{My script won't run and I don't know where I went wrong.
What should I do?}
{If a script won't run, the first thing to do is to check that there
are no hidden characters in the script file.  When constructing a
script or a transformation cost matrix, it is important to do so
in a text editor such as Notepad (for Windows), TextEdit (for Mac),
or Nano (for Linux). Generating these files in a word processing
application such as Microsoft Word may lead to the insertion of
hidden characters, which can result in an error.

Secondly, the user is advised to check the log.  If none was set,
the user is advised to do so and rerun the script.  The log will
give some indication as to which errors were encountered, or which
warnings were issued.}

\question{When I run \poy in parallel, I get multiple, identical
outputs to the screen, why?}
{It is likely that \poy was not properly compiled
in parallel. You should check the \texttt{make} options.}

\question{In running an analysis of custom alphabet characters,
with the characters being transformed to \poyargument {level} 5, I
got the following error "\texttt{seg fault:11}".  In a previous
analysis, the characters were transformed to \poyargument {level}
4, and that worked without issue.  What's wrong?} 
{This is most likely an `out of memory' error and is system specific 
and difficult to predict. This is beyond the control of the program. Storage and
set up time increase combinatorially with level number.}

\question{In running an analysis using a *.ss or Hennig86 file, I encountered 
a `syntax' error, however, the characters continued to load and the analysis seemed
to run.  Is something wrong?}
{Syntax errors are of the form: 
\\
Error: Syntax error\\
Error: Unrecognized command between characters 93 and 94 \\
Information: The file Fly.ss defines 9 static homology characters, \\
0 unaligned sequences, and 0 trees, containing 5 taxa\\
\\
Hennig86 or *.ss files, have no formal definition.  \poy does its
best to parse the file properly, but it is incumbent upon the user
to make sure that the data was read/parsed correctly.  If an error
such as this is encountered, the user should report the data
(\poyargument{report(data)}) to verify.  A similar caution should
be taken with NEXUS files, as very often files that have been
generated and exported from other programs are not in the correct
NEXUS format.}

%\section{SAQs: Stoopid Ass Questions} 
\question{My FASTA file contains sequences that are of poor 
`quality', especially in the 5 prime and 3 prime regions of the sequences.
How should these data files be prepared for analysis in \poy?}
{In cases such as this, partitioning the data is {\bf highly} recommended.  
Partitioning or fragmenting the data can help to ameliorate the effects of
poor sequences or missing data.  Moreover, when a data file contains
 sequences that were downloaded from a data base such as GenBank, 
 very often there is a lack of overlap of many of the sequences,
as different studies may have utilized different priming regions. 
How best to `chop' up your data is discussed in the \poy Heuristics chapter.}

\question{I read in a tree that was generated from a previous analysis, 
however, the cost reported in the output window of the \texttt {Interactive 
Console} is different, why is this the case?} 
{If you are reading in a tree generated from a previous analysis it is 
important to make sure that the same transformation cost matrix has 
been applied to the data. In addition, the tree must be fully resolved, 
otherwise it will be resolved arbitrarily.}

\question{Having run an analysis in \poy, I imported my tree file into 
TNT, but the tree costs are different.  Is this an error?} 
{When importing \poy tree files into another program, such as TNT, 
it is important to mirror the same `conditions' as to those in \poy
during the time of analysis, i.e. same cost matrix, gaps treated as a 
fifth state. The correct data file associated with this tree file, must also 
be imported---this corresponds to the implied alignment that is 
associated with this tree. In addition, the tree must be fully resolved, 
otherwise it will be resolved arbitrarily.}

\question{In trying to calculate Jackknife support values, I believe
that all the values are inflated for the resulting tree.  Why?}
{Although it is possible to calculate Jackknife and Bootstrap support
values for trees constructed using dynamic homology characters, it
is not recommended since resampling of dynamic characters
occurs at the fragment, rather than nucleotide, level. (Of course
this is a mute point if the dataset consists of a multitude of fragments.)
Consequently, the bootstrap and jackknife support values calculated for dynamic
characters are not directly comparable to those calculated based
on static character matrices. In order to perform character sampling
at the level of individual nucleotides, the dynamic characters {\bf
must} be transformed into static characters using \poyargument
{static\_approx} argument of the command transform (Section 3.3.26)
prior to executing calculate support.  The static\_approx is conditioned
or based on that tree.}

\question{Why is my prealigned data not treated as prealigned?} 
{By default, upon importing prealigned sequence data, the gaps are
removed and the sequences are treated as dynamic homology characters.
To preserve the alignment the data must be imported using the
\poyargument{prealigned} argument of the command \poycommand{read}.
Unless specified using the \poyargument {prealigned}, data that is
read by the program is UNALIGNED and the gaps are stripped from the
data file.}

\question{Is it possible to exclude certain terminals from an analysis?}
{The exclusion of terminals (or for that matter, characters) is easily
achieved by selecting, with the use of the identifiers.  For example
\texttt{select(terminals,not files:("Taxa\_removed.txt"))} will exclude all the 
taxa that are included in this file.  This is the inverse of 
\texttt{select(terminals,\\ files:("Taxa\_keep.txt"))}. Alternatively, if the user does not wish 
to generate a terminals file, the taxon names can be specified using
the \texttt{not names} identifier, 
e.g. \texttt{select(terminals,not names:("Taxon1","Taxon4"))}.}

\question{Is it possible to report parsimony branch lengths in \poy?}
{Yes it is. This is achieved by reporting \texttt{branches}, along with 
the \texttt{trees}, and specifying the collapse mode. For example
\texttt{report("Run1.tre", trees:(branches:single))} will report a
tree to the file \texttt{Run1.tre} with the parsimony branch lengths 
included.  The argument \texttt{single} specifies that zero length
branches will be collapsed.}

\question{I would like to import trees from an earlier run, at what
stage of the analysis should this be performed?} 
{When running a script that includes reading in trees from a previous analysis,
these trees {\bf must} be read in {\bf after} the build stage.  If
the trees are read in before the build they will be replaced by the
trees generated during the build.}

\question{Why is the root in the diagnosis file that I reported at
the end of my analysis, not the same as the root that I \poyargument{set}?}
{This is because the tree length heuristics may be based on an
alternate rooting scheme than that used for the \poyargument{newick}
or \poyargument {graphic trees} output.}

\question{What are "Numerical.linesearch; Very large slope in
optimization function" warnings?} 
{These messages normally appear in the Dynamic likelihood routines. 
They indicate that the gradient
of the parameters for the current data-set has a large absolute
slope and the numerical routine may not converge properly. Normally,
because of multiple passes of the optimization routine, we easily
break out of these regions and the routine will stabilize.

Under static likelihood data this warning message is rarely seen
and may be an issue. One should rediagnose the tree for stabilization
of the parameters of the model.}

\question{What does the "Numerical.brent; hit max number of
iterations" warning mean?} 
{This happens when the numerical routine
does not converge in a maximum number of steps. This is usually an
acceptable situation to happen as more optimization rounds will
likely occur over the data.}

\question{Why are likelihood scores worse/different than other
applications?} 
{There are a number of issues to consider. If the
values are close under the same model of evolution then numerical
issues due to finite precision arithmetic of decimal numbers can
cause slight rounding errors in an analysis to build up. Although
we use standard techniques to limit the accumulation of these errors,
they inevitably occur and small differences in likelihood scores
are absolutely normal and should not be a concern. These differences even
occur within the same application run on different architectures
and compiler options.

If the model is not hierarchical then comparisons are not relevant.
Unfortunately, the number of states in the model of evolution
matters, as well as cost assignments of Maximum Parsimonious
Likelihood (MPL) and Maximum Average Likelihood (MAL). Thus, a four
state model of evolutions' likelihood score cannot be compared
directly with a five state model. There is added cost that result
from the probability of an additional state in an analysis.

One should also check that the \poycommand{exhaustive} option is
set for the optimization routines.  It is set by default but ensure
that, if it changed somewhere in the script, one sets it back--especially 
if one plans to do application comparisons.}


\addcontentsline{toc}{section}{Bibliography}%%
\bibliography{doc/poylibrary}
\bibliographystyle{plain}

%\addtocontents{toc}{General Index}%%
\printindex{general}{General Index}
\printindex{poy3}{POY 3.0 Commands Index}
\input{doc/poy3commands}%%

\end{document}
